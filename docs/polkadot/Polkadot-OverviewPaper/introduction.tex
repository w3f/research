\section{Introduction}\label{sec:intro}
Until recently, most of the applications on the web were controlled in a centralised fashion.
We refer not to the centralisation of physical infrastructure, which is often economically efficient, but rather to the logical centralisation of the ability to deploy, take down, alter, and change the rules of the application.
Two good examples are Google and Facebook: while they do have servers all around the world, they are each ultimately controlled by a single legal entity.

Giving a central entity control over a system comes with risks for users and application developers.
The central entity can stop the service at any moment, can sell users' data depending on the jurisdictions they provide service to and manipulate how the service is working without user consent.
With the need for security and the urge for more freedom and fairness, the era of decentralised web applications, where no single entity can control the system, is emerging.
This decentralised web is comprised of many types of applications such as \handan{games, storage, decentralised exchanges, auctions, financial systems, etc.}{ we may add some references for each applications.}

One of the fundamental challenges of the decentralised web is keeping state. There is no central entity who follows the current state or can decide what the current valid state is if there is any doubt.
Blockchains are one of the technologies who address this problem.

In order to make the decentralised web usable for end-users, these separate blockchains need to interact, otherwise, each will become isolated and not adopted by as many users, and the \handan{sum}{replace: overall?} functionality of the system will be insufficient to compete with the centralised web. \handan{However, to interact with different chains, we need to build in interoperability mechanisms, which introduces challenges. Many of these challenges arise from the ways that different blockchains have different technical characteristics.}{Therefore, interoperability mechanisms are urgent for blockchain systems but such mechanisms bring along new challenges. Many of these challenges arise because of different technical infrastructures and functionalities that blockchains have.} For example, \handan{Bitcoin and Ethereum}{refs} are proof-of-work (PoW) blockchains where \handan{preventing the control of the whole system by one entity}{replace: decentralization?} is carried out by introducing \handan{challenges}{maybe puzzle or something else is better than `challenges' not to confuse with the challenges that we talked one sentence before} that need a large amount of \handan{processing}{maybe computing?} power to solve. Their security relies on whether this amount of processing power exceeds the amount of processing power any single entity would possibly have. An alternative to PoW are proof-of-stake (PoS) systems, where the entities who control the system have to lock a large amount of funds and would be punished if they misbehave in any way.
The security relies on the fact that the total amount of locked funds exceeds the budget any adversary is able to invest in the attack.
These technical differences present difficulties for one blockchain to be able to trust another.

\handan{}{this paragraph looks like repetition of the previous one. Maybe it is better to combine them.}Currently, there have been hundreds of chains implemented in the wild for various functions and non-compatible underlying technology.
Most one of these systems have different properties they aim to achieve and certain security threshold that they set up.
Hence, there is a need for a system that can connect these heterogeneous chains.
Moreover, having multiple security setups causes a split in the security these systems can provide. Gathering all this security power and have a shared security system increases security substantially.

\handan{}{we need to improve this paragraph}In this paper, we introduce Polkadot, a multi-chain system with shared security guarantees.
Polkadot consists of the main chain called the \emph{Relay Chain} that guarantees the security and multiple heterogeneous parallel chains called \emph{Parachains}.
The security goal of Polkadot is to be Byzantine fault-tolerant. Polkadot enables the parachains to communicate together and have shared security.

\handan{Furthermore, scalability is another challenge for blockchains to make them comparable with centralised services in functionality.}{it may be better to talk about it when we mention challenges of interoperability}
Polkadot's hierarchical structure addresses this challenge.

\subsection{Comparison with other multi-chain systems}\label{sec:comparison}

\paragraph{Cosmos} 

Similar to Polkadot, Cosmos is a system which aims as solving blockchain interoperability problem to improve scalability. In this sense, there are many similarities between the two systems such as components which play similar roles as sub-components of Polkadot, For example Cosmos's Hub roles resembles' of Polkadot Relay chain. Or similar to Polkadot parachain, Cosmos's zone are the blockchains which use the Hub to communicate. There are however significant differences between the two systems. 

Most importantly while Polkadot system as whole is a sharded state machine (See Section \ref{sec:relaychain}), Cosmos does not attempt to unify the state among zones and therefore individual zone's state is not reflected in the Hub's state. As the result, unlike Polkadot, Cosmos does not offer shared security among the zones. In such a system, the cross-chain messages are no longer trust-less, That is to say that a receiver zone need to fully trust the sender zone in order to act upon messages it receives from the sender. If one consider Cosmos system as whole including all zones in a similar way one analyzes the Polkadot system, the security of such system is equal to the security of least secure zone. Similarly the security promise of Polkadot guarantees that validated parachain data are available in later time for retrieval and audit (See Section \ref{sec:validity-and-availability}). In the case Cosmos, users are ought to trust the Zone operators to keep the history of the chain state.

It is noteworthy that using SPREE modules, Polkadot offers even stronger security than shared security (See Section \ref{sec:spree}. When a parachain signs up for a SPREE module, Polkadot guarantees that certain XCMP messages received by that parachain are being processed by the  pre-defined SPREE module set of code. No similar cross-zone trust framework is offered by Cosmos system.

Another significant difference between Cosmos and Polkadot is that on how blocks are produced and finalized. In Polkadot because all parachain states are strongly connected to relay chain states, parachain can temporarily fork alongside the relay chain. This allows for block production to decouple from finality logic. In this sense, Polkadot blocks can be produced over unfinalized blocks and multi blocks can be finalized at once. One the other hand, Cosmos zone depends on instant finality of Hub states to perform cross-chain openation and therefore delayed finalization halts the cross zone operations.


%The paper is organised as follows. In Section~\ref{sec:properties} we review the properties that Polkadot is aiming for and Section~\ref{sec:preliminiary} examines the structural components and roles that we defined for Polkadot.
%Section~\ref{sec:components} presents all the underlying components including protocols and primitives that have been designed for Polkadot as well as a note about incentives and economics.
