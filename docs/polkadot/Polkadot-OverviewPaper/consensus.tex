\subsection{Consensus}\label{sec:grandpa}



%In our consensus protocols, we assume that a message sent by a validator arrives to other validators at most $\D$ times later where $\D$ is an unknown parameter. So, validators are in a partially synchronous network that guarantees  eventual delivery.
\subsubsection{Blind Assignment for Blockchain Extension (BABE)}

In Polkadot, we produce relay chain blocks using our Blind Assignment for Blockchain Extension protocol, abbreviated BABE. BABE assigns validators randomly to blocks production slots using  the randomness generated with blocks. These assignments are completely private until the assigned parties produce their blocks. Therefore, we use ``Blind Assignment'' in the protocol name. 

We may have slots without any assignments. We call these slot as empty slot. In order to fill the empty slots, we have secondary block production mechanism based on Aura. We note that these blocks do not contribute the security of BABE. 

Next, we describe BABE together with its security properties. Then, we explain the secondary block production mechanism and discuss about its contribution on BABE.

\paragraph{BABE:}









\subsubsection{GRANDPA} \label{sec:grandpa}
\begin{figure}[h!]
  \centering
  \includegraphics[width=0.4\textwidth]{images/Grandpa.jpg}
  \caption{GRANDPA votes and how they are aggregated.}
    \label{fig:grandpa}
\end{figure}
For the relay chain consensus, we want provably finality, which makes bridges to chains outside Polkadot easier, and we also want the ability to revert chains in the short term if parachain blocks are unavailable for invalid.
So we combine a block production mechanism BABE with the ability to fork and that gives eventual consensus with a finality gadget GRANDPA, that can finalise blocks and provide proofs that they are final.
BABE is similar to Ourobors Praos[citation needed] but with a fixed set of validators and improved resilience to clock drift.
It works by dividing time into slots in each of which validators computing a Verifiable Random Function(VRF) which will decide whether they have the ability to produce a valid block in that slot.
For fork-choice, we go with the longest chain that includes all blocks finalised by GRANDPA.
