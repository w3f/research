\subsection{Consensus}\label{sec:consensus}

In this section, we explain the hybrid consensus protocol of Polkadot which consists of BABE: a block production mechanism of the relay chain that provides probabilistic finality and GRANDPA which provides provable, deterministic finality and works independently from BABE.  Informally, probabilistic finality implies that after certain time \eray{passed}{passes}, a block in the relay chain will be finalized with very high probability (close to 1) and deterministic finality implies a finalized block stays final forever. Furthermore, provable finality means that  we can prove to parties not actively involved in the consensus that a block is final.

We need provable finality to make bridges to chains outside Polkadot easier and for that we need Byzantine agreement. However, the validity and availability scheme may also require us to revert blocks, which would mean that getting Byzantine agreement on every block, as in Tendermint or Algorand, would not be suitable. However, this should happen rarely as a lot of stake will be slashed when we do this. As a result, we want a scheme that generates blocks and optimistically executes them, but it may take some time to finalise them. 

Because of the way XCMP works, message passing speed is constrained by block time, but not by finality time, so if we delay finality but in the end do not revert, then message passing is fast.
\eray{Even the speed at \eray{whoch}{which} we finalise blocks may \eray{be variable}{vary} - if we do not recieve reports of invalidity and unavailability then we can finalise fast, but if we do then we may need to delay finality while we execute more involved checks.}{-comment:?? This paragraph needs to be rewritten as it is not clear what's meant here.-}

As a result of these requirements, we have chosen to seperate the mechanisms for block production and finalising blocks as much as possible. In the next \eray{twos ections}{two sections}, we describe the protocols BABE and GRANDPA that \eray{do each of these}{-comment:??-}.

%In our consensus protocols, we assume that a message sent by a validator arrives to other validators at most $\D$ times later where $\D$ is an unknown parameter. So, validators are in a partially synchronous network that guarantees  eventual delivery.
\subsubsection{Blind Assignment for Blockchain Extension (BABE)}

In Polkadot, we produce relay chain blocks using our Blind Assignment for Blockchain Extension protocol, abbreviated BABE. BABE assigns validators randomly to blocks production slots using  the randomness generated with blocks. These assignments are completely private until the assigned parties produce their blocks. Therefore, we use ``Blind Assignment'' in the protocol name. 

We may have slots without any assignments. We call these slot as empty slot. In order to fill the empty slots, we have secondary block production mechanism based on Aura. We note that these blocks do not contribute the security of BABE. 

Next, we describe BABE together with its security properties. Then, we explain the secondary block production mechanism and discuss about its contribution on BABE.

\paragraph{BABE:}









\subsubsection{GRANDPA} \label{sec:grandpa}
\begin{figure}[h!]
  \centering
  \includegraphics[width=0.4\textwidth]{images/Grandpa.jpg}
  \caption{GRANDPA votes and how they are aggregated. \eray{}{-comment:The figure and fonts are too small, make it bigger-}}
    \label{fig:grandpa}
\end{figure}

As mentioned above, we want a finalisation mechanism that is flexible and seperated from block production, which is achieved by GRANDPA. The only modofication to BABE required for it to work with GRANDPA is to change the fork-choice rule: instead of building on the longest chain, a validitor producing a block should build on the longest chain including all blocks that it sees as finalised. GRANDPA can work with many different block production mechanisms and it should be possible to switch out BABE with another.

Intuitively GRANDPA is a Byzantine agreement rotocol that works to agree on a chain, out of many possible forks, by following some simpler fork choice rule, which together with the block production mechainism would give probabilistic finality if GRANDPA itself stopped finalising blocks. We want to be able to agree on many blocks at once, in contrast to single-block Byzantine agreement protocols.

We assume that we can ask the fork choice rule for the best block given a particular block. 

The basic idea is that we want to agree by Byzantine agreement on the prefix of the chain that everyone agrees on. To make this more robust, we try to agree on the prefix of the chain that 2/3 of validators agree on.

We make use of a GHOST on votes rule, much like Casper TFG or some of the fork choice rules suggested for use with Casper FGG. We use this rule inside what is structured like a more traditional Byzantine agreement protocol, to process votes. The 2/3 GHOST (pictured) rule works as follows. We have a set of votes, given by block hashes.  in which honest validatoprs should not have more than one vote, and we take the head of the chain formed inductively as follows. We start with the genesis block and then include the child of that block that 2/3 of the voters voted for descendents of, as long as there is exactly one such child. The head of this chain is $g(V)$ where $V$ is the set of votes.
