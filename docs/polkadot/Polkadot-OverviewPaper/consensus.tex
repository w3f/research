\subsection{Consensus}\label{sec:consensus}

In this section, we explain the hybrid consensus protocol of Polkadot which consists of BABE: a block production mechanism of the relay chain that provides probabilistic finality and GRANDPA which provides provable, deterministic finality and works independently from BABE.  Informally probabilistic finality implies that after certain time passes, a block in the relay chain will be finalised with very high probability (close to 1) and deterministic finality implies a finalised block stays final forever. Furthermore provable finality means that we can prove to parties not actively involved in the consensus that a block is final.

We need provable finality to make bridges to chains outside Polkadot easier; another blockchain, not part of Polkadot's consensus could then be convinced of when it is safe to act on data in a relay chain or parachain block without any danger of it being reverted. The best way of getting that is to have Byzantine agreement among the validators on the state of Polkadot and its parachains. However the availability and validity scheme \ref{sec:validity-and-availability} may also require us to revert blocks, which would mean that getting Byzantine agreement on every block, as in Tendermint \cite{Tendermint} or Algorand \cite{ALGORAND}, would not be suitable. However, this should happen rarely as a lot of stake will be slashed when we do this. As a result, we want a scheme that generates blocks and optimistically executes them, but it may take some time to finalise them. Thus GRANDPA voters need to wait for assurances of availability and validity of a block before voting to finalise that block.
Even the speed at which we finalise blocks may vary - if we do not receive reports of invalidity and unavailability then we can finalise fast, but if we do then we may need to delay finality while we execute more involved checks. 

Because of the way Polkadot's messaging protocol (XCMP \ref{sec:XCMP}) works, message passing speed is constrained by block time, but not by finality time. Thus if we delay finality but in the end do not revert, then message passing is still fast.

As a result of these requirements, we have chosen to separate the mechanisms for block production and finalising blocks as much as possible. In the next two sections, we describe the protocols BABE and GRANDPA that do these respectively.

%In our consensus protocols, we assume that a message sent by a validator arrives to other validators at most $\D$ times later where $\D$ is an unknown parameter. So, validators are in a partially synchronous network that guarantees  eventual delivery.
\subsubsection{Blind Assignment for Blockchain Extension (BABE)}

In Polkadot, we produce relay chain blocks using our Blind Assignment for Blockchain Extension protocol, abbreviated BABE. BABE assigns validators randomly to blocks production slots using  the randomness generated with blocks. These assignments are completely private until the assigned parties produce their blocks. Therefore, we use ``Blind Assignment'' in the protocol name. 

We may have slots without any assignments. We call these slot as empty slot. In order to fill the empty slots, we have secondary block production mechanism based on Aura. We note that these blocks do not contribute the security of BABE. 

Next, we describe BABE together with its security properties. Then, we explain the secondary block production mechanism and discuss about its contribution on BABE.

\paragraph{BABE:}









\subsubsection{GRANDPA} \label{sec:grandpa}


As mentioned above, we want a finalisation mechanism that is flexible and separated from block production, which is achieved by GRANDPA. The only modification to BABE required for it to work with GRANDPA is to change the fork-choice rule: instead of building on the longest chain, a validator producing a block should build on the longest chain including all blocks that it sees as finalised. GRANDPA can work with many different block production mechanisms and it will be possible to switch out BABE with another.

Intuitively GRANDPA is a Byzantine agreement protocol that works to agree on a chain, out of many possible forks, by following some simpler fork choice rule, which together with the block production mechanism would give probabilistic finality if GRANDPA itself stopped finalising blocks. We want to be able to agree on many blocks at once, in contrast to single-block Byzantine agreement protocols.

We assume that we can ask the fork choice rule for the best block given a particular block. The basic idea is that we want to reach Byzantine agreement on the prefix of the chain that everyone agrees on. To make this more robust, we try to agree on the prefix of the chain that 2/3 of validators agree on.

\begin{figure}[h!]
  \centering
  \includegraphics[width=0.7\textwidth]{images/Grandpa.png}
  \caption{GRANDPA votes and how they are aggregated (image credit: Ignasio Albero).}
    \label{fig:grandpa}
\end{figure}

We make use of a Greedy Heaviest Observed Subtree (GHOST) on votes rule, much like Casper TFG \cite{CasperTFG} or some of the fork choice rules suggested for use with Casper FFG \cite{CasperFFG}. We use this rule inside what is structured like a more traditional Byzantine agreement protocol, to process votes. The 2/3 GHOST rule (pictured in Figure \ref{fig:grandpa})  works as follows. We have a set of votes, given by block hashes in which honest validators should not have more than one vote, and we take the head of the chain formed inductively as follows. We start with the genesis block and then include the child of that block that 2/3 of the voters voted for descendants of, as long as there is exactly one such child. The head of this chain is $g(V)$ where $V$ is the set of votes. For example, in Figure \ref{fig:grandpa}, the left hand side gives the votes for individual blocks and the right hand side the total votes for each block and all of its descendants. The genesis block is at the top and we take its child with 100\% $> 2/3$ of the votes. The children of that block have 60\% and 40\% of the votes respectively and since these are below $2/3$ we stop and return the second block.

There are two voting phases in a round of GRANDPA: prevote and precommit. Firstly validators prevote on a best chain. Then they apply the 2/3-GHOST rule, $g$, to the set of prevotes $V$ they see and precommit to $g(V)$. Then similarly they take the set of precommits $C$ they see and finalise $g(C)$.

To ensure safety, we ensure that all votes are descendants of any block that could possibly have been finalised in the last round. Nodes maintain an estimate of the last block that could have been finalised in a round, which is calculated from the prevotes and precommits. Before starting a new round, a node waits until it sees enough precommits for it to be sure that no block on a different chain or later on the same chain as this round's estimate can be finalised. Then it ensures that it only prevotes and precommits in the next round to blocks that are descendants of the last round's estimate which it keeps updating by listening to precommits from the last round. This ensures safety.

To ensure liveness, we select one validator in rotation to be the primary. They start the round by broadcasting their estimate for the last round. Then when validators prevote, if the primary's block passes two checks, that it is at least the validator's estimate and that it got $>2/3$ prevotes for it and its descendants in the last round, then it prevotes for the best chain including the primary's block. The idea here is that if the primary's block has not been finalised, then progress is made by finalising the block. If the primary's block has not been finalised and all validators agree on the best chain including the last finalised block, which we should do eventually because BABE gives probabilistic finality on its own, then we now make progress by finalising that chain.
  


