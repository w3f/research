Ethereum 2.0 promises to move from proof of work to proof-of-stake consensus mechanism and improve the speed and the throughput via sharding. Therefore, we compare Ethereum 2.0 sharding with Polkadot. Shards in Ethereum 2.0  are homogeneous blockchains while parachains are independent heterogeneous blockchains whose security is connected to the relay chain. Hence, Polkadot provides a more generic sharding solution in this sense. 

Ethereum 2.0 has validators as Polkadot where anyone staking a certain amount of ether can become a validator. In Polkadot, a certain number of validators are elected in every era with NPOS (See \S~\ref{sec:validators}) by nominators. NPoS allows for virtually all DOT holders to continuously participate, thus maintaining high levels of security by putting more value at stake and allowing more people to earn a yield based on their holdings. Besides, allowing all DOT holders to participate to NPOS and limiting the number of participants to be selected with NPOS in every era to be a validator ensure better scalability than Ethereum 2.0.

Ethereum 2.0 provides the availability of transactions via erasure codes as Polkadot but it needs 2D erasure code to reduce the validity proof size. The availability and validity protocol \cite{availabilityETH2} of Ethereum 2.0 has a different logic than Polkadot's protocol (See \S~\ref{sec:validity-and-availability}). Validators in Ethereum 2.0 are assigned to each shard for attesting block of shards as parachain validators in Polkadot thus constitute the committee of the shard. The committee members receive a Merkle proof of randomly chosen code piece from a full node of the shard and verify them. If all pieces are verified and no fraud-proof is announced, then the block is considered as valid. The security of this scheme is based on having an honest majority in the committee while the security of Polkadot's scheme based on having at least one honest validator either among parachain validators or secondary checkers (See \S~\ref{sec:validity-and-availability}). Therefore, the committee size in Ethereum 2.0 is considerably large comparing to the size of parachain validators in Polkadot. 

The beacon chain in Ethereum 2.0 is a proof-of-stake protocol as Polkadot's relay chain. Similarly, it has a finality gadget called Casper \cite{CasperFFG,CasperCBC} as GRANDPA in Polkadot. Casper also combines  eventual finality and  Byzantine agreement as GRANDPA but GRANDPA gives better liveness property than Casper \cite{2018:Stewart:Grandpa}.