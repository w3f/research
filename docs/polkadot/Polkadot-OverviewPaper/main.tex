\documentclass{article}
%=================Begin:Packages===================
\usepackage[english]{babel}
\usepackage{mathtools}
\usepackage{hyperref}

\mathtoolsset{showonlyrefs=true}
\addto\extrasenglish{ \renewcommand{\sectionautorefname}{\S} \let\subsectionautorefname\sectionautorefname \let\subsubsectionautorefname\sectionautorefname }
% The lines above allow for the option to use \autoref{}, which adds the section symbol § next to the section number.

\usepackage[margin=1.3in]{geometry}


\usepackage{graphicx}
\usepackage{bmpsize}

\usepackage{enumitem}
\usepackage{url}

\usepackage{amsfonts}
\usepackage{amsmath}
\usepackage{hyperref}
\usepackage{xargs}                      % Use more than one optional parameter in a new commands
\usepackage[pdftex,dvipsnames]{xcolor}  % Coloured text etc.
%
\usepackage[colorinlistoftodos,prependcaption,textsize=tiny]{todonotes}
\usepackage[normalem]{ulem}
\usepackage{longtable}
\sloppy
%=================End:Packages===================

%=================Begin:Macros===================


%*** To comment a text use e.g. \yourname{the text that you comment}{ your comment}
\newcommandx{\fatemeh}[2]{{\uline{#1}}\textcolor{pink}{(#2)}}
\newcommandx{\handan}[2]{{\uline{#1}}\textcolor{red}{(#2)}}
\newcommandx{\eray}[2]{{\uline{#1}}\textcolor{blue}{(#2)}}
\newcommandx{\jeff}[2]{{\uline{#1}}\textcolor{green}{(#2)}}
\newcommandx{\al}[2]{{\uline{#1}}\textcolor{cyan}{(#2)}}
\newcommandx{\sergey}[2]{{\uline{#1}}\textcolor{gray}{(#2)}}
\newcommandx{\alfonso}[2]{{\uline{#1}}\textcolor{purple}{(#2)}}
\newcommandx{\syed}[2]{{\uline{#1}}\textcolor{orange}{(#2)}}
\newcommandx{\ximin}[2]{{\uline{#1}}\textcolor{red}{(#2)}}
\newcommandx{\leon}[2]{{\uline{#1}}\textcolor{brown}{(#2)}}

%***To uncomment all the comments comment out the above macros and replace with below

%\newcommandx{\fatemeh}[2]{{#1}{}}
%\newcommandx{\handan}[2]{{#1}{}}
%\newcommandx{\eray}[2]{{#1}{}}
%\newcommandx{\jeff}[2]{{#1}{}}
%\newcommandx{\al}[2]{{#1}{}}
%\newcommandx{\sergey}[2]{{#1}{}}
%\newcommandx{\alfonso}[2]{{#1}{}}
%\newcommandx{\syed}[2]{{#1}{}}

\newcommand{\Nom}{\ensuremath{\mathcal{N}}} % Set of nominators
\newcommand{\Val}{\ensuremath{\mathcal{V}}} % Set of elected validators
\newcommand{\nval}{\ensuremath{n_{val}}} % Number of validators to elect
\newcommand{\Can}{\ensuremath{\mathcal{C}}} %Set of candidate validators
\newcommand{\nom}{\ensuremath{n}} % A nominator
\newcommand{\val}{\ensuremath{v}} %A validator
\newcommand{\col}{\ensuremath{c}} %A collator
\newcommand{\Par}{\ensuremath{P}} %A particular parachain
\newcommand{\Col}{\ensuremath{\mathcal{C}}} %Set of collators
\newcommand{\slot}{\ensuremath{sl}} %slot number
\newcommand{\ep}{\ensuremath{e}} %epoch number
\newcommand{\lclock}{\ensuremath{\mathsf{clock}}} %local clock
\newcommand{\block}{\ensuremath{B}} % a block
\newcommand{\bchain}{\ensuremath{\bar{B}}} %blockchain
\newcommand{\D}{\ensuremath{\Delta}}
\newcommand{\sk}{\ensuremath{\mathsf{sk}}} %vrf secret key
\newcommand{\pk}{\ensuremath{\mathsf{pk}}} %vrf public key

%OTHER NOTATIONS WITHOUT MACROS
%\Val_\Par -> Validator set for a particular parachain \par

%=================End:Macros=====================

%=================Begin:Definitions===================
%Please add you variable definitions here
%=================End:Definitions===================



\title{Overview of Polkadot and its Design Considerations}

\author{
  Jeff Burdges$^1$, Alfonso Cevallos$^1$, Peter Czaban$^1$\\
  \and
  Rob Habermeier$^2$, Syed Hosseini$^1$, Fabio Lama$^1$,\\
  \and
  Handan K{\i}l{\i}n\c{c} Alper$^1$, Ximin Luo$^1$, Fatemeh Shirazi$^1$, \\
  \and
  Alistair Stewart$^1$, Gavin Wood$^{1,2}$\\
  \\
  \texttt{$^1$ Web3 Foundation},\\
  \texttt{$^2$ Parity Technologies}
%  \texttt{firstname@web3.foundation},
%\texttt{firstname@parity.io}
}



\begin{document}

\maketitle

\begin{abstract}
In this paper we describe the design components of the heterogenous multi-chain protocol Polkadot and explain how these components help Polkadot address some of the existing shortcomings of blockchain technologies.
Currently, there a vast number of blockchain projects have been introduced and employed with various features that are not necessarily designed to work  with each other.  This makes it difficult for users to utilise a large number of applications on different blockchain projects. Moreover, with the increase in number of projects the security of each one is providing individually becomes weaker.
Polkadot aims to provide a scalable and interoperable framework for multiple chains with shared security that is achieved by the collection of components described in this paper.
\end{abstract}

%\newpage
\tableofcontents
\newpage
\section{Introduction}\label{sec:intro}
Internet was originally designed as a collection of interconnected decentralised entities which is noticeable by the design of its early protocols such as TCP/IP and SMTP. However, its commercialisation has led to the centralisation of most of the applications present on the web. We refer not to the centralisation of physical infrastructure, which is often economically efficient, but rather to the logical centralisation of the ability to deploy, take down, alter, and change the rules of the application. Two prominent examples are giant corporations like Google and Facebook: while they do have run/maintain servers all around the world, these are ultimately controlled by a single legal entity. Giving a central entity control over a system poses several risks as much for the users as for the application developers. For example, the central entity can stop the service at any moment, can sell users' data to third parties depending on the jurisdictions they provide service to, and manipulate how the service is working without the users consent.

With all this current development around corporative interests related to personal data and applications ownership, there is a current countermovement and a growing need for a better security, freedom and fairness for the users on the web, and with this - for more decentralised applications, where no single entity can control the whole system. The process of decentralisation is not new. It has been used in a number of areas of the development of web and other systems, starting with the free software movement in the 1970s. Today, we can see decentralisation practices in different areas of the online development such as as games \cite{}.

One of the fundamental challenges of developing a decentralised application is storing and communicating the current state of the application. That is because there is no central entity who follows the current state or can decide what the current valid state is if there is any doubt. Blockchains are technologies which are proposed to address this problem.

The decentralised web can only compete with the centralised web if it becomes usable for end-users. This means, often separate blockchain applications need to interact, otherwise, each will become isolated and not adopted by as many users. However, to interact with different chains, one needs to build an interoperability mechanism, which introduces challenges. Many of these challenges which are missing in the centralised model arise because of the fundamental differences in the trust model between the two paradigms. 
%The simple authentication mechanisms such as OAuth \cite{hardt2012oauth} can not easily be adapted to the decentralised web where there is no central entity guarding the secret key. 
Furthermore, it is not straight forward for decentralised entities to establish trust between each other when they use different mechanisms to ensure the security of their respective systems. For example, Bitcoin\cite{} and Ethereum\cite{} are proof-of-work (PoW) blockchains where decentralisation is carried out by introducing puzzles that need a large amount of computing power to solve. Their security relies on whether this amount of processing power exceeds the amount of processing power any single entity would possibly have. An alternative to PoW is a proof-of-stake (PoS) system, where the entities that control the system have to lock a large amount of funds and would be punished if they misbehave in any way. The security relies on the fact that the total amount of locked funds exceeds the budget any the adversary is able to invest in the attack. These technical differences present difficulties for one blockchain to be able to trust another.

Currently, there have been hundreds of chains implemented in the wild for various functions and non-compatible underlying technology. Most of these systems have different properties they aim to achieve and a certain security threshold that they set up. Having multiple security setups causes a substantial split in the security all these systems can provide. Another challenge that blockchain technologies need to tackle is scalabiltiy. Both Bitcoin and Ethereum (Eth1.0) do not scale well that leads to high latency where they can only have tens of transactions per second \cite{scaling} \cite{}. Whereas credit card companies such as Mastercard or Visa carry out thousands to transactions per second \cite{}. While there have been a number of solutions suggested to tackle scalability at the high level the prominent solutions to the scalability problem of block chain technologies is having sharded chains \footnote{Sharding refers to the use of a second layer chains that are secured by the main chain and abstracting away transactions from the main chain} \cite{sharding}, which is adopted by Polkadot. %and enabling off-chain transactions \cite{gudgeon2019sok} where only the final balance is recorded on the blockchain.   
In this paper, we introduce Polkadot, a multi-chain system that aims at gathering the security power of all these chains to have a shared security system. Polkadot utilises a central chain called the \emph{Relay Chain} which communicates with multiple heterogeneous and independent sharded chains called \emph{Parachains} (portmanteau of parallel chains). The Relay chain is responsible to provide shared security for all parachains. Another goal of Polkadot is providing trust-free interchain transactability among parachains. By featuring these points the shortcomings of blockchain technologies that Polkadot is addressing are interoperability, scalablility, and split security .

\paragraph{Paper Organisation} In the next section, we review preliminary information such as description of roles of Polkadot's participants and our adversary model. We then continue to review all sub-protocols, aka components, of Polkadot and low-level cryptographic and networking primitives. We conclude with a summary of the described components and explain what they try to achieve \ref{sec:summary}. We provide a glossary and comparison to other multi-chain system in the appendices. 

%\syed{Polkadot enables the parachains to communicate together and have shared security.}{So we do not need to repeat this}

%\handan{Furthermore, scalability is another challenge for blockchains to make them comparable with centralised services in functionality.}{it may be better to talk about it when we mention challenges of interoperability}\syed{}{I agree with Handan, specially that scalability is partially related to interoperability, for example, the reason that Ethereum tried to be generic computer is because it wanted to have all Dapps lives inside so they can communicate efficiently which resulted in scalability problem. But we also can talk much more about scalability }
%Polkadot's hierarchical structure addresses this challenge.


%The paper is organised as follows. In Section~\ref{sec:properties} we review the properties that Polkadot is aiming for and Section~\ref{sec:preliminiary} examines the structural components and roles that we defined for Polkadot.
%Section~\ref{sec:components} presents all the underlying components including protocols and primitives that have been designed for Polkadot as well as a note about incentives and economics.


\section{Polkadot in Practice}
KUSAMA
Polkadot

\section{Preliminaries}\label{s:prel}

Throughout the paper we consider the following approval-based multiwinner election instance. 
We are given a bipartite approval graph $G=(N\cup C, E)$ where $N$ is a set of voters and $C$ is a set of candidates. 
We are additionally given a vector $s\in\R^N$ of vote strengths, where $s_n$ is the strength of $n$'s vote, and a target number $k$ of candidates to elect, where $0< k<|C|$.
For each voter $n\in N$, $C_n:=\{c\in C: \ nc\in E\}$ represents her approval ballot, i.e.~the subset of candidates that $n$ approves of, and for each candidate $c\in C$ we denote by $N_c:=\{n\in N: \ nc\in E\}$ the set of voters approving $c$, where $nc$ is shorthand for edge $\{n,c\}$. 
To avoid trivialities, we assume that graph $G$ in the input has no isolated vertices. 
For any $c\in C\setminus A$, we write $A+c$ and $A-c$ as shorthands for $A\cup\{c\}$ and $A\setminus \{c\}$ respectively. 

\emph{Proportional justified representation (PJR).} 
The PJR property was introduced in~\cite{sanchez2017proportional} for voters with unit vote strength. We present its natural generalization to arbitrary vote strengths. A committee $A\subseteq C$ of $k$ members satisfies PJR if there is no group $N'\subseteq N$ of voters and integer $0<r\leq k$ such that:
%\begin{itemize}
%\item[a)] $\sum_{n\in N'} s_n \geq \frac{r}{k} \sum_{n\in N}s_n$,
%\item[b)] $|\cap_{n\in N'} C_n|\geq r$, and
%\item[c)] $|A\cap (\cup_{n\in N'} C_n)|<r$.
%\end{itemize}
%
$$\text{a) } \sum_{n\in N'} s_n \geq \frac{r}{k} \sum_{n\in N}s_n, \quad \quad \text{b) } |\cap_{n\in N'} C_n|\geq r, 
\quad \quad \text{and c) } |A\cap (\cup_{n\in N'} C_n)|<r.$$
%
In words, if there is a group $N'$ of voters with at least $r$ commonly approved candidates, and enough aggregate vote strength to provide each of them with a vote support of value $\hat{t}:=\sum_{n\in N} s_n / k$, then this group has a justified right to be represented by at least $r$ members in committee $A$, though not necessarily commonly approved. 
Notice that $\hat{t}$ is an upper bound on the average vote support that voter set $N$ can possibly provide to any committee of $k$ members. 

\emph{Maximin support objective.} 
For the given instance, we consider a solution consisting of a tuple $(A,w)$, where $A\subseteq C$ is a committee of $k$ elected candidates, and $w\in\R^E$ is a vector of non-negative edge weights that represents a fractional distribution of each voter's vote among her approved candidates.%
\footnote{This weight vector is related to the notions of \emph{support distribution function} in~\cite{sanchez2016maximin} and \emph{price system} in~\cite{peters2019proportionality}. In particular, all voting rules considered in this paper are \emph{priceable}, as defined in~\cite{peters2019proportionality}.} 
For instance, for voter $n$ this distribution may assign a third of $s_n$ to $c_1$ and two thirds of $s_n$ to $c_2$, where $c_1, c_2\in C_n$.
Vector $w$ is considered \emph{feasible}%
\footnote{Intuitively, a feasible solution $(A,w)$ should also observe $w_{nc}=0$ for each edge $nc$ with $c\not\in A$. 
However, as this constraint can always be enforced in post-computation, we ignore it so that the feasibility of a vector $w$ is independent of any committee.} 
if  % 
%
\begin{equation}
    \sum_{c\in C_n} w_{nc}\leq s_n \quad \text{ for each voter } n\in N. \label{eq:feasible}
\end{equation}

In our analyses, we will also consider \emph{partial} committees, with $|A|\leq k$. If $|A|=k$, we call it \emph{full}. 
All solutions $(A,w)$ in this paper are assumed to be feasible and full unless stated otherwise. 
Given a (possibly partial, unfeasible) solution $(A,w)$, we define the \emph{support} over the committee members as 
\begin{equation}
supp_w(c):=\sum_{n\in N_c} w_{nc} \quad \text{for each $c\in A, \quad$ and } \quad supp_w(A):=\min_{c\in A} supp_w(c), \label{eq:support}
\end{equation}
where we use the convention that $supp_w(\emptyset)=\infty$ for any weight vector $w\in\R^E$. 
The maximin support objective, introduced in~\cite{sanchez2016maximin}, asks to maximize the least member support $supp_w(A)$ over all feasible full solutions $(A,w)$. 

\emph{Balanced solutions.}
For a fixed committee $A$, a feasible weight vector $w\in\R^E$ that maximizes $supp_w(A)$ can be found efficiently. In this paper we seek additional desirable properties on a weight vector which can still be achieved efficiently. We say that a feasible $w\in\R^E$ is \emph{balanced for $A$}, or that $(A,w)$ is a balanced solution, if
\begin{enumerate}
    \item it maximizes the sum of member supports, $\sum_{c\in A} supp_w(c)$, over all feasible weight vectors, and 
    \item it minimizes the sum of supports squared, $\sum_{c\in A} (supp_w(c))^2$, over all vectors that observe the point above. 
\end{enumerate}

In other words, a balanced weight vector maximizes the sum of supports and then minimizes their variance. 
In the next lemma, whose proof is delayed to Appendix~\ref{s:proofs}, we establish some key properties that we exploit in our analyses. 

\begin{lemma}\label{lem:balanced}
Let $(A,w)$ be a balanced, possibly partial solution. Then,
\begin{enumerate}
    \item for each $1\leq r\leq |A|$, vector $w$ simultaneously maximizes $\min_{A'\subseteq A, |A'|=r} \ \sum_{c\in A'} supp_{w'}(c)$ over all feasible weight vectors $w'\in\R^E$; 
		\item the sum of member supports must be exactly $\sum_{c\in A} supp_w(c)=\sum_{n\in \cup_{c\in A} N_c} s_n$; and
    \item if there is a voter $n\in N$ and candidates $c\in C_n$, $c'\in C_n\cap A$ with $supp_w(c)>supp_w(c')$, then $w_{nc}=0$.
\end{enumerate}
Furthermore, a feasible solution $(A,w)$ is balanced if and only if it observes properties 2 and 3 above.
\end{lemma}

Notice that by setting $r=1$ on the first point, we obtain that balanced vector $w$ indeed maximizes the least member support $supp_w(A)$ over all feasible weight vectors. 
More generally, for each $r$ the quantity defined in the first point defines a lower bound on the cost for an adversary to get $r$ representatives in the validator committee in NPoS, so maximizing these objectives for all thresholds $r$ aligns with our security objective as it makes any attack as costly as possible. 
The second point follows from the fact that the sum of member supports is maximal, so the feasibility inequality~\eqref{eq:feasible} must be tight for each voter $n$ with approved candidates in $A$. The third point is a consequence of having the supports as evenly distributed as possible within $A$: if candidate $c$ has a higher support than another candidate $c'$, then none of $n$'s vote can go to $c$, as all of it must have been assigned to $c'$ or to other members with lower support. 

In Appendix~\ref{s:balanced} we present new algorithms for computing a balanced weight vector for a given committee $A$. 
In particular, we prove that one can be found in time $O(|E|\cdot k + k^3)$ using parametric flow techniques, which to the best of our knowledge is the current fastest algorithm in the literature even for the simpler problem of maximizing $supp_w(A)$.

\begin{remark}\label{rem:bal}
In the remainder of the paper, we denote by $\bal$ the time complexity of finding a balanced weight vector, which will depend on the precise algorithm used.
\end{remark}

\emph{Equivalence of objectives.} 
We now establish that our security objective~\eqref{eq:security} is indeed equivalent to maximin support. 
For this we use the fact that, in view of point 1 in Lemma~\ref{lem:balanced}, in the maximin support objective we can reduce the solution space to only balanced solutions without loss of generality. The proof of the next lemma is delayed to Appendix~\ref{s:proofs}.

\begin{lemma} \label{lem:equivalence} If $(A,w)$ is a balanced solution, then
$$supp_w(A) = \min_{\emptyset\neq A' \subseteq A} \frac{1}{|A'|} \sum_{n\in \cup_{c\in A'} N_c} s_n.$$
Consequently, maximin support, the problem of maximizing the left-hand side over all balanced full solutions $(A,w)$, 
is equivalent to the problem of maximizing the right-hand side over all full committees $A$. 
Furthermore, this equivalence preserves approximations, as any balanced solution $(A,w)$ provides the same objective value to both problems.
\end{lemma}

\emph{Network flows.}
In many proofs we deal with a vector $f\in\mathbb{R}^{E}$ of edge weights over the input graph $G=(N\cup C,E)$, which we regard as a vector of flows with positive signs considered to be flow directed toward $C$, and negative signs as flow directed toward $N$. 
Consequently, the \emph{excess} of a voter $n\in N$ relative to $f$ is $f(n):=\sum_{c\in C_n} f_{nc}$, and the excess of a candidate $c\in C$ is $f(c):= -\sum_{n\in N_c} f_{nc}$. 
A set of vertices $S\subseteq N\cup C$ has \emph{net excess} if $\sum_{x\in S} f(x)>0$, and it has \emph{net demand} if $\sum_{x\in S} f(x)<0$.    
A vector $f'\in\mathbb{R}^E$ is a \emph{sub-flow of $f$} if a) for each edge $e\in E$ with $f'_e\neq 0$, flows $f'_e$ and $f_e$ have the same sign and $|f'_e|\leq |f_e|$, and b) for each vertex $x\in N\cup C$ with $f'(x)\neq 0$, excesses $f'(x)$ and $f(x)$ have the same sign and $|f'(x)|\leq |f(x)|$. 
The proof of the next lemma is delayed to Appendix~\ref{s:proofs}.

\begin{lemma}\label{lem:subflow}
If weight vectors $w, w'\in\R^E$ are non-negative and feasible for the given instance, and $f'\in\mathbb{R}^E$ is a sub-flow of $f:=w'-w$, then both $w+f'$ and $w'-f'$ are non-negative and feasible as well.
\end{lemma}

\begin{remark}
In all algorithms analyzed, we assume that all numerical operations take constant time.
\end{remark}

%\section{Properties}\label{sec:properties}

\eray{Polkadot provides the following properties.}{-comment: This sentence by itself is strange and does not add any content. Either remove it or put some more explanation, e.g., ``Polkadot provides utility, finality, decentralization, availability, messaging reliability, reasonable size and bandwidth properties that we explain in detail in this section. ".-} \syed{}{We need something which relates these properties together, perhaps says why they are necessary,...}
 
\subsection{Utility}

\syed{Are we equating parachain with a state machine in this section? if so we need to make that clear here or somewhere in the background section}

\syed{}{One can divides the Polkadot state machine into sub state machines which are recording the state of each parachain}. In such model, the state machine of each parachain \syed{in Polkadot}{-remove as mentioned above} provides a utility to the \syed{system}{-which system? Polkadot as whole or the Parachain} participants. 
 \eray{It is ensured by state machines that interprets the willingness of a participant to pay for a transaction to be included}{-comment: This sentence does not have an object. It is not clear what is ensured. I did not understand it well, but here is what I got from it: `` State machines interpret the willingness of a participant to pay for a transaction."- }.  Polkadot governance mechanism enables participants to decide \syed{what}{which} state machines should be included based on \syed{}{the} needs of participants. 

\subsection{Validity}
\syed{}{By Validity we refer to the} Validity of a new state of a parachain \syed{is}{as} defined by its own state transition function \syed{that}{which} defines \syed{how a parachain can move from \eray{a}{one} state to another \eray{state}{-remove-}}{the transition rules of the parachain state machine}. Validators who are responsible to (in)validate and produce blocks in the relay chain  know these functions so that they can  check whether a given state of a parachain is valid or not.  In Polkadot, we have three levels of validity checks for each state of each parachain. The first-level check of a parachain state is executed by validators who are responsible for this parachain. These validators are called parachain validators and they shift from one parachain to another \eray{parachain}{-remove-} periodically. The second level of check is executed by staked parties called fishermen \eray{which}{who} report to validators if they see any invalid state to receive some reward. And the last level of check is executed by randomly chosen validators after the block including the state is produced. These checks guarantee that it is almost impossible to have a finalized invalid state in the relay chain.

\subsection{Finality}

\syed{}{We need a minimal definition of finality before we say finality is important to us} It is necessary to provide finality on a state of a parachain  to provide a reliable communication in Polkadot so that parachains act by relying on the fact that the data provided by Polkadot related to other parachains will never change.  Polkadot provides the finality property via relay chain which is based on a heterogeneous consensus mechanism \eray{:}{that includes} provable \eray{consensus}{-remove-} and probable \eray{consensus}{consensuses}. The relay chain provides provable consensus with GRANDPA (GHOST-based Recursive ANcestor Deriving Prefix Agreement)  finality gadget. Validators are supposed to vote for a chain in GRANDPA  which has valid blocks (e.g., having valid states of parachains). The probable consensus is based on the block production mechanism of the relay chain that is called BABE (Blind Assignment for Blockchain Extension). BABE is a proof-of-stake based block production mechanism that privately and evenly assigns validators to produce blocks.

\subsection{Decentralization}

\syed{In}{While decentralization has been one of the main motivation to adopt the blockchain model,} some of the most popular blockchain projects\syed{, there are concerns of centralization of power.}{are becoming more and more centralized.} 
This centralization can be caused by several factors, such as
a) a block-producer selection method where some minorities are over-represented or receive disproportionate power, or
b) an incentive mechanism that concentrates wealth or encourages cartel formation.
As we explain in the corresponding sections, our validator selection method (based on nominated proof-of-stake)
as well as our incentive layer are particularly designed to \syed{fight}{defy/resist}  centralization,
and we provide new and precise decentralization guarantees.


\subsection{Availability}
\syed{}{We need a one-liner definition of each property before talking about it}
The availability of data in parachains is very critical because only available data can be validated.  In Polkadot, we provide availability via erasure codes of data which are distributed to validators. Thus, when a validator \eray{needs}{-comment: needs or wants? If needs, what does he/she need?-}, he/she can contact with some subset of validators to construct \eray{it}{-comment: what does it refer to? Availability? -}. We guarantee that any state with some unavailable data cannot be finalized in the relay chain.

%TODO IN FUTURE: We may need to add more detail about the following two properties
\subsection{Messaging Reliability} Polkadot provides message reliability by regulating and ordering what \syed{messages}{what messages? we need context here} have been sent and received by which parachain or parathread.

\subsection{Reasonable Size and Bandwidth} In Polkadot, the block size is selected carefully with respect to scalability and security. 
\eray{}{-comment: What is the block size? If mentioned in the coming sections, refer to that. The title has bandwidth, but nothing is mentioned about it.-}

% How do we achieve them?

% Briefly breaking them down into components.

\section{Polkadot Overview of Sub-protocols and Components}\label{sec:summary}\label{sec:components}
%\begin{samepage}
Next, we summarise Polkadot functionality shortly for an overall picture and then continue to describe the individual sub-protocols and components. %We conclude with a summary and description of what each sub-protoclol and components achieve in Section \ref{sec:summary}. 

Polkadot's validators are selected by the NPoS scheme \ref{sec:validators}. Nominated Proof-of-Stake or NPoS is our adaptation of PoS where an unlimited amount of token holders can participate as nominators, backing with their stake a large but limited set of validators. This paradigm simultaneously achieves high levels of security and scalability, as well as an unprecedented level of decentralisation by ensuring a property known in voting theory as proportional justified representation \cite{sanchez2017proportional, brill2017phragmen}. Nominators, who are economically vested in the security of the system, act as watchdogs over the validators' performance. Based on the nominators' expressed preferences over candidates, every era the system selects a set of validators with stake backings that are as high and as evenly distributed as possible. Nominators are also economically disincentivized from concentrating their votes on too few validators, which helps maintain the system decentralised over time. Furthermore, the election mechanism is highly adaptive to sudden changes, such as some validators being kicked out after a slashing, as it automatically redistributes the nominators' backings even when the votes themselves do not change. 

The security goal of Polkadot is to be Byzantine fault tolerant when the participants are rational (see \ref{sec:economics} for more detail on incentives and economics). We assume that witht he properties NPoS gives, the stakeholders elect a set of validators that has a more than $2/3$ fraction of honest validators.

The elected validators are responsible for running the relay chain \ref{sec:relaychain}. While each parachain's collators are responsible for generating parachain blocks \ref{sec:parachainblockproduction}, the validators are divided into rotating subsets, one for each parachain and need to attest to the validity of parachain blocks before the headers of those blocks are included in the relay chain.

To achieve good scalability the number of validators in each of these subsets is small. Nonetheless, thanks to the NPoS guarantee that every validator is well backed, the availability and validity scheme (Section~\ref{sec:validity-and-availability}) can ensure that any attack on the validity of Polkadot is very expensive in expectation. In fact, the entirety of Polkadot's economic security backs every parachain. This is in stark contrast to having, say, 100 independent blockchains with an equivalent sum total of stake, where on average each blockchain is backed by 1/100-th of the stake, and thus only benefits from 1/100-th the level of security. We guarantee availability by using erasure coding of each parachain block to make the validators collectively and robustly responsible for the availability of these blocks without breaking scalability. 

For this to work, we need to be able to revert the chain until we know with good probability that all parachains are correct. This means that we need to be able to reorganise the chain and for that the chain needs to be capable of forking. Thus we use a block production mechanism, BABE \ref{sec:babe}, that while run by validators, has similar properties to proof-of-work chains. Specifically we can use the longest chain rule as part of our consensus, and that the next block producer is not known in advance. On its own BABE would require us to wait a long time after a block is produced for finality, when we expect with high probability that a block would never be reverted. Slow finality is required in some circumstances to deal with challenges to availability. Most of the time, however, we would prefer to finalise blocks much faster.  For this purpose, validators finalise blocks using GRANDPA \ref{sec:consensus}, a finality gadget that is cleanly separated from block production. This separation makes it very adaptive and here allows us to delay finalising blocks until challenges are dealt with, without slowing down block production. GRANDPA gets Byzantine agreement on finalised blocks and will allow us to prove to an entity that keeps track of the validator set which blocks are finalised, which will be important for bridges \ref{sec:bridge}

If an account on one parachain sends tokens to another parachain, then XCMP \ref{sec:XCMP} ensures that this message is delivered correctly. It is sent at a speed which is not dependent on how long it takes to finalise blocks, which means that it needs to deal with the possibility of Polkadot forking. The parachain and XCMP logic ensure that a fork of the relay chain defines a consistent history of Polkadot and so that messages only arrive when they have been sent previously in the history defined by this fork. If the token transfer is carried out in conjunction with SPREE modules \ref{sec:SPREE} then that ensures that so long as the parachains execute correctly then tokens can only be created and destroyed in an agreed upon way. In turn the correct execution of the chains code is guaranteed by the availability and validity scheme. SPREE ensures that this code is correct as well, even though chains can change their own code.

The governance mechanism \ref{sec:governance} allows Polkadot dot-holders to participate in the decision-making process rather than being imposed changes to the system by a central authority or in the case of some decentralised systems, the developers of the system. 

Lastly we review some of the primitives that Polkadot sub-protocols are suing such as the cryptographic keys and networking scheme in Section \ref{sec:crypto} and Section \ref{sec:networking}, respectively. %Sell these!






%Polkadot consists of a main chain called the relay chain and multiple sharded chains called parachains. 
 %that are also responsible for producing the relay chain blocks \ref{sec:babe} and keeping the state of all the parachains \ref{sec:relaychain}.
%Validators need to vote on the consensus over all the parachains, see the consensus scheme \ref{sec:grandpa} for more details.

%Collators and fishermen that are responsible for parachain block production  and reporting invalid parachain blocks respectively.
%The parachain validators assigned to each parachain validate each parachain block and are responsible to keep it available, see \ref{sec:validity-and-availability}. Moreover, another feature of Polkadot is enabling interchain messaging among parachains, see \ref{sec:XCMP} for more details.
%Furthermore, Polkadot has a decentralised governance scheme \ref{sec:governance} that can change any Polkadot design decisions and parameterisation.

%What do they achieve (refer to properties)?
\subsection{Nominated proof-of-stake and validator selection}\label{sec:validators}
% The role of the validator is central to Polkadot, and as such validators need to run costly operations,
% ensure high communication responsiveness, and build long-term reputation of reliability.
% They are heavily staked and prone to slashing in case of offenses,
% but are otherwise highly remunerated by the network.
Polkadot will use Nominated Proof-of-Stake (NPoS), our very own version of proof-of-stake (PoS).
Consensus protocols with deterministic finality, such as the one in Polkadot, require a set of registered validators of bounded size.
Polkadot will maintain a number $\nval$ of validators, in the order of hundreds or thousands.
This number will be ultimately decided by governance, and is intended to grow along with the number of parachains;
yet, it will be independent of the number of users in the network, to ensure scalability.
However, NPoS allows for an unlimited number of DOT holders to participate as \emph{nominators},
thus maintaining high levels of security by putting more value at stake.
As such, NPoS is not only much more \emph{efficient} than proof-of-work (PoW),
but also considerably more \emph{secure} than standard PoS.
Furthermore, we introduce new guarantees on \emph{decentralization} hitherto unmatched by any other PoS-based blockchain.

A new set of validators is selected at the beginning of every era (a period during roughly one day),
to serve for that era, according to the nominators' preferences.
More precisely, any DOT holder may choose to become a validator candidate or a nominator.
Each candidate indicates the amount of stake he is willing to stake and his desired commission fee for operational costs.
In turn, each nominator locks some stake and publishes a list (of any size) of the candidates that she trusts.
Then, a public protocol (discussed below) takes these lists as input and selects the candidates
with the most backing to serve as validators for the next era.

Nominators share the rewards, or eventual slashings, with the validators they selected, on a per-staked-DOT basis
(see more details in \autoref{sec:economics}).
Nominators are thus economically incentivised to act as watchdogs for the system, and they should base their preferences 
on parameters such as validators' stakes, commission fees, past performance, and security practices.
Our scheme allows for the system to select validators with massive amounts of aggregate stake
- much higher than any single party's DOT holdings -
and thus helps turn the validator selection process into a meritocracy rather than a plutocracy.
In fact, at any given moment we expect there to be a considerable fraction of all the DOT supply be staked in NPoS.
This makes it very difficult for an adversarial entity to get validators elected
(as it either needs a large amount of DOTs or high enough reputation to get the required nominators' backing)
as well as very costly to attack the system (as it is liable to lose all of its stake, stake backing, and reputation).

Polkadot elects validators via a decentralized protocol with carefully selected, simple and publicly known rules,
taking the nominators' lists of trusted candidates as input. Formally, the protocol solves a multi-winner election
problem based on approval ballots, where nominators have voting power proportional to their stake,
and where the goals are \emph{decentralization} and \emph{security}.

\paragraph{Decentralization.} In the late 19th century, Swedish mathematician Edvard Phragm\'{e}n
proposed a method for electing members to his country’s parliament~\cite{brill2017phragmen}.
He noticed that the election methods at the time tended to give all the seats
to the most popular political party; in contrast, his new method ensured that the number of seats
assigned to each party were proportional to the votes given to them, so it gave more representation to minorities.

In the literature of computational social choice, Phragm\'{e}n's method has been recently revisited
and shown to achieve a property called \emph{proportional justified representation} (PJR).
In the context of NPoS, this property turns out to be ideal to guarantee decentralization.
It ensures that the set of selected validators represents as many nominator minorities as possible,
proportional to their stake, and that no minority is under-represented.
(These minorities may be defined by common political views, geographical location, etc.)

Our validator selection protocol will observe the PJR property.
Formally, this means that if each nominator $\nom \in \Nom$ has stake $stake_\nom$ and backs a subset
$\Can_\nom\subseteq \Can$ of candidates,\footnote{For ease of presentation, we consider here
that candidates have no stake. This can be achieved by representing a candidate's stake
as an additional nominator that exclusively approves that candidate.}
the protocol will select a set $\Val\subseteq \Can$ of $\nval$ validators such that, 
if there is a subset $\Nom'\subseteq \Nom$ and some $1\leq t\leq \nval$ such that
$$|\cap_{\nom\in \Nom'} \Can_n| \geq t \quad \text{ and } \quad
\frac{1}{t} \sum_{\nom\in \Nom'} stake_\nom \geq \frac{1}{n} \sum_{\nom\in \Nom} stake_\nom,$$
then $|\Val\cap (\cup_{\nom\in\Nom'} \Can_n)| \geq t$.
In other words, if there is a group $\Nom'$ of nominators with at least $t$ commonly trusted candidates, who can "afford" to provide a support of value $\frac{1}{n} \sum_{\nom\in \Nom} stake_\nom$ to each one of them, then these nominators must indeed be represented by at least $t$ candidates in $\Val$, though not necessarily commonly trusted.

\paragraph{Security.} If a nominator gets two or more of its trusted candidates elected as validators,
the protocol must also decide how to split her stake and assign these fractions to them.
In turn, these assignations define the total stake backing that each validator receives.
Our objective is to make these validators' backings as high and as balanced as possible.
In particular, we focus on maximizing the \emph{minimum validator backing}.
Intuitively, the minimum backing corresponds to a lower bound on the cost for an adversary to gain control
over one validator, as well as a lower bound on the potential slashable amount for a misconduct.

Formally, if each nominator $\nom \in \Nom$ has $stake_\nom$ and backs a candidate subset $\Can_\nom\subseteq \Can$,
the protocol must not only find a set $\Val\subseteq \Can$ of $\nval$ validators
with the PJR property, but also define a distribution of each nominator's stake among the elected validators that she backs,
i.e. a function $f:\Nom\times\Val \rightarrow \mathbb{R}_{\geq 0}$ so that
$$\sum_{\val\in \Val\cap \Can_\nom} f(\nom, \val) = stake_\nom \quad \text{ for each nominator } \nom\in\Nom,$$
and the objective is then
$$\max_{\text{solutions } (\Val, f)} \min_{\val\in \Val} backing_f(\val),
\quad \text{ where } backing_f(\val) := \sum_{\nom\in \Nom: \ \val\in \Can_\nom} f(\nom, \val). $$

The problem defined by this objective is called \handan{\emph{maximin support}}{refs} in the literature, and is known to be NP-hard.
We have developed for it \handan{several efficient algorithms}{either we should write the algorithms here or give reference to these algorithms} which offer theoretical guarantees
(constant-factor approximations), and which also scale well and perform well on our testnet.

\eray{}{-comment: Why is there this empty space between subsections?-}

\subsection{Relay Chain State Machine}\label{sec:relaychain}

Formally, Polkadot is a replicated sharded state machine where shards are the parachains and Polkadot relay chain is part of the protocol ensuring global consensus among all the parachains. Therefore the Polkadot relay chain protocol, can itself be considered as a replicated state machine on its own. In this sense, this section describes the relay chain protocol by specifying the state machine governing the relay chain. To that end, we describe the relay chain state and the detail of state transition govern by transactions grouped the relay chain blocks.

\paragraph{State}
Polkadot relay chain state is represented similar to of the Ethereum. In the sense that the state is represented using an <i>associative array</i> data structure composed of a collection of $(key, value)$ pairs where each key is a unique. There is no assumption on the format of the key or the value stored under it beside the fact that they are finite byte arrays.

A <i>Merkle radix-16 trie</i> keeps the Merkle hashes corresponding to the $(key, value)$ pairs stored in the relay chain state enable identifying current state using its root hash and providing efficient proof of inclusion of a specific pair.

To keep state size in control, the relay chain state is solely used to facilitate relay chain operation such as staking and identifying Validators. It is not suppose to store any information regarding the internal operation of the parachains.

\paragraph{State transition}
Like any transaction-based transition system, Polkadot state changes via an executing ordered set of instructions. These instructions (traditionally known as transactions), are refereed to as extrinsics in Polkadot Jargon covering any data is provided from ``outside'' of the machine's state which can affect state transition. Polkadot relay chain is divided into two major components. The execution logic of the state-transition function is mainly encapsulated in the ``Runtime'' while all other generic operations which are commonly shared among modern block chain based replicated state machines are embedded into the runtime environment, In particular the latter is in charge of network communication, block production and consensus engines.

Runtime functions are compiled into a Web assembly module and is stored as part of the state. The runtime environment communicates the extrinsics with the Runtime and interact with it to execute enable the state transition. In this way, the state transition logic itself can be upgraded as a part of state transition.

\paragraph{Extrinsics}

Extrinsics are the input data supplied to the Polkadot Relay chain state machine to transition to a new states. Extrinsics are needed to be stored into blocks of the relay chain in order to achieve consensus among the state machine replica. Extrinsics are divided into two broad categories namely Transactions and Inherents.

Transactions are signed and are gossiped around on the network between nodes. In contrast, Inherents are not signed and are not gossiped individually but rather only when they are included in a block. The inherents in a block are assumed to be valid if super majority of validators assume so. Timestamp is an example of inherent extrinsics which must be included in each Polkadot Relay chain block.

Transactions on the relay chain are mainly concerned with the operation of the relay chain and Polkadot protocol as whole, such as \texttt{set\_code}, \texttt{transfer}, \texttt{bond}, \texttt{validate}, \texttt{nominate}, \texttt{vote}.

Relay chain block producers listen to all transactions network messages. Upon receiving a transaction message, the transaction(s) are validated by the Runtime. The valid transactions then are arranged in a queue based on their priority and dependency and are considered for inclusion in future blocks accordingly.

\paragraph{Block format}
%% Block header Block body 
A typical relay chain block is consists of header and body. The body is simply consists of list of extrinsics.

The header contains: the {\it hash of parent block}, {\it block number}, the {\it root of the state trie}, the {\it root of the Merkle tree} resulting from arranging the extrinsics in such a tree. and the {\it digest}. The digest stores auxiliary information from the consensus engines which are required for to validate the block and its origin as well as information helping light clients to validate the block without having access to the state storage.

\paragraph{Consensus}

%%Block production, Babe,

%%Block Finality Grandpa.

\paragraph{Block Building}\label{sec:relaychainblockproduction}

%% All happens after babe



\subsection{Consensus}\label{sec:grandpa}



%In our consensus protocols, we assume that a message sent by a validator arrives to other validators at most $\D$ times later where $\D$ is an unknown parameter. So, validators are in a partially synchronous network that guarantees  eventual delivery.
\subsubsection{Blind Assignment for Blockchain Extension (BABE)}

In Polkadot, we produce relay chain blocks using our Blind Assignment for Blockchain Extension protocol, abbreviated BABE. BABE assigns validators randomly to blocks production slots using  the randomness generated with blocks. These assignments are completely private until the assigned parties produce their blocks. Therefore, we use ``Blind Assignment'' in the protocol name. 

We may have slots without any assignments. We call these slot as empty slot. In order to fill the empty slots, we have secondary block production mechanism based on Aura. We note that these blocks do not contribute the security of BABE. 

Next, we describe BABE together with its security properties. Then, we explain the secondary block production mechanism and discuss about its contribution on BABE.

\paragraph{BABE:}









\subsubsection{GRANDPA} \label{sec:grandpa}
\begin{figure}[h!]
  \centering
  \includegraphics[width=0.4\textwidth]{images/Grandpa.jpg}
  \caption{GRANDPA votes and how they are aggregated.}
    \label{fig:grandpa}
\end{figure}
For the relay chain consensus, we want provably finality, which makes bridges to chains outside Polkadot easier, and we also want the ability to revert chains in the short term if parachain blocks are unavailable for invalid.
So we combine a block production mechanism BABE with the ability to fork and that gives eventual consensus with a finality gadget GRANDPA, that can finalise blocks and provide proofs that they are final.
BABE is similar to Ourobors Praos[citation needed] but with a fixed set of validators and improved resilience to clock drift.
It works by dividing time into slots in each of which validators computing a Verifiable Random Function(VRF) which will decide whether they have the ability to produce a valid block in that slot.
For fork-choice, we go with the longest chain that includes all blocks finalised by GRANDPA.

\subsection{Parachains}
 \paragraph{Block Production}

 \paragraph{Validity and Availability}

 \paragraph{ICMP}

\subsection{Economics and Incentive Layer}\label{sec:economics}

Polkadot will have a native token called DOT. Its various functions are described in this section.


\subsubsection{Staking rewards and inflation}\label{sec:inflation}

We start with a description of staking rewards, i.e.~payments to \emph{stakers} -- validators and nominators -- 
coming from the minting of new DOTs. 
Unlike some other blockchain protocols, the amount of tokens in Polkadot will not be bounded by an absolute constant, but there will rather be a controlled yearly inflation rate. Indeed, recent research~\cite{chitra2019competitive} suggests that in a proof-of-stake based protocol the staking rewards must remain competitive, in order to maintain high staking rates and high security levels, so deflationary policies are advised against. 

In our design, staking rewards are the only mechanism that mints DOTs. 
%Dots minted for staking rewards are the main driver of inflation in the system. This is because Treasury (Section \ref{sec:governance}), which receives a fraction of all transaction fees, slashings and missed validator rewards, has a mechanism design to closely matches its expenditure to its income, and thus is neither a net burner not net minter of Dots. 
Thus, it is convenient to introduce our inflation model in this section as well. 
%We do not consider rewards coming from transaction fees, slashings, nor rewards to fishermen and other reporters of offenses; these will be analyzed in further sections. 

Recall from the description of the NPoS protocol (Section \ref{sec:validators}) that both validators and nominators stake DOTs. 
They get paid roughly proportional to their stake, but can be slashed up to $100\%$ in case of a misconduct. 
Even though they are actively engaged for only one era%
\footnote{Recall that an era lasts approximately one day. See Table~\ref{t:time} in the Appendix.} 
at a time, they can continue to be engaged for an unlimited number of eras. 
During this period their stake is locked, meaning it cannot be spent, and it remains locked for several weeks after their last active era, to keep stakers liable to slashing even if an offence is detected late.

\paragraph{Staking rate, interest rate, inflation rate:} Let the staking rate be the total amount of DOTs 
currently staked by validators and nominators, divided by the current total DOT supply. 
The stakers' average interest rate will be a function of the staking rate: 
if the staking rate dips below a certain target value selected by governance, 
the average interest rate is increased, thus incentivising more participation in NPoS, and vice versa. 
For instance, a target staking rate of $50\%$ could be selected as a middle ground between security and liquidity. 
If the stakers' average yearly interest rate is then set to $20\%$ at that level, 
we can expect the inflation rate to fluctuate closely around $50\%\times 20\% = 10\%$. 
Hence, by setting targets for the staking rate and stakers' interest rate, we also control the inflation rate. 
Following this principle, every era we adjust our estimate of the staking rate, 
and use it to compute the total amount of DOTs to be paid to stakers for that era.

\paragraph{Rewards across validator supports:} 
Once the total payout for the current era is computed, we need to establish how it is distributed.
Recall that the validator election protocol (Section \ref{sec:validators}) partitions the active stake into 
\emph{validator supports}, where each validator support is composed of the full stake of one validator 
plus a fraction of the stake of its backing nominators, and this partition is made so as to make validator supports 
as high and evenly distributed as possible, hence ensuring security and decentralisation. 
A further incentive mechanism put in place to ensure decentralisation over time 
is paying validator supports equally for equal work, regardless of their stake. 
As a consequence, if a popular validator has a high support, its nominators will likely be paid less per staked DOT 
than nominators backing a less popular validator. Hence, nominators will be incentivised to change their preferences 
over time in favour of less popular validators (with good reputation nonetheless), helping the system converge to the ideal case where all validator supports have equal stake.

In particular, we devise a point system in which validators accumulate points for each payable action performed, 
and at the end of each era validator slots are rewarded proportional to their points. 
This ensures that validators are always incentivised to maintain high performance and responsiveness. 
Payable actions in Polkadot include: a) validating a parachain block, 
b) producing a relay chain block in BABE, 
c) adding to a BABE block a reference to a previously unreferenced uncle block,%
\footnote{In the BABE protocol, at times two block producers may generate different blocks A and B at the same height, leading to a temporary fork in the relay chain. The fork will quickly be resolved and one of the blocks selected, say A, as part of the main chain, while block B becomes an \emph{uncle} to all descendents of A. For security reasons, it is convenient to record and timestamp \emph{all} blocks produced, but since uncle blocks cannot be accessed via parent relations, we encourage block producers to explicitly add these references to the main chain.}
 and d) producing an uncle block.

\paragraph{Rewards within a validator slot:} As a nominator's stake is typically split among several validator supports, 
their payout in an era corresponds to the sum of their payouts relative to each of these supports. 
Within a validator support, the payment is as follows: 
First, the validator is paid a \emph{commission fee}, which is an amount intended to cover its operational costs. 
Then, the remainder is shared among all stakers -- both validator and nominators -- proportional to their stake. 
Thus, the validator receives two separate rewards: a fee for running a node, and a payout for staking. 
We remark that the commission fee is up to each validator to set, and must be publicly announced in advance. 
A higher fee translates to a higher total payout for the validator, and lower payouts to its nominators, 
so nominators will generally prefer to back validators will lower fees, and the market regulates itself in this regard. 
Validators who have built a strong reputation of reliability and performance 
will however be able to charge a higher commission fee, which is fair.

\medskip

We finalise the section with some observations on the incentives that our payout scheme is expected to cause on stakers. 
First, as validators are well remunerated and their number is limited, 
they have an incentive to ensure high backing levels from nominators to ensure getting elected, 
and thus they will value their reputation. Over time, we expect elections to be highly competitive 
and for elected validators to have strong track records of performance and reliability and large stake backings.
Second, even if payouts across different validator supports are independent of their stake, 
within a validator support each actor is paid proportional to their stake, 
so there is always an individual incentive to increase one's own stake. 
Finally, if a validator gains a particularly high level of backing, it can profit from it by either increasing 
its commission fee, which has the effect of raising its own reward at the risk of losing some nominations, 
or launching a new node as a validator candidate and splitting its backing among all its nodes. 
On this last point, we welcome operators with multiple validator nodes, 
and even aim to make their logistics simpler. 
%, so long as they disclose such correlations 
%so that nominators can adjust their strategy accordingly.


\subsubsection{Relay-chain block limits and transaction fees}

\paragraph{Limits on resource usage:} We bound the amount of transactions that a relay-chain block can process, 
in order to a) ensure that each block can be processed efficiently even on less powerful nodes and avoid delays in block production, and b) have guaranteed availability for a certain amount of high-priority, operational transactions such as misconduct reports, even when there is high network traffic. 
In particular, we set block constraints on the following resources: on-chain byte-length, 
and time and memory required to process the transactions.

We classify transactions into several types, according to their priority level and resource consumption profile. 
For each of these types we have run tests based on worst-case scenarios for state, and for different input arguments. 
From these tests, we establish conservative estimates on resource usage for each transaction, and we use these estimates to ensure that all constraints on resource usage are observed.

We also add an extra constraint on resources: we distinguish between regular and high-priority transactions, and only let regular transactions account for up to $75\%$ of each block resource limit. This is to ensure that each block has a guaranteed space for high-priority transactions of at least $25\%$ of resources.

\paragraph{Transaction fees:} We use the model described above to set the fee level of a transaction based on three parameters: its type, its on-chain length, and its expected resource usage. This fee differentiation is used to reflect the different costs that a transaction incurs on the network and on the state, and to encourage the processing of certain types of transactions over others. A fraction of every transaction fee is paid to the block producer, while another fraction goes to finance the Treasury (Section~\ref{sec:treasury}). We highlight that, for a block producer, the rewards coming from transaction fees may constitute only a small fraction of their overall revenue, just enough to incentivise inclusion on the block.

We also run an adaptive transaction fee schedule that reacts to the traffic level, and ensures that blocks are typically far from full, so that peaks of activity can be dealt with effectively and long inclusion times are rare. In particular, the fee of each transaction is multiplied by a parameter that evolves over time depending on the current network traffic. 

 

We make fees evolve slowly enough, so that the fee of any transaction can be predicted accurately within a frame of an hour. In particular, we do not intend for transaction fees to be the main source of income for stakers.



%\subsubsection{Slashing}

\subsection{Governance}
 \paragraph{Relay Chain Council Referenda}
 \paragraph{Parachain Allocation}
 \paragraph{Treasury}

\subsection{Cryptography}\label{sec:crypto}

In Polkadot, we necessarily distinguish among different permissions and functionalities with different keys and key types, respectively.  We roughly categorise these into account keys with which users interact and session keys that nodes manage without operator intervention beyond a certification process.

\subsubsection{Account keys}

Account keys have an associated balance of which portions can be {\em locked} to play roles in staking, resource rental, and governance, including waiting out a couple types of unlocking period.  We allow several locks of varying duration, both because these roles impose different restrictions, and for multiple unlocking periods running concurrently. 

We encourage active participation in all these roles, but they all require occasional signatures from accounts.  At the same time, account keys have better physical security when kept in inconvenient locations, like safety deposit boxes, which makes signing arduous.  We avoid this friction for users as follows.

Accounts that lock funds for staking are called {\em stash accounts}.  All stash accounts register a certificate on-chain that delegates all validator operation and nomination powers to some {\em controller account}, and also designates some {\em proxy key} for governance votes.  In this state, the controller and proxy accounts can sign for the stash account in staking and governance functions respectively, but not transfer fund.  

\smallskip

At present, we suport both ed25519 \cite{} and schnorrkel/sr25519 \cite{} for account keys.  These are both Schnorr-like signatures implemented using the Ed25519 curve, so both offer extremely similar security.  We recommend ed25519 keys for users who require Hardware Security Module (HSM) support or other external key management solution, while schnorrkel/sr25519 provides more blockchain-friendly functionality like Hierarchical Deterministic Key Derivation (HDKD) and multi-signatures \cite{}.  

In particular, schnorrkel/sr25519 uses the Ristretto implementation \cite{Ristretto} of Mike Hamburg's Decaf \handan{\cite[\S7]{Decaf}}{a strange symbol appears in pdf}, which provide the 2-torsion free points of the Ed25519 curve as a prime order group.  Avoiding the cofactor like this means Ristretto makes implementing more complex protocols significantly safer.  We employ Blake2b for most conventional hashing in Polkadot, but schnorrkel/sr25519 itself uses STROBE128 \cite{STROBE}, which is based on Keccak-f(1600) and provides a hashing interface well suited to signatures and Non-Interactive Zero-Knowledge Proofs (NIZK) \cite{} \cite{}.
% See https://github.com/w3f/schnorrkel/blob/master/annoucement.md for more detailed design notes.

\subsubsection{Session keys}\label{sec:session_keys}

Session keys each fill roughly one particular role in consensus or security.  As a rule, session keys gain authority only from a session certificate, signed by some controller key, that delegates appropriate stake.  

At any time, the controller key can pause or revoke this session certificate and/or issue replacement with new session keys.  All new session keys can be registered in advance, and most must be, so validators can cleanly transition to new hardware by issuing session certificates that only become valid after some future session.  We suggest using pause mechanism for emergency maintenance and using revocation if a session key might be compromised.  

We prefer if session keys remain tied to one physical machine because doing so minimizes the risk of accidental equivocation.  We ask validator operators to issue session certificate using an RPC protocol, not to handle the session secret keys themselves.  

Almost all early proof-of-stake networks have a negligent public key infrastructure that encourages duplicating session secret keys across machines, and thus reduces security and leads to pointless slashing.
% TODO: I'd meant to cite this somewhere, but not sure where now.  It's easy to cite the slashing to part to cosmos.  Thoughts?

\smallskip

We impose no prior restrictions on the cryptography employed by specific components or their associated session keys types.\footnote{We always implement cryptography for polkadot in native code, not just because the runtime suffers from WASM's performance penalties, but because all of Polkadot's consensus protocols are partially implemented outside the runtime in Substrate modules.}

In BABE \ref{sec:babe}, validators use schnorrkel/sr25519 keys both for regular Schnorr signatures, as well as for a verifiable random function (VRF) based on NSEC5 \cite{NSEC5}.  

A VRF is the public-key analog of a pseudo-random function (PRF), aka cryptographic hash function with a distinguished key, such as many MACs.  We award block production slots when the block producer scores a low enough VRF output $\mathtt{VRF}_{\sk}(r_e || \mathtt{slot_number} )$, so anyone with the VRF public keys can verify that blocks were produced in the correct slot, but only the block producers know their slots in advance via their VRF secret key.

As in \cite{Praos}, we provide a source of randomness $r_e$ for the VRF inputs by hashing together all VRF outputs form the previous session, which requires that BABE keys be registered at least two full epochs before being used.

We reduce VRF output malleability by hashing the signer's public key along side the input, which dramatically improves security when used with HDKD.  We also hash the VRF input and output together when providing output used elsewhere, which improves compossibility when used as a random oracle in security proofs.  See the 2Hash-DH construction from Theorem 2 on page 32 in appendix C of \cite{Praos}.  

In GRANDPA \ref{sec:grandpa}, validators shall vote using BLS signatures, which supports convenient signature aggregation and select ZCash's BLS12-381 curve for performance.  There is a risk that BLS12-381 might drops significantly below 128 bits of security, due to number field sieve advancements.  If and when this happens, we expect upgrading GRANDPA to another curve to be straightforward. 
% https://mailarchive.ietf.org/arch/msg/cfrg/eAn3_8XpcG4R2VFhDtE_pomPo2Q

% TODO: ImOnline
% ref. https://github.com/paritytech/substrate/issues/3546

We treat libp2p's transport keys roughly like session keys too, but they include the transport keys for sentry nodes, not just for the validator itself.  As such, the operator interacts slightly more with these.

We permit controller keys to revoke session key validity of course, but controllers could pause operation for shorter periods.  We similarly permit controllers to register new session keys in advance, which enables a clean handover between validator machines.


\subsection{Networking}

We need networking for a number of components of Polkadot.
These components are: the consensus protocol GRANDPA \ref{},
the block production mechanism for relay chain \ref{},
receiving of parachain blocks,
distribution and recovery of erasure coded pieces for the availability and validity scheme \ref{}, and
the interchain messaging scheme \ref{}.

We use gossiping for all parts mentioned above, except distributing and recovery of erasure coded pieces for the availability and validity mechanism, where we use direct routing for scalability.

For node discovery we use a similar networking scheme as many other blockchains do that is using the widely used distributed hast table (DHT), Kademlia \cite{}.
Kademlia is a DHT that uses XOR distance metric for finding a node and is often used for networks with high churn.
We use Protocols Labs libp2p libraries \cite{} Kademlia implementation with some changes for this purpose.

%\end{samepage}

\section{Future Work}
%In this section we summarise Polkadot's components and describe what they aim to achieve. 




Nominated Proof-of-Stake or NPoS (Section~\ref{sec:validators}) is our adaptation of PoS where an unlimited number of token holders can participate as nominators, backing with their stake a large but limited set of validators. This paradigm achieves high levels of security and scalability simultaneously, as well as an unprecedented level of decentralization by ensuring a property known in voting theory as proportional justified representation \cite{sanchez2017proportional, brill2017phragmen}. Nominators, who are economically vested in the performance of validators, act as watchdogs monitoring the system security. Based on the nominators' expressed preferences over candidates, at the start of every era the system selects a set of validators with stake backings that are as high and as evenly distributed as possible. Nominators are also economically disincentivized from concentrating their votes on too few validators, which helps maintain the system decentralized over time. Furthermore, the election mechanism is highly adaptive to sudden changes, such as some validators being kicked out after a slashing event, as it automatically redistributes the nominators' backings even when the votes themselves do not change. 

Thanks to the NPoS guarantee that every validator is well backed, the availability and validity scheme (Section~\ref{sec:validity-and-availability}) can ensure that any attack on the validity of Polkadot is very expensive in expectation. In fact, the entirety of Polkadot's economic security backs every parachain. 
This is in stark contrast to having, say, 100 independent PoS-based blockchains with an equivalent sum total of stake, where on average each blockchain is backed by 1/100-th of the stake, and thus only benefits from 1/100-th the level of security.  

 

 . 



  

%Moreover, the token economics of Polkadot aims at 

The parachain allocation scheme aims at assigning temporary parachain slots among interested projects in a decentralized and fair manner. The assignation

%\section{Polkadot in Practice}
Many of the components described in this paper have been used in the recently launched "canary network" Kusama (https://kusama.network).


\section*{Acknowledgement}
We would like to thank Bill Laboon from Web3 Foundation for his feedback and Parity Technologies developers for their useful input and good discussions.
\bibliographystyle{plain}
\bibliography{references,crypto,rt,grandpa}
\begin{appendix}
\section{Glossary and Background}

\eray{- state machine
- data structures}{-comment: Is this meant to be a list? It looks strange like this.-}


%\eray{
\begin{table}[h] \label{t:time}
    \begin{tabular}{llll}
        \textbf{Name} & \textbf{Symbol} & \textbf{Def} & \textbf{Description} \\
        \hline
        &&& a new batch of relay chain blocks being finalized \\
        BABE &&& \\
        BABE Slot & \slot & \ref{sec:babe} & a period for which a relay chain block is produced \\
        Block & \block && Data-structure containing extrinsics \\
        Blockchain & \bchain && A chain of blocks \\
        BLS signatures &&& \\
        Byzantine &&& \\
        Candidate validators & \Can && Set of candidate validators \\
        Casper FGG &&& \\
        Casper TFG &&& \\
        CD &&& \\
        CG &&& \\
        Collator & \col & \ref{par:collators} & Assist validators in block production \\
        Collators & \Col && Set of collators \\
        Consensus &&& \\
        CP &&& \\
        CQ &&& \\
        DHT &&& \\
        DOT &&& \\
        Eclipse attacks &&& \\
        ed25519 &&& \\
        Elected validators & \Val && Set of elected validators \\
        Epoch & \ep && a period for which randomness is generated by BABE \\
        Era &&& a period for which a new validator set is decided \\
        Extrinsics &&& \\
        Fishermen & & \ref{par:fishermen} & Monitors the network for misbehavior \\
        GHOST &&& \\
        Gossip &&& \\
        GRANDPA &&& \\
        GRANDPA Round &&& a state of the GRANDPA algorithm which lead to  \\
        HDKD &&& \\
        Kademlia &&& \\
        Local clock & \lclock & \ref{par:network_coms} & A local clock used by validators \\
        MAC &&& \\
        Merkle proof &&& \\
        Merkle tree &&& \\
        Nominator & \nom & \ref{par:nominators} & Stake-holding party who nominates validators \\
        Nominators & \Nom && Set of nominators \\
        NPoS &&& \\
        NSEC5 &&& \\
        Ouroboros Praos &&& \\
        Parachain & \Par && Heterogeneous independent chain \\
        Phragmén &&& \\
        PJR &&& \\
        PoS &&& \\
        PoV &&& \\
        PoW &&& \\
        PRF &&& \\
        Relay Chain && \ref{sec:relaychain} & Ensures global consensus among parachains \\
        reorgs &&& \\
        RPC &&& \\
        schnorrkel &&& \\
        Sentry nodes &&& \\
        Shuffle &&& \\
        sr25519 &&& \\
        State &&& \\
        State transition &&& \\
        STROBE128 &&& \\
        STVF &&& \\
        Validator & \val & \ref{par:validators} & The highest in charge party who seals new blocks \\
        Validators to elect & \nval && Number of validators to elect \\
        VRF public key & \pkvrf & \ref{sec:session_keys} & Key used for the VRF function \\
        VRF secret key & \skvrf & \ref{sec:session_keys} & Key used for the VRF function \\
        XCMP &&& \\
        XOR &&& \\
        Zero-Knowledge Proofs &&& \\
        ZKP &&& \\
    \end{tabular}
\caption{Time periods used in Polkadot}
\end{table}
%}

\alfonso{}{I think the table should contain more information. I would add a) possibly longer descriptions, b) a reference to the section that introduces them (and where we give an even longer description + its reason of being), and c) their lengths in seconds/minutes/hours, where we put a big note saying all lengths are tentative and subject to change considerably.}
\alfonso{}{Also, we should either add "session" to the table, or remove all mentions of sessions. Simplifying could be a good idea, so maybe the latter?}
\end{appendix}

\end{document}
