\section{Glossary and Background}

\eray{- state machine
- data structures}{-comment: Is this meant to be a list? It looks strange like this.-}


%\eray{
\begin{longtable}{p{.15\textwidth}p{.55\textwidth}p{.1\textwidth}p{.1\textwidth}} \label{t:time}
    \textbf{Name}  & \textbf{Description} & \textbf{Symbol} & \textbf{Def} \\
    \hline
    BABE & A mechanism to assign validators randomly to block production. && \ref{sec:babe} \\
    BABE Slot & A period for which a relay chain block is produced. & \slot & \ref{sec:babe} \\
    Candidate\newline- validators & A set of candidate validators. & \Can & \\
    Collator & Assist validators in block production. A set of collators is defined as \Col . & \col & \ref{par:collators} \\
    DOT & The Polkadot native token. && \ref{sec:economics} \\
    Elected\newline- validators & A set of elected validators. & \Val & \\
    Epoch & A period for which randomness is generated by BABE. & \ep & \\
    Era & A period for which a new validator set is decided. && \\
    Extrinsics & Input data supplied to the Relay Chain to transition states. && \ref{par:extrinsics} \\
    Fishermen & Monitors the network for misbehavior. && \ref{par:fishermen} \\
    Gossiping & Broadcast every newly received message to peers. && \ref{sec:gossiping} \\
    GRANDPA & Mechanism to finalize blocks. && \ref{sec:grandpa} \\
    GRANDPA\newline- Round & a state of the GRANDPA algorithm which leads to block finalization. && \ref{sec:grandpa} \\
    Nominator & Stake-holding party who nominates validators. A set of nominators is defined as \Nom . & \nom & \ref{par:nominators} \\
    NPoS & \emph{Nominated Proof-of-Stake} - Polkadot's version of PoS, where nominated validators get elected to be able to produce blocks. && \ref{sec:validators} \\
    Parachain & Heterogeneous independent chain. & \Par & \\
    PJR & \emph{Proportional-Justified-Representation} - Ensures that validators represent as many nominator minorities as possible. && \ref{par:decentralization} \\
    PoS & \emph{Proof-of-Stake} - Alternative to PoW, where parties vote with locked funds. && \ref{sec:validators} \\
    PoV & \emph{Proof-of-Validity} - Mechanism where a validator can verify a block without having its full state. && \ref{sec:parachainblockproduction} \\
    PoW & \emph{Proof-of-Work} - Mechanism where parties vote with processing power. && \\
    Relay\newline- Chain & Ensures global consensus among parachains. && \ref{sec:relaychain} \\
    Runtime & The Wasm blob which contains the state transition functions, including other core operations required by Polkadot. && \ref{par:state_transition} \\
    Sentry\newline- nodes & Specialized proxy server which forward traffic to/from the validator. && \\
    STVF & \emph{State-Transition-Validation-Function} - A function of the Runtime to verify the PoV. && \ref{sec:parachainblockproduction} \\
    Validator & The highest in charge party who seals new blocks. A set of the number of validators to elect is defined as \nval . & \val & \ref{par:validators} \\
    VRF & \emph{Verifiable-Random-Function} - Cryptographic function for determining elected validators for block production. && \ref{sec:babe} \\
    XCMP & A protocol that parachains use to send messages to each other. && \ref{sec:XCMP} \\
\caption{Time periods used in Polkadot}
\end{longtable}
%}

\alfonso{}{I think the table should contain more information. I would add a) possibly longer descriptions, b) a reference to the section that introduces them (and where we give an even longer description + its reason of being), and c) their lengths in seconds/minutes/hours, where we put a big note saying all lengths are tentative and subject to change considerably.}
\alfonso{}{Also, we should either add "session" to the table, or remove all mentions of sessions. Simplifying could be a good idea, so maybe the latter?}