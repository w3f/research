%%
%% This is file `sample-sigconf.tex',
%% generated with the docstrip utility.
%%
%% The original source files were:
%%
%% samples.dtx  (with options: `sigconf')
%% 
%% IMPORTANT NOTICE:
%% 
%% For the copyright see the source file.
%% 
%% Any modified versions of this file must be renamed
%% with new filenames distinct from sample-sigconf.tex.
%% 
%% For distribution of the original source see the terms
%% for copying and modification in the file samples.dtx.
%% 
%% This generated file may be distributed as long as the
%% original source files, as listed above, are part of the
%% same distribution. (The sources need not necessarily be
%% in the same archive or directory.)
%%
%% The first command in your LaTeX source must be the \documentclass command.
\documentclass[sigconf, anonymous, review]{acmart}
%\documentclass[sigconf, anonymous]{acmart}
%\documentclass[sigconf]{acmart}
%%
%% \BibTeX command to typeset BibTeX logo in the docs
%\AtBeginDocument{%
%  \providecommand\BibTeX{{%
%    \normalfont B\kern-0.5em{\scshape i\kern-0.25em b}\kern-0.8em\TeX}}}

%% Rights management information.  This information is sent to you
%% when you complete the rights form.  These commands have SAMPLE
%% values in them; it is your responsibility as an author to replace
%% the commands and values with those provided to you when you
%% complete the rights form.
%\setcopyright{acmcopyright}
%\copyrightyear{2018}
%\acmYear{2018}
%\acmDOI{10.1145/1122445.1122456}

%% These commands are for a PROCEEDINGS abstract or paper.
%%\acmConference[Woodstock '18]{Woodstock '18: ACM Symposium on Neural
%%  Gaze Detection}{June 03--05, 2018}{Woodstock, NY}
%%\acmBooktitle{Woodstock '18: ACM Symposium on Neural Gaze Detection,
%%  June 03--05, 2018, Woodstock, NY}
%%\acmPrice{15.00}
%%\acmISBN{978-1-4503-XXXX-X/18/06}


%%
%% Submission ID.
%% Use this when submitting an article to a sponsored event. You'll
%% receive a unique submission ID from the organizers
%% of the event, and this ID should be used as the parameter to this command.
%%\acmSubmissionID{123-A56-BU3}

%%
%% The majority of ACM publications use numbered citations and
%% references.  The command \citestyle{authoryear} switches to the
%% "author year" style.
%%
%% If you are preparing content for an event
%% sponsored by ACM SIGGRAPH, you must use the "author year" style of
%% citations and references.
%% Uncommenting
%% the next command will enable that style.
%%\citestyle{acmauthoryear}


%\usepackage{arxiv}

%\usepackage[utf8]{inputenc} % allow utf-8 input
%\usepackage[T1]{fontenc} %\usepackage{ae} \usepackage{aecompl} % use 8-bit T1 fonts
%\usepackage{hyperref}       % hyperlinks
%\usepackage{url}            % simple URL typesetting
%\usepackage{booktabs}       % professional-quality tables
%\usepackage{amsfonts}       % blackboard math symbols
%\usepackage{nicefrac}       % compact symbols for 1/2, etc.
%\usepackage{microtype}      % microtypography
%\usepackage{lipsum}

\usepackage{bbm}           % allows to write the "blackboard 1" symbol \mathbbm{1}
\usepackage{amsthm}
\usepackage{amsmath}
%\usepackage{amssymb}  
\usepackage[]{algorithm2e}

%\usepackage{xfrac}

%\usepackage{pgf,tikz,pgfplots}
%\pgfplotsset{compat=1.14}
%\usepackage{mathrsfs} %provides nicer typesetting of calligraphic letters: $\mathscr{ABC}$
%\usetikzlibrary{arrows}

\hyphenation{block-chain}
\hyphenation{demo-cracy}

\newcommand{\R}{\mathbb{R}_{\geq 0}}
\newcommand{\eps}{\varepsilon}
\DeclareMathOperator*{\argmax}{arg\,max}
\DeclareMathOperator*{\argmin}{arg\,min}
\DeclareMathOperator*{\st}{stop}
\DeclareMathOperator{\ins}{Insert}
\DeclareMathOperator{\maxpscore}{MaxPscore}
\DeclareMathOperator{\maxprescore}{MaxPscore}
\DeclareMathOperator{\maxscore}{MaxScore}
\DeclareMathOperator{\interval}{FindInterval}
\DeclareMathOperator{\MMS}{MMS}
\DeclareMathOperator{\lazy}{LazyMMS}
\DeclareMathOperator{\phragmen}{seqPhragmen}
\DeclareMathOperator{\maxphragmen}{maxPhragmen}
\DeclareMathOperator{\phragmms}{Phragmms}
%\DeclareMathOperator{\balanced}{BalPhragmms}
\DeclareMathOperator{\LSPJR}{LS-Phragmms}
\DeclareMathOperator{\local}{LS-Phragmms}
%\DeclareMathOperator{\balancing}{balancing}
\DeclareMathOperator{\bal}{Bal}
\DeclareMathOperator{\supp}{supp}
\DeclareMathOperator{\score}{score}
\DeclareMathOperator{\pscore}{pscore}
\DeclareMathOperator{\prescore}{pscore}
\DeclareMathOperator{\slack}{slack}

\newtheorem{theorem}{Theorem}
\newtheorem{corollary}[theorem]{Corollary}
\newtheorem{lemma}[theorem]{Lemma}
\newtheorem*{lemma*}{Lemma}
\newtheorem{definition}[theorem]{Definition}
\newtheorem{remark}[theorem]{Remark}
\newtheorem*{heuristic}{Heuristic}

\title{A verifiably secure and proportional committee election rule}

\author{Alfonso Cevallos}
\email{alfonso@web3.foundation}
\orcid{0000-0001-8622-5830}
\affiliation{%
  \institution{Web 3.0 Technologies Foundation}
  \city{Zug}
  \country{Switzerland}
}
\author{Alistair Stewart}
\email{alistair@web3.foundation}
\affiliation{%
  \institution{Web 3.0 Technologies Foundation}
  \city{Zug}
  \country{Switzerland}
}

\begin{document}
\begin{abstract}

The concept of proportional representation in approval-based committee elections has appeared in the social choice literature for over a century and is typically understood as avoiding the underrepresentation of minorities. 
However, we argue that the security of some distributed systems critically depends on the opposite goal of preventing the overrepresentation of any minority, a goal not previously formalized that leads us to an optimization objective known as \emph{maximin support}. 
After a thorough analysis of its computational complexity, we propose a new efficient election rule 
%inspired in Phragm\'{e}n's methods 
that simultaneously achieves a) a constant-factor approximation guarantee for the objective, and b) the property of \emph{proportional justified representation} (PJR) --one of the strongest forms of proportional representation. However, the most striking feature of the new rule is that one can \emph{verify in linear time} in the input size that the winning committee satisfies the two aforementioned properties, even when the algorithm is executed by an untrusted party who only communicates the output. 
As a consequence, the rule can be adapted into a \emph{verifiable computing scheme}.
%Finally, we present an efficient post-computation that, when paired with any approximation algorithm for maximin support, returns a new solution that a) preserves the approximation guarantee and b) can be efficiently verified to satisfy PJR.

Our work is motivated by an application on blockchain networks that implement \emph{nominated proof-of-stake}, where the community elects a committee of validators to participate in the consensus protocol, and where preventing overrepresentation protects the network against attacks by an adversarial minority. 
Our election rule enables a validator selection protocol with formal guarantees on security and proportionality, and its adaptation as a verifiable computing scheme proves to be key for its successful implementation given the computationally limited nature of the blockchain architecture. 
We provide details of such an implementation in the \emph{Polkadot} network, launched in 2020.

\end{abstract}

\keywords{computational social choice, approval-based committee election, approximation algorithms, proof-of-stake, blockchain}

\maketitle


%\newpage

\section{Introduction}

The property of proportional representation in approval-based committee elections, discussed in the literature of computational social choice for a long time, is typically understood as guaranteeing that small minorities within the electorate are not \emph{underrepresented} by the winning committee. This is certainly an imperative in any fair election process as it ensures that all voices are heard and all communities are satisfied and engaged. 
In this paper, we complement this notion by formalizing the opposite goal of preventing the \emph{overrepresentation} of any minority. We consider this goal to be a matter of security, and analyze a scenario where a large adversarial minority may use strategic voting to attempt to gain overrepresentation, in order to capture the governance body or interfere with its correct functioning. 

To resist such an attack, we define an optimization problem where --in simple terms-- the amount of votes backing any seat in the committee is maximized. The objective function ensures that an individual's vote is not double-counted across multiple seats, which neutralizes the effect of strategic voting to a large extent. In this work we retrieve this objective function from first principles and provide a thorough analysis of its theoretical complexity. We also establish its connection to relevant axioms of proportional representation in the literature, and compare how popular election rules fare with respect to it. Finally, we present a new efficient election rule that provides theoretical guarantees in the prevention of both underrepresentation and overrepresentation, thus ensuring both proportionality and security.

Our work is motivated by an application on public, permissionless blockchain networks. These networks are equipped with computational and economic capabilities and have no central authority nor single point of failure, which provides them with unprecedented levels of decentralization and resistance to attacks, and for the first time removes the need for trusted intermediaries in peer-to-peer value transfer across the world. Recent years have seen an explosion of blockchain-based applications in finance, commerce, corporations, art and gaming. Rather than controlling the identity and correct execution of its nodes, a blockchain network freely allows nodes to join or leave the network pseudonymously, and adds enough redundancy to resist the erroneous execution of any one of them. Nodes that participate in the blockchain consensus protocol are denoted as \emph{validators}, and the correct functioning of the network is guaranteed as long as a supermajority of validators execute correctly.

Yet, the advantages of a blockchain architecture come at the cost of hard computational limitations. For every new block of transactions, all validators around the globe need to perform the same computations locally, and the network must wait for all of them to finish and agree on the results before processing the next block. Furthermore, a robust design must account for computationally weak validators, such as consumer-grade computers, as imposing high performance requirements would lead to centralization, so the per-block computational load must remain conservatively low. Because of this, the Bitcoin network for instance can process a maximum of 7 transactions per second. 

As a consequence, only the simplest and most efficient of protocols admit a straightforward implementation over a blockchain network. This represents a considerable hindrance for the use of committee election rules, in particular those sophisticated enough to provide guarantees on proportional representation. As a case in point, the $\MMS$ rule provides very strong guarantees on proportionality and security, as we establish in this paper, but its slow runtime makes it unsuitable for implementation. On the other hand, the EOS network implements the multiwinner Approval Voting rule to elect its validators, a rule that is highly efficient yet known to perform very poorly in terms of proportional representation. The use of this rule, in all likelihood selected due to performance considerations, has led to user dissatisfaction and claims of excessive centralization of the EOS network.%
\footnote{For an opinion piece on the topic, see the blogpost ``EOS voting structure encourages centralization'' by Priyeshu Garg at \url{https://cryptoslate.com/eos-voting-structure-encourages-centralization/}.}
The issue of implementability of election rules has become ever more relevant in recent time more and more blockchain projects abandon the notion of immutable code and in favor of designs that facilitate future code upgrades via governance processes. Many projects even embrace complex governance structures in which users can vote on committees and referenda, launch their own candidacies, raise proposals, form commissions, etc. 

Several solutions have been proposed and implemented for scaling up the computational capabilities of blockchain networks, such as sharded architectures and layer-2 solutions. Of relevance to our work is the use of \emph{verifiable computing schemes}. Such a scheme offloads a heavy task to one or more \emph{off-chain workers}, that is, entities that are logically separated from the network and may process the task on high-performance machines and/or with relaxed time frames as their operations do not interfere with regular block production. Once the task is completed and the output is loaded back onto the network, its correctness is verified by the validators. This is a sensible scheme for any task that admits a \emph{verification process} on its output that guarantees correctness, even when the task is performed privately by an untrusted party that only communicates such output, provided that this verification process is much more efficient than performing the task from scratch. A main contribution of this paper is showing that, from the winning committee that is output by our proposed election rule, along with the original input instance, one can verify the guarantees on proportionality and security in only linear time. Hence, our election rule can be successfully implemented as a verifiable computing scheme, and in fact it has been implemented as such in the Polkadot and Kusama networks, where it is applied in their validator election protocols; we include details of this implementation in one of our sections. Therefore, the present work constitutes a first effort in the study of verifiability within the election theory community. Naturally, giving anyone the power to efficiently verify the correctness of the results in an election process is a valuable goal in itself, beyond any concerns of implementation over blockchain networks.

\vspace{5mm}
\textbf{Problem definition and the validator election protocol.}
Let us now introduce our motivating application, out of which the theoretical goals and optimization problem that we pursue in this paper will emerge naturally. 
Many blockchain networks launched in recently years substitute the highly inefficient Proof-of-Work (PoW) component of Nakamoto's consensus protocol~\cite{nakamoto2019bitcoin} with Proof-of-Stake (PoS), in which validators participate in block production with a frequency proportional to their holdings of the native token, as opposed to their computational power. 
While a pure PoS system allows any willing token holder to participate directly as a validator, most projects place a hard bound $k$ on the number of validators that can be active at any given moment. Arguments for setting such a bound are that the increase in operational costs and communication complexity eventually outmatches the marginal increase in benefits stemming from decentralization as $k$ grows, and that it is typically observed in networks (both PoW- and PoS-based) with a large number of validators that the latter tend to form pools anyway, in order to decrease the variance on their revenue and profit from economies of scale. 
As an alternative, the system formalizes and facilitates the formation of these pools by allowing users to vote for candidates that they trust to represent them as validators, and then publicly electing a committee of $k$ validators. 
Networks that follow this approach include Polkadot, Cardano, EOS, Tezos, Cosmos, and Kusama, among many others. 
While similar in spirit, the validator election protocols in these networks vary in terms of design choices such as incentive structures, validator committee size $k$, and the election rule used. These choices are of the utmost importance as they affect the decentralization and security levels achieved by the network; yet, most projects lack any rigorous analysis behind their protocol design. An important contribution of this paper is precisely to provide such an analysis based on first principles.

We focus on Nominated Proof-of-Stake (NPoS), a variant of PoS introduced by Polkadot and Kusama~\cite{burdges2020overview}. In NPoS, any user is free to become a validator candidate, or a \emph{nominator} who provides an unranked list of any number of candidates that she trusts. At regular intervals of a few hours, a committee of $k$ validators --in the order of hundreds or thousands-- is elected according to the current nominators' votes. As a security measure, both validators and nominators have their tokens staked as collateral, so that if a validator ever shows negligent or adversarial behavior, all backing tokens are susceptible to being lost. Conversely, during normal execution the network provides economic rewards to all elected validators and their backing nominators on a pro-rata basis. Nominators are thus indirect participants in the consensus protocol with an economic incentive to follow the performance of validators and support only the most capable and trustworthy candidates. For the sake of simplicity, in what follows we consider a model where only nominators have staked tokens, not candidates, and we equate their stake amount to their voting strength. This leads to a vote-weighted, approval-based committee election problem. As mentioned before, our two goals behind this problem are proportionality and security. We formalize each of these goals next. 

\emph{Proportionality.} We aim to guarantee that nominators are not \emph{underrepresented} relative to their stake by the elected validators. We highlight that diverse preferences and factions may naturally arise among nominators for reasons that range from economically and technically motivated to political, geographical, etc., and that preserving this diversity among the validators helps ensure that the blockchain network stays decentralized. Electoral system designs that achieve some version of proportional representation have been present in the literature of social choice for a very long time. Of special note is the work of Scandinavian mathematicians Edvard Phragm\'{e}n and Thorvald Thiele in the late nineteenth century \cite{phragmen1894methode, phragmen1895proportionella, phragmen1896theorie, phragmen1899till, thiele1895om, janson2016phragmen}. 
Several axioms have been recently proposed to define the property mathematically -- we mention the most relevant ones. 
\emph{Justified representation} (JR)~\cite{aziz2017justified} states that if a group of voters is cohesive enough in terms of candidate preferences and has a large enough aggregate vote strength, then it has a justified claim to be represented by a member of the committee.
\emph{Proportional justified representation} (PJR)~\cite{sanchez2017proportional} says that such a group deserves not just one but a minimum number of representatives according to its vote strength, where a committee member is said to represent the group as long as it represents any voter in it.
Finally, \emph{extended justified representation} (EJR)~\cite{aziz2017justified} strengthens this last condition and requires not only that the group have enough representatives collectively, but some voter in it must have enough representatives individually.
It is known that EJR implies PJR and PJR implies JR, but converse implications are not true~\cite{sanchez2017proportional}. %
For each of these properties, a committee voting rule is said to satisfy said property if its output committee always satisfies it for any input instance. 
While the most common voting rules usually achieve JR, they fail the stronger properties of PJR and EJR, and up to recently there were no known efficient voting rules that satisfy the latter two. 
For instance, the \emph{proportional approval voting} (PAV) method \cite{thiele1895om, janson2016phragmen} proposed by Thiele satisfies EJR but is NP-hard to compute, while efficient heuristics based on it, such as reweighted approval voting, fail PJR \cite{aziz2014computational, skowron2016finding, aziz2017justified}. 
Only very recently have efficient algorithms that achieve PJR or EJR finally been proposed \cite{brill2017phragmen, sanchez2016maximin, aziz2018complexity, peters2019proportionality}. 

Among these axioms, \textbf{we set to achieve PJR}, defined formally in Section~\ref{s:prel}, for two reasons. 
First, because it is more \emph{Sybil resistant}~\cite{douceur2002sybil} than JR, meaning that in our blockchain application a strategic voter may be incentivized to assume several nominator identities in the network under JR, but not under PJR. 
Second, because PJR seems to be most compatible with our security objective, as we argue below. Indeed, as argued in~\cite{peters2019proportionality} and \cite{lackner2020approval}, the PJR and EJR axioms correspond to different notions of proportionality: while EJR is primarily concerned with the general welfare or satisfaction of the voters, PJR considers proportionality of the voters' decision power, and our security objective aligns best with the latter notion.

\emph{Security.} 
As is the case in any PoS-based blockchain network, under NPoS the basic security assumption is that most of the tokens are held by actors who behave honestly or rationally. Under this assumption, we consider a scenario where an adversary with a limited token budget attempts to carry out an attack on the network. Depending on the attack, it will require control of some minimum number of validators to succeed, so the adversary first needs to procure itself said number of representatives in the elected committee, and it may recur to strategic voting to achieve this. Therefore, the security level corresponds to how difficult it is for a voter or group of voters with limited aggregate voting strength to gain \emph{overrepresentation} in the elected committee. 

Further formalizing our election problem, we consider finite sets $N$ and $C$ of voters and candidates respectively, where every voter $n\in N$ provides a list $C_n\subseteq C$ of approved candidates and has a vote strength $s_n$. 
%There is also a target number $1\leq k< |C|$ of candidates to elect.
Suppose we want to make it as difficult as possible for an adversary to gain a certain threshold $1\leq r\leq k$ of representatives within the $k$-validator committee. 
Then, our goal would be to elect a committee $A\subseteq C$ that maximizes 
$$\min_{A'\subseteq A, |A'|=r} \sum_{n\in N: \ C_n\cap A'\neq \emptyset} s_n.$$ 

For any subset $A'\subseteq A$ of $r$ seats in committee $A$, the quantity above is the total vote strength that is backing any seat in $A'$, with no vote being double counted. In our blockchain application, this quantity also corresponds to the total collateral susceptible to being lost if $A'$ carries on an attack; hence, maximizing this amount not only makes it difficult for the adversary to gain enough representatives, but also costly to carry the attack if it does. Of course, we remark that on top of the potential collateral loss resulting from an attack, rational validators should consider the potential loss of reputation, which translates to further loss of future payouts. We thus obtain a different optimization objective for each value of threshold $r$. 
If we are only concerned about a particular threshold $r$, then we can fix the corresponding objective. 
For example, for $r=1$, maximizing this objective is equivalent to the classical multiwinner approval voting: selecting the $k$ candidates $c\in C$ with highest total approval $\sum_{n\in N: \ c\in C_n} s_n$. 
Or, we could set $r=\lceil k/3\rceil$, which is the minimum threshold required to carry on a successful attack in classical Byzantine fault tolerant protocols~\cite{pease1980reaching}. 
However, different types of attacks require different thresholds, and some attacks succeed with higher probability with more attacking validators. Hence, a more pragmatic approach is to incorporate the threshold into the objective and maximize \emph{the least possible cost per seat over all thresholds}, i.e.,  
\begin{align}\label{eq:security}
    \text{Maximize } \min_{A'\subseteq A, A'\neq \emptyset} \quad \frac{1}{|A'|} \sum_{n\in N: \ C_n\cap A' \neq \emptyset} s_n, \quad \text{over all committees $A\subseteq C$ with $|A|=k$}.
\end{align}

We establish in Lemma~\ref{lem:equivalence} that this objective is equivalent to the \textbf{maximin support objective}, recently introduced by Sánchez-Fernández et al.~\cite{sanchez2016maximin}, and which we thus set to optimize. 
We define it formally in Section~\ref{s:prel}.
%To define this last objective, which we do formally in Section~\ref{s:prel}, one needs the election rule to establish not only a winning committee $A\subseteq C$, but also a \emph{vote distribution}; that is, a fractional distribution of each voter $n$'s vote strength $s_n$ among her approved committee members in $C_n\cap A$.%
%\footnote{This is called a \emph{support distribution function} in~\cite{sanchez2016maximin}, and is related to the notion of a \emph{price system} in~\cite{peters2019proportionality}.} 
%For instance, for voter $n$ the election rule may assign a third of $s_n$ to $c_1$ and two thirds of $s_n$ to $c_2$, where $c_1, c_2\in C_n\cap A$. 
%The objective is then to maximize, over all possible committees and distributions, the least amount of vote assigned to any committee member. 
%We observe here that unlike most other applications of multiwinner elections, in NPoS there is practical utility in computing a vote distribution from nominators to the elected validators: by reversing its sense, it establishes the exact way in which the validators' payouts or penalties must be distributed back to the nominators.
The authors in~\cite{sanchez2016maximin} remark that in its exact version, maximin support is equivalent to another objective, $\maxphragmen$, devised by Phragm\'{e}n and recently analyzed in~\cite{brill2017phragmen}, and in this last paper it is shown that $\maxphragmen$ is NP-hard and incompatible with EJR. 
Thus, the same hardness and incompatibility with EJR holds true for our security objective. 
%To the best of our knowledge, the approximability of maximin support has not previously been studied.

\vspace{5mm}
\textbf{Our contribution.}
Our analysis of security of the validator election protocol leads us to pursue the maximin support objective, which fights overrepresentation. Conversely, we equate our proportionality goal with the PJR property, which fights underrepresentation. We show that these two goals complement each other well, and prove the existence of efficient election rules that achieve guarantees for both of them. 

\begin{theorem}\label{thm:intro1}
There is an efficient election rule for approval-based committee elections that simultaneously achieves the PJR property and a 3.15-factor approximation guarantee for the maximin support objective.
\end{theorem}

Our proposed election rule is inspired in the $\phragmen$ method~\cite{brill2017phragmen}, and to the best of our knowledge we present the first analysis of approximability for a Phragm\'{e}n objective. 
In contrast, several approximation algorithms for Thiele objectives have been proposed; see~\cite{lackner2020approval} for a survey. To complement this result, we also prove that a constant-factor approximation is theoretically best possible for the maximin support objective. 
%
Next comes the question of applicability: as we previously mentioned, the blockchain architecture adds very stringent time constraints to computations. However, if the output can be \emph{verified} much faster than it can be obtained from scratch, then the computation can be implemented as a verifiable computing scheme. This is the case for our election rule.

\begin{theorem}\label{thm:intro2}
There is a linear-time test that takes as input an election instance and an arbitrary solution to it, such that if the test passes then the input solution satisfies the PJR property and a 3.15-factor approximation guarantee for the maximin support objective. 
Moreover, the output of the election rule mentioned in Theorem~\ref{thm:intro1} always passes this test.
\end{theorem}

We remark that a solution passing said test is a sufficient but not a necessary condition for it to have the properties above, hence the fact that the output of our election rule always passes the test is not straightforward.
In the context of our blockchain application, this result enables the implementation of a fast validator election protocol with strong theoretical guarantees on security and proportionality. 
We provide details of such an implementation over the Polkadot network in Section~\ref{s:objectives}.
%
Finally, we derive from the new election rule a post-computation which, when paired with any approximation algorithm for the maximin support problem, makes it also satisfy the PJR property in a black-box manner.

\begin{theorem}\label{thm:intro3}
There is an efficient computation that takes as input an election instance and an arbitrary solution to it, and outputs a new solution which a) is no worse than the input solution in terms of the maximin support objective, b) satisfies the PJR property, and in particular c) can be efficiently tested to satisfy the PJR property.
\end{theorem}

%This result shows that the PJR property is strongly compatible with the maximin support objective, and can be easily added to future approximation algorithms that may be developed for this objective.

\vspace{5mm}
\textbf{Organization of the paper and technical overview.}
In Section~\ref{s:prel} we formalize our multiwinner election problem and its objectives and provide all the required technical definitions. Then, in Section~\ref{s:complexity} we present a thorough analysis of the complexity of the maximin support objective, where we exhibit constant-factor approximation algorithms for it, as well as prove a hardness result showing that a constant-factor guarantee is best possible. 
We also compare the performance of several common election rules with respect to this objective, and show that in fact most of them do not achieve constant-factor approximations for it.
In Section~\ref{s:heuristic} we propose a new heuristic, $\phragmms$, inspired in $\phragmen$~\cite{brill2017phragmen} but allowing for a more robust analysis and better guarantees than the latter both in terms of the PJR property and the maximin support objective, and we use it to prove Theorem~\ref{thm:intro1}. 

In these sections we formalize the \emph{load balancing} technique used in $\phragmen$ in terms of network flow, give a precise definition of what a \emph{balanced solution} is, and provide an algorithm to compute one efficiently using notions of parametric flow. 
We then synthesize the strategies of different relevant heuristics in terms of how they compute or approximate balanced solutions, and use the flow decomposition theorem to establish approximation guarantees. 
Our comparison provides new tools to discern between election rules. For instance, the survey paper by Lackner and Skowron~\cite{lackner2020approval} mentions $\phragmen$ and $\MMS$~\cite{sanchez2016maximin} as two efficient rules that achieve the PJR property and leaves as an open question which of the two is preferable, whereas we show that the latter provides a constant-factor approximation guarantee for maximin support but the former does not. 
Furthermore, contrary to the common notion that the EJR property is categorically superior than PJR, we establish PJR as the better choice in terms of security.

In Section~\ref{s:local} we prove Theorems \ref{thm:intro2} and \ref{thm:intro3}, and explore how the PJR property and a constant-factor approximation guarantee for maximin support can be efficiently tested, even if the algorithm is privately executed by an untrusted party who only communicates the solution. 
To do so, we define a parametric version of the PJR property, and link it to a notion of local optimality for our new heuristic. Finally, in Section~\ref{s:objectives} we provide details on how our proposed voting rule has been implemented in the validator election protocol of Polkadot, as a verifiable computing scheme.
%We present some conclusion and open question in Section~\ref{s:conclusions}.


\section{Preliminaries}\label{s:prel}

Throughout the paper we consider the following approval-based multiwinner election instance. 
We are given a bipartite approval graph $G=(N\cup C, E)$ where $N$ is a set of voters and $C$ is a set of candidates. 
We are additionally given a vector $s\in\R^N$ of vote strengths, where $s_n$ is the strength of $n$'s vote, and a target number $k$ of candidates to elect, where $0< k<|C|$.
For each voter $n\in N$, $C_n:=\{c\in C: \ nc\in E\}$ represents her approval ballot, i.e.~the subset of candidates that $n$ approves of, and for each candidate $c\in C$ we denote by $N_c:=\{n\in N: \ nc\in E\}$ the set of voters approving $c$, where $nc$ is shorthand for edge $\{n,c\}$. 
To avoid trivialities, we assume that graph $G$ in the input has no isolated vertices. 
For any $c\in C\setminus A$, we write $A+c$ and $A-c$ as shorthands for $A\cup\{c\}$ and $A\setminus \{c\}$ respectively. 

\emph{Proportional justified representation (PJR).} 
The PJR property was introduced in~\cite{sanchez2017proportional} for voters with unit vote strength. We present its natural generalization to arbitrary vote strengths. A committee $A\subseteq C$ of $k$ members satisfies PJR if there is no group $N'\subseteq N$ of voters and integer $0<r\leq k$ such that:
%\begin{itemize}
%\item[a)] $\sum_{n\in N'} s_n \geq \frac{r}{k} \sum_{n\in N}s_n$,
%\item[b)] $|\cap_{n\in N'} C_n|\geq r$, and
%\item[c)] $|A\cap (\cup_{n\in N'} C_n)|<r$.
%\end{itemize}
%
$$\text{a) } \sum_{n\in N'} s_n \geq \frac{r}{k} \sum_{n\in N}s_n, \quad \quad \text{b) } |\cap_{n\in N'} C_n|\geq r, 
\quad \quad \text{and c) } |A\cap (\cup_{n\in N'} C_n)|<r.$$
%
In words, if there is a group $N'$ of voters with at least $r$ commonly approved candidates, and enough aggregate vote strength to provide each of them with a vote support of value $\hat{t}:=\sum_{n\in N} s_n / k$, then this group has a justified right to be represented by at least $r$ members in committee $A$, though not necessarily commonly approved. 
Notice that $\hat{t}$ is an upper bound on the average vote support that voter set $N$ can possibly provide to any committee of $k$ members. 

\emph{Maximin support objective.} 
For the given instance, we consider a solution consisting of a tuple $(A,w)$, where $A\subseteq C$ is a committee of $k$ elected candidates, and $w\in\R^E$ is a vector of non-negative edge weights that represents a fractional distribution of each voter's vote among her approved candidates.%
\footnote{This weight vector is related to the notions of \emph{support distribution function} in~\cite{sanchez2016maximin} and \emph{price system} in~\cite{peters2019proportionality}. In particular, all voting rules considered in this paper are \emph{priceable}, as defined in~\cite{peters2019proportionality}.} 
For instance, for voter $n$ this distribution may assign a third of $s_n$ to $c_1$ and two thirds of $s_n$ to $c_2$, where $c_1, c_2\in C_n$.
Vector $w$ is considered \emph{feasible}%
\footnote{Intuitively, a feasible solution $(A,w)$ should also observe $w_{nc}=0$ for each edge $nc$ with $c\not\in A$. 
However, as this constraint can always be enforced in post-computation, we ignore it so that the feasibility of a vector $w$ is independent of any committee.} 
if  % 
%
\begin{equation}
    \sum_{c\in C_n} w_{nc}\leq s_n \quad \text{ for each voter } n\in N. \label{eq:feasible}
\end{equation}

In our analyses, we will also consider \emph{partial} committees, with $|A|\leq k$. If $|A|=k$, we call it \emph{full}. 
All solutions $(A,w)$ in this paper are assumed to be feasible and full unless stated otherwise. 
Given a (possibly partial, unfeasible) solution $(A,w)$, we define the \emph{support} over the committee members as 
\begin{equation}
supp_w(c):=\sum_{n\in N_c} w_{nc} \quad \text{for each $c\in A, \quad$ and } \quad supp_w(A):=\min_{c\in A} supp_w(c), \label{eq:support}
\end{equation}
where we use the convention that $supp_w(\emptyset)=\infty$ for any weight vector $w\in\R^E$. 
The maximin support objective, introduced in~\cite{sanchez2016maximin}, asks to maximize the least member support $supp_w(A)$ over all feasible full solutions $(A,w)$. 

\emph{Balanced solutions.}
For a fixed committee $A$, a feasible weight vector $w\in\R^E$ that maximizes $supp_w(A)$ can be found efficiently. In this paper we seek additional desirable properties on a weight vector which can still be achieved efficiently. We say that a feasible $w\in\R^E$ is \emph{balanced for $A$}, or that $(A,w)$ is a balanced solution, if
\begin{enumerate}
    \item it maximizes the sum of member supports, $\sum_{c\in A} supp_w(c)$, over all feasible weight vectors, and 
    \item it minimizes the sum of supports squared, $\sum_{c\in A} (supp_w(c))^2$, over all vectors that observe the point above. 
\end{enumerate}

In other words, a balanced weight vector maximizes the sum of supports and then minimizes their variance. 
In the next lemma, whose proof is delayed to Appendix~\ref{s:proofs}, we establish some key properties that we exploit in our analyses. 

\begin{lemma}\label{lem:balanced}
Let $(A,w)$ be a balanced, possibly partial solution. Then,
\begin{enumerate}
    \item for each $1\leq r\leq |A|$, vector $w$ simultaneously maximizes $\min_{A'\subseteq A, |A'|=r} \ \sum_{c\in A'} supp_{w'}(c)$ over all feasible weight vectors $w'\in\R^E$; 
		\item the sum of member supports must be exactly $\sum_{c\in A} supp_w(c)=\sum_{n\in \cup_{c\in A} N_c} s_n$; and
    \item if there is a voter $n\in N$ and candidates $c\in C_n$, $c'\in C_n\cap A$ with $supp_w(c)>supp_w(c')$, then $w_{nc}=0$.
\end{enumerate}
Furthermore, a feasible solution $(A,w)$ is balanced if and only if it observes properties 2 and 3 above.
\end{lemma}

Notice that by setting $r=1$ on the first point, we obtain that balanced vector $w$ indeed maximizes the least member support $supp_w(A)$ over all feasible weight vectors. 
More generally, for each $r$ the quantity defined in the first point defines a lower bound on the cost for an adversary to get $r$ representatives in the validator committee in NPoS, so maximizing these objectives for all thresholds $r$ aligns with our security objective as it makes any attack as costly as possible. 
The second point follows from the fact that the sum of member supports is maximal, so the feasibility inequality~\eqref{eq:feasible} must be tight for each voter $n$ with approved candidates in $A$. The third point is a consequence of having the supports as evenly distributed as possible within $A$: if candidate $c$ has a higher support than another candidate $c'$, then none of $n$'s vote can go to $c$, as all of it must have been assigned to $c'$ or to other members with lower support. 

In Appendix~\ref{s:balanced} we present new algorithms for computing a balanced weight vector for a given committee $A$. 
In particular, we prove that one can be found in time $O(|E|\cdot k + k^3)$ using parametric flow techniques, which to the best of our knowledge is the current fastest algorithm in the literature even for the simpler problem of maximizing $supp_w(A)$.

\begin{remark}\label{rem:bal}
In the remainder of the paper, we denote by $\bal$ the time complexity of finding a balanced weight vector, which will depend on the precise algorithm used.
\end{remark}

\emph{Equivalence of objectives.} 
We now establish that our security objective~\eqref{eq:security} is indeed equivalent to maximin support. 
For this we use the fact that, in view of point 1 in Lemma~\ref{lem:balanced}, in the maximin support objective we can reduce the solution space to only balanced solutions without loss of generality. The proof of the next lemma is delayed to Appendix~\ref{s:proofs}.

\begin{lemma} \label{lem:equivalence} If $(A,w)$ is a balanced solution, then
$$supp_w(A) = \min_{\emptyset\neq A' \subseteq A} \frac{1}{|A'|} \sum_{n\in \cup_{c\in A'} N_c} s_n.$$
Consequently, maximin support, the problem of maximizing the left-hand side over all balanced full solutions $(A,w)$, 
is equivalent to the problem of maximizing the right-hand side over all full committees $A$. 
Furthermore, this equivalence preserves approximations, as any balanced solution $(A,w)$ provides the same objective value to both problems.
\end{lemma}

\emph{Network flows.}
In many proofs we deal with a vector $f\in\mathbb{R}^{E}$ of edge weights over the input graph $G=(N\cup C,E)$, which we regard as a vector of flows with positive signs considered to be flow directed toward $C$, and negative signs as flow directed toward $N$. 
Consequently, the \emph{excess} of a voter $n\in N$ relative to $f$ is $f(n):=\sum_{c\in C_n} f_{nc}$, and the excess of a candidate $c\in C$ is $f(c):= -\sum_{n\in N_c} f_{nc}$. 
A set of vertices $S\subseteq N\cup C$ has \emph{net excess} if $\sum_{x\in S} f(x)>0$, and it has \emph{net demand} if $\sum_{x\in S} f(x)<0$.    
A vector $f'\in\mathbb{R}^E$ is a \emph{sub-flow of $f$} if a) for each edge $e\in E$ with $f'_e\neq 0$, flows $f'_e$ and $f_e$ have the same sign and $|f'_e|\leq |f_e|$, and b) for each vertex $x\in N\cup C$ with $f'(x)\neq 0$, excesses $f'(x)$ and $f(x)$ have the same sign and $|f'(x)|\leq |f(x)|$. 
The proof of the next lemma is delayed to Appendix~\ref{s:proofs}.

\begin{lemma}\label{lem:subflow}
If weight vectors $w, w'\in\R^E$ are non-negative and feasible for the given instance, and $f'\in\mathbb{R}^E$ is a sub-flow of $f:=w'-w$, then both $w+f'$ and $w'-f'$ are non-negative and feasible as well.
\end{lemma}

\begin{remark}
In all algorithms analyzed, we assume that all numerical operations take constant time.
\end{remark}


\section{Analysis of maximin support}\label{s:complexity}

Consider a multiwinner election instance $(G=(N\cup C, E), s, k)$ as defined in Section~\ref{s:prel}. 
In this section we present a computational complexity analysis of the maximin support problem, including new results both on approximability and on hardness. 

We start by establishing that our security objective \eqref{eq:security} is indeed equivalent to maximin support.
For this we use the fact that, by property 1 in Lemma~\ref{lem:balanced}, in maximin support we can reduce the solution space to only balanced solutions without loss of generality. 
The proof of the next theorem is delayed to Appendix~\ref{s:proofs}.

\begin{theorem} \label{thm:equivalence} 
If $(A,w)$ is a balanced solution, then
$$\supp_w(A) = \min_{\emptyset\neq A' \subseteq A} \frac{1}{|A'|} \sum_{n\in \cup_{c\in A'} N_c} s_n.$$
Hence maximin support, the problem of maximizing the left-hand side over all balanced full solutions $(A,w)$, 
is equivalent to that of maximizing the right-hand side over all full committees $A$. 
Furthermore, this equivalence preserves approximations, as any balanced solution $(A,w)$ provides the same objective value to both problems.
\end{theorem}

Next, we compare how some of the most relevant efficient election rules perform in terms of security, and highlight the effectiveness of the maximin support objective for preventing overrepresentation. 
Fix a committee size $k$ and consider an unweighted instance with $k+1$ voters and $2k$ candidates. 
For $1\leq i\leq k$, voter $n_i$ supports the candidate set $\{c_1, \cdots, c_i\}$, and the last voter $n'$ supports the candidate set $\{c_1', \cdots, c_k'\}$. See Figure~\ref{fig:example}.
%
We assume that an adversary controls the last voter and the last $k$ candidates, and will use any elected representatives to attempt to disrupt the duties of the committee. 
How many representatives will it get? 

\begin{figure}[tb]
  \centering
	\includegraphics[width={0.45\linewidth},natwidth=180,natheight=220]{figure-example.pdf}
  \caption{In this example, the last voter and the last $k$ candidates are considered adversarial.}
  \label{fig:example}
\end{figure}

\begin{lemma}\label{lem:badexamples}
For any $\alpha \geq 1$, in the example above the number of adversarial candidates elected by a rule with an $\alpha$-approximation guarantee for maximin support is at most $\lfloor \alpha \rfloor$. 
On the other hand, this number is $\Omega(\sqrt{k})$ for proportional approval voting (PAV), and $\Omega(\log k)$ for both $\phragmen$ and Rule X. 
Hence, none of these three rules guarantees a constant-factor approximation. 
\end{lemma}

The proof is delayed to Appendix~\ref{s:proofs}. 
For definitions of these rules, we direct the reader to the survey paper~\cite{lackner2020approval}.
We only remark here that Rule X~\cite{peters2019proportionality} is a recently proposed rule inspired in $\phragmen$ -- much as our own rule presented in the next section -- that achieves the EJR property. 

The maximin support problem was introduced in~\cite{sanchez2016maximin}, where it was observed to be NP-hard. We show now a stronger hardness result for it, which in particular rules out the existence of a PTAS.%
\footnote{A \emph{polynomial time approximation scheme} (PTAS) for an optimization problem is an algorithm that, for any constant $\eps>0$ and any given instance, returns a $(1+\eps)$-factor approximation in polynomial time.} 


\begin{theorem}
For any constant $\eps>0$, it is NP-hard to approximate the unweighted maximin support problem within a factor $\alpha=1.2-\eps$.
\end{theorem}

\begin{figure}[tb]
  \centering
	\includegraphics[width=\linewidth,natwidth=400,natheight=185]{figure-hardness.pdf}
  \caption{Reducing an instance of the $k$-independent set problem on cubic graphs to one of the maximin support problem. Set $N$ is represented by triangles and $C$ by circles.}
  \label{fig:hardness}
\end{figure}

\begin{proof}
We present a reduction from the $k$-independent set problem on cubic graphs, which is known to be NP-hard~\cite{johnson1979computers}. In this problem, one is given a graph $G'=(V',E')$ where every vertex has degree exactly 3, and a parameter $k'$, and one must decide whether there is a vertex subset $I\subseteq V'$ of size $k'$ such that no two vertices in $I$ are adjacent, i.e., $I$ is an independent set. 
Given such an instance, we define an instance $(G=(N\cup C, E), s, k)$ of maximin support where $k=k'$, $C=V'$ (each vertex in $V'$ corresponds to a candidate), and $N=E'$ with $s_n=1$ and $C_n=n$ for each $n\in N$ (each edge in $E'$ corresponds to a voter with unit vote that approves of the two candidates on its endpoints); see Figure~\ref{fig:hardness}.
Notice that in this instance, each candidate is approved by exactly 3 voters, and two candidates $c, c'$ have an approving voter in common if and only if $c$ and $c'$ are adjacent in $V'$.

Hence, if there is an independent set $I$ of size $k$ in $G'$, the same committee of validators in $G$ can be assigned the full vote of each of its three approving voters, so that each receives a support of 3 units, which is clearly maximal. On the other hand, if there is no independent set of size $k$ in $G'$, then for any solution $(A,w)$ of the maximin support instance there must be two committee members $c,c'\in A$ who have an approving voter in common. These two members have at most five voters approving either of them, so one of them must have a support of at most $5/2$. This shows that $\supp_w(A)\leq 5/2$ for any feasible solution $(A,w)$. Finally, the ratio between the objective values $3$ and $5/2$ is $6/5=1.2>\alpha$, so the assumed $\alpha$-approximation algorithm for maximin support would allow us to distinguish between these two cases and decide whether such an independent set $I$ exists. This completes the proof.
\end{proof}

In contrast, we show that the recently proposed $\MMS$ rule~\cite{sanchez2016maximin}, known to achieve the PJR property, also provides a 2-approximation for maximin support. 
In simple terms, $\MMS$ (Algorithm~\ref{alg:mms}) starts with an empty committee $A$ and iteratively adds candidates to it; in each iteration, it computes a balanced weight vector for each possible augmented committee that can be obtained by adding a candidate, and then inserts the candidate whose corresponding augmented committee has the highest least member support.
%
We will need the following key technical result, whose proof uses the flow decomposition theorem and is delayed to Appendix~\ref{s:flow}. 


\begin{algorithm}[tb]
\SetAlgoLined
\KwData{Bipartite approval graph $G=(N\cup C, E)$, vector $s$ of vote strengths, target committee size $k$.}
Initialize $A=\emptyset$\ and $w=0\in \R^E$\;
\For{$i$ from $1$ to $k$}{
	\For{each candidate $c\in C\setminus A$}{
		Compute a balanced\footnotemark ~weight vector $w_c$ for $A+c$\;
		}
Find $c_i\in \arg\max_{c\in C\setminus A} \supp_{w_c}(A+c)$\;
Update $A\leftarrow A+c_i$ and $w\leftarrow w_{c_i}$\;
}
\Return $(A,w)$\;
\caption{$\MMS$, proposed in~\cite{sanchez2016maximin}}
\label{alg:mms}
\end{algorithm}
\footnotetext{The original algorithm in~\cite{sanchez2016maximin} does not compute balanced weight vectors, but any vector $w$ that maximizes $\supp_w(A)$, which is indeed sufficient for our analysis. 
However, we propose the use of balanced vectors here as they achieve further desirable properties (Lemma \ref{lem:balanced} and Theorem \ref{thm:equivalence}) and because adding such requirement does not seem to cause any additional overhead in complexity.}



\begin{lemma}\label{lem:2sols}
If $(A^*, w^*)$ is an optimal solution to maximin support, and $(A,w)$ is a partial solution with $|A|\leq k$ and $A\neq A^*$, there is a candidate $c'\in A^*\setminus A$ and feasible solution $(A+c', w')$ such that 
$$\supp_{w'}(A+c')\geq \min\Big\{\supp_w(A), \frac{1}{2} \supp_{w^*}(A^*)\Big\}.$$
\end{lemma}

\begin{theorem}\label{thm:mms}
The $\MMS$ rule provides a 2-approximation for maximin support.
\end{theorem}

\begin{proof}
Let $(A_i, w_i)$ be the partial solution at the end of the $i$-th round of MMS, and let $(A^*, w^*)$ be an optimal solution. 
We prove by induction on $i$ that $\supp_{w_i}(A_i)\geq \frac{1}{2}\supp_{w^*}(A^*)$, where the case $i=0$ holds trivially as we use the convention that $\supp_w(\emptyset)=\infty$.
Assuming that the inequality holds for $i$, an application of Lemma~\ref{lem:2sols} for $(A_i, w_i)$ and $(A^*, w^*)$ implies that there is a candidate $c'\in A^*\setminus A_i$ and a feasible solution $(A_i+c', w')$ such that 
\begin{align*}
\supp_{w'}(A_i+c') &\geq \min\Big\{\supp_{w_i}(A_i), \frac{1}{2} \supp_{w^*}(A^*)\Big\} \\
 &= \frac{1}{2} \supp_{w^*}(A^*).
\end{align*}
%
As the algorithm is bound to inspect candidate $c'$ in round $i+1$, and compute for it a balanced weight vector $w_{c'}$ which maximizes the support of $A_i+c'$ (by Lemma~\ref{lem:balanced}), the solution $(A_{i+1}, w_{r+1})$ at the end of round $i+1$ must have an even higher support, i.e., %
\begin{align*}
\supp_{w_{i+1}}(A_{i+1}) &\geq \supp_{w_{c'}}(A_i+c) \\ 
  &\geq \supp_{w'}(A_i+c) \geq \frac{1}{2} \supp_{w^*}(A^*).
\end{align*}
%
This completes the proof.
\end{proof}

$\MMS$ is an greedy algorithm with a runtime of $O(\bal \cdot |C|\cdot k)$, where we recall that $\bal$ is the time complexity of computing a balanced weight vector; see Remark~\ref{rem:bal}. 
To conclude the section we mention that a "lazy greedy" version of it can save a factor $\Theta(k)$ in the runtime while keeping the approximation guarantee virtually unchanged. 
In Appendix~\ref{s:lazymms} we prove the following result.

\begin{theorem}\label{thm:2eps}
There is an algorithm $\lazy$ that, for any $\eps>0$, offers a $(2+\eps)$-approximation for the maximin support problem, satisfies the PJR property, and executes in time $O(\bal\cdot |C|\cdot \log(1/\eps))$.
\end{theorem}



\section{A new heuristic for candidate selection}\label{s:heuristic}

Prior to this paper, the only polynomial-time algorithms known to achieve the PJR property were $\phragmen$~\cite{brill2017phragmen} and $\MMS$~\cite{sanchez2016maximin} (Algorithms \ref{alg:phragmen} and \ref{alg:mms} respectively). 
Both methods build a solution by starting with an empty committee and iteratively adding to it a new candidate over $k$ rounds, following some specific rule for candidate selection. 
In this section we introduce a new rule to select a candidate to add to a given partial solution. 
This heuristic constitutes the basis for our results in the next two sections. 

We start with a brief analysis of the approaches taken in $\MMS$ and $\phragmen$. For a given partial solution, $\MMS$ computes a balanced weight vector for each possible augmented committee resulting from adding a candidate, and keeps the one with largest support. Naturally, this heuristic provides great control over the evolution of the maximin support objective (including a 2-approximation guarantee, see Theorem~\ref{thm:mms}), but is relatively slow, as computing balanced weight vectors is costly. On the other hand, $\phragmen$ foregoes balancing and performs only minimal modifications to the current weight vector to adapt it to the new augmented committee, and hence it is very fast but provides poor guarantees over the maximin support objective. Our heuristic also foregoes balancing, and indeed is almost as fast as $\phragmen$, but provides greater control over the maximin support objective.

In all algorithms described in this section, we assume that there is a known background instance $(G=(N\cup C, E), s, k)$ that does not need to be passed as input. Rather, the input is a partial solution $(A,w)$ with $|A|\leq k$. We also assume that the list of committee member supports $(supp_w(c))_{c\in A}$ is implicitly passed by reference and updated in every algorithm, so it does not need to be recomputed every time.

Let $c'\in C\setminus A$ be a candidate that we consider adding to $(A,w)$. To do so, we need to modify weight vector $w$ to a new feasible vector $w'$ that redirects towards $c'$ some of the votes of voters in $N_{c'}$, in turn decreasing the support of other committee members approved by these voters. Now, for a given threshold $t\geq 0$, we want to make sure not to reduce the support of any member $c$ below $t$, assuming it starts above $t$, and not to reduce it at all otherwise. A simple rule to ensure this is as follows: for each voter $n$ in $N_{c'}$ and each member $c\in A\cap C_n$, reduce the weight on edge $nc$ from $w_{nc}$ to $w_{nc}\cdot \min\{1, t/supp_w(c)\}$, and assign the difference to edge $nc'$. That way, it is clear that even if all edges incident to a member $c$ are so reduced in weight, the support of $c$ is scaled by a factor at most $\min\{1, t/supp_w(c)\}$ and hence its new support does not fall below $t$.

Thus, if for each $n\in N$ and $t\geq 0$ we define that voter's \emph{slack} as

\begin{align}
    slack_{(A,w)}(n,t):= s_n - \sum_{c\in A\cap C_n} w_{nc} \cdot\min \Big\{ 1, t/supp_w(c)\Big\} \label{eq:slack}
\end{align}
%
and for each $c'\in C\setminus A$ and $t\geq 0$ we define that candidate's \emph{pre-score} as
%
\begin{equation}\label{eq:prescore}
    prescore_{(A,w)}(c',t) := \sum_{n\in N_{c'}} slack_{(A,w)}(n,t),
\end{equation}
%
then we can add $c'$ to the solution with support $prescore_{(A,w)}(c',t)$, while not making any other member's support decrease below threshold $t$. The resulting weight modification rule is formalized in Algorithm~\ref{alg:ins}. The next lemma follows from the previous exposition and its proof is skipped.

\begin{algorithm}[htb]\label{alg:ins}
\SetAlgoLined
\KwData{Partial feasible solution $(A,w)$, candidate $c'\in C\setminus A$, threshold $t\geq 0$.}
Initialize $w'\leftarrow w$\;
\For{each voter $n\in N_c$}{
Set $w'_{nc'} \leftarrow s_n$\;
\For{each member $c\in A\cap C_n$}{
\lIf{$supp_w(c)>t$}{update $w'_{nc} \leftarrow w'_{nc}\cdot\frac{t}{supp_w(c)}$}
Update $w'_{nc'}\leftarrow w'_{nc'} - w'_{nc}$\;
}
}
\Return $(A+c',w')$\;
 \caption{$\ins(A,w,c',t)$}
\end{algorithm}

\begin{lemma}\label{lem:insert}
For a feasible partial solution $(A,w)$, candidate $c'\in C\setminus A$ and threshold $t\geq 0$, 
Algorithm $\ins(A,w,c',t)$ executes in time $O(|E|)$, and returns a feasible solution $(A+c',w')$ 
where $supp_{w'}(c')=prescore_{(A,w)}(c',t)$ and $supp_{w'}(c)\geq \min\{supp_w(c),t\}$ for each member $c\in A$. 
In particular, if $prescore_{(A,w)}(c',t)\geq t$ then $supp_{w'}(A+c')\geq \min\{supp_w(A),t\}$.
\end{lemma}

Whenever partial solution $(A,w)$ is clear from context, we drop the subscript from our definitions of slack and pre-score. 
When we add the new candidate $c'$ to the solution, we want to ensure that inequality $prescore(c',t)\geq t$ holds, as we do not want to increase the number of validators with support below threshold $t$. How high can we make threshold $t$ and still have $prescore(c',t)\geq t$ hold? For each candidate $c'\in C\setminus A$, we define $score_{(A,w)}(c')$ (shortened to $score(c')$) to be the highest value of $t$ such that $prescore(c',t) \geq t$. Our heuristic now becomes apparent.

\begin{heuristic}
Given a partial solution $(A,w)$, find a candidate $c_{\max}\in C\setminus A$ with highest score $t_{\max}=score(c_{\max})=\max_{c'\in C\setminus A} score(c')$, and execute $\ins(A,w,c_{\max},t_{\max})$, so that its output solution $(A+c_{\max},w')$ observes 

$$\forall c\in A, \ supp_{w'}(c)\geq \min\{supp_w(c), t_{\max}\}, \quad \text{ and } \quad supp_{w'}(A+c_{\max})\geq \min \Big\{ supp_w(A), t_{\max}\Big\}.$$
\end{heuristic}

In the remainder of the section we describe efficient algorithms to find the candidate with highest pre-score for a given threshold $t$, and the candidate with highest score. The reader uninterested in these computational details may skip ahead to the next section.

We start with Algorithm~\ref{alg:maxprescore}, which shows how to find the candidate with highest pre-score for a given threshold $t$.

\begin{algorithm}[htb]\label{alg:maxprescore}
\SetAlgoLined
\KwData{Partial solution $(A,w)$, threshold $t\geq 0$.}
\For{each voter $n\in N$}{
Compute $slack(n,t)=s_n-\sum_{c\in A\cap C_n} w_{nc}\cdot \min\{1, t/supp_w(c)\}$\;
}
\For{each candidate $c'\in C\setminus A$}{
Compute $prescore(c',t)=\sum_{n\in N_{c'}} slack(n,t)$\;
}
Find a candidate $c_t\in\arg\max_{c'\in C\setminus A} prescore(c', t)$\;
\Return $(c_t, prescore(c_t, t))$\;
 \caption{$\maxprescore(A,w,t)$}
\end{algorithm}

\begin{lemma}
For a partial solution $(A,w)$ and threshold $t\geq 0$, $\maxprescore(A,w,t)$ executes in time $O(|E|)$, 
and returns a tuple $(c_t,p_t)$ such that $c_t\in C\setminus A$ 
and $p_t=prescore(c_t,t)=\max_{c'\in C\setminus A} prescore(c',t)$.
\end{lemma}

\begin{proof}
The correctness of the algorithm directly follows from the definitions of slack and pre-score. The running time is $O(|E|)$ because each edge in the approval graph $G=(N\cup V, E)$ is inspected at most once in each of the two loops. The first loop also inspects each voter, but we have $|N|=O(|E|)$ since we assume that $G$ has no isolated vertices.
\end{proof}


For a fixed partial solution $(A,w)$ and for each candidate $c'\in C\setminus A$, consider the function 
$f_{c'}(t):=prescore(c',t)-t$ in the interval $[0,\infty)$. 
Notice from the definition of pre-score~\eqref{eq:prescore} that this function is convex, continuous and strictly decreasing with no lower bound, and that $f_{c'}(0)\geq 0$; hence it has a unique root which corresponds precisely to $score(c')$. We could approximate this root via binary search -- however, we can do better. 
Function $f_{c'}(t)$ is piece-wise linear: if we sort the member supports $\{supp_w(c): \ c\in A\}=\{t_1, \cdots, t_r\}$ so that $t_1 < \cdots < t_r$ for some $r\leq |A|$, then $f_{c'}(t)$ is linear in each of the intervals $[0, t_1), [t_1, t_2), \cdots, [t_r, \infty)$.

Similarly, function $f(t):= \max_{c'\in C\setminus A} f_{c'}(t) = \max_{c'\in C\setminus A} prescore(c',t) -t$ is continuous and strictly decreasing in the interval $[0,\infty)$, with a unique root $t_{\max}=\max_{c'\in C\setminus A} score(c')$. Unfortunately, this function is in general not linear within each of the intervals above.%
%
\footnote{In each of these $O(k)$ intervals, function $f(t)$ is piece-wise linear with $O(|C|)$ pieces. We could thus find the root of $f(t)$ via binary search by performing $O(\log k + \log |C|)$ calls to $\maxprescore$. However, our approach only requires $O(\log k)$ such calls.} %
%
Still, it will be convenient to use binary search to identify the interval that contains $t_{\max}$. We do so in Algorithm~\ref{alg:interval}. The next lemma follows from our exposition and its proof is skipped.

\begin{algorithm}[htb]\label{alg:interval}
\SetAlgoLined
\KwData{Partial solution $(A,w)$.}
Sort the member supports to obtain $0=t_0<t_1<\cdots <t_r$, where $\{t_1, \cdots, t_r\}=\{supp_w(c): \ c\in A\}$\;
\lIf{$p_{t_r}\geq t_r$ where $(c_{t_r},p_{t_r})\leftarrow \maxprescore(A,w,t_r)$}{\Return $t_r$}
Let $j_{lo}=0$, $j_{hi}=r-1$\;
\While{$j_{lo}<j_{hi}$}{
  Let $j=\lceil (j_{lo}+j_{hi})/2 \rceil$\;
  \eIf{$p_{t_j}\geq t_j$ where $(c_{t_j},p_{t_j})\leftarrow \maxprescore(A,w,t_j)$}{
  Set $j_{lo}\leftarrow j$\;}{
  Set $j_{hi}\leftarrow j-1$\;}
}
\Return $t_{j_{lo}}$\;

 \caption{$\interval(A,w)$}
\end{algorithm}

\begin{lemma}
For a partial solution $(A,w)$, Algorithm $\interval(A,w)$ makes $O(\log |A|)$ calls to $\maxprescore$, and thus runs in time $O(|E|\cdot \log k)$. It returns a value $t'$ so that $t'\leq t_{\max}:=\max_{c'\in C\setminus A} score(c')$, and for each candidate $c'\in C\setminus A$, the value of $prescore(c',t)$ is linear in $t$ within the interval $[t',t_{\max}]$.
\end{lemma}

Once we have identified such a value $t'$, we exploit the fact that for each $c'\in C\setminus A$, function $f_{c'}(t)$ is linear inside the interval $[t',t_{\max}]$. If we fix a candidate $c'$ and linearize function $f_{c'}(t)$ by extending its linear behavior within $[t', t_{\max}]$ onto the full domain $[0,\infty)$, and we denote its corresponding unique root by $t_{c'}$, then we have 
%
\begin{align*}
    0&= f_{c'}(t_{c'})|_{\text{linearized around } [t', t_{\max}]}\\
    &=prescore(c', t_{c'})|_{\text{linearized around } [t', t_{\max}]} - t_{c'}\\
    &=\sum_{n\in N_{c'}} slack(n,t)|_{\text{linearized around } [t', t_{\max}]} - t_{c'}\\
    &=\sum_{n\in N_{c'}} \Big( s_n - \sum_{c\in A\cap C_n: \ supp_{w}(c)< t'}w_{nc} - \sum_{c\in A\cap C_n: \ supp_w(c)\geq t'} w_{nc}\cdot t_{c'}/supp_w(c) \Big) - t_{c'},
\end{align*}
%
where we used the definitions of pre-score and slack. Solving for $t_{c'}$, we obtain
%
\begin{align*}
    t_{c'}=\frac{\sum_{n\in N_{c'}} \Big( s_n - \sum_{c\in A\cap C_n: \ supp_w(c)< t'} w_{nc} \Big)}%
    {1+\sum_{n\in N_{c'}} \sum_{c\in A\cap C_n: \ supp_w(c)\geq t'} \frac{w_{nc}}{supp_w(c)}}=:linscore_w(c',t').
\end{align*}

Notice that the expression above depends only on $t'$ and not on $t_{\max}$. We refer to this value as \emph{the linearized score of $c'$ around $t'$}, and denote it by $linscore_{(A,w)}(c',t')$ (shortened to $linscore(c',t')$). 
For each candidate $c'\in C\setminus A$ we have that, since $f_{c'}(t)$ is a convex decreasing function over $[0,\infty)$ and its linearization around $t'$ is tangential to it, the root of this linearization must be lower than its own root, i.e. $linscore_w(c',t')\leq score_w(c')$. 
On the other hand, for the candidate $c_{\max}$ with highest score $t_{\max}$, these two roots must coincide, i.e. $t_{\max}=score(c_{\max})=linscore(c_{\max}, t')$. Consequently, $c_{\max}$ also has the highest linearized score around $t'$ among all candidates in $C\setminus A$, and we can exploit this fact to find it. We formalize these observations in Algorithm~\ref{alg:maxscore} and the lemma below.

\begin{algorithm}[htb]\label{alg:maxscore}
\SetAlgoLined
\KwData{Partial solution $(A,w)$.}
Let $t'\leftarrow \interval(A,w)$\;

\For{each voter $n\in N$}{
Compute $p_n:=s_n-\sum_{c\in A\cap C_n: \ supp_w(c)< t'} w_{nc}$\;
Compute $q_n:=\sum_{c\in A\cap C_n: \ supp_w(c)\geq t'} w_{nc}/supp_w(c)$\;
}
\For{each candidate $c'\in C\setminus A$}{
Compute $linscore(c',t')=\frac{\sum_{n\in N_{c'}} p_n}{1+\sum_{n\in N_{c'}} q_n}$\;
}
Find a candidate $c_{\max}\in\arg\max_{c'\in C\setminus A} linscore(c',t')$\;
\Return $(c_{\max}, linscore(c_{\max},t'))$\;
 \caption{$\maxscore(A,w)$}
\end{algorithm}

\begin{lemma}\label{lem:maxscore}
For a partial solution $(A,w)$, Algorithm $\maxscore(A,w)$ runs in time $O(|E|\cdot \log k)$ and returns a tuple $(c_{\max}, t_{\max})$ such that $t_{\max}=score(c_{\max})=\max_{c'\in C\setminus A} score(c')$.
\end{lemma}
\begin{proof}
The correctness of the algorithm follows from the definition of linearized score and the arguments above. Each of the \textbf{for} loops executes in time $O(|E|)$ because in each one of them each edge is examined at most once. Thus, the running time is dominated by the call to function $\interval(A,w)$, which takes time $O(|E|\cdot \log k)$.
\end{proof}


To conclude the section we remark that our full heuristic, which finds a candidate with highest score and adds it to the current partial solution (Algorithm $\maxscore$ followed by $\ins$) executes in time $O(|E|\cdot \log k)$. It thus matches the running time of the heuristic in $\phragmen$ up to a logarithmic term.
We also highlight that the linearized score of a candidate $c'$ around the origin (i.e. setting $t'=0$) is such that %
%
$$\frac{1}{linscore_w(c',0)}=\frac{1+\sum_{n\in N_{c'}} \sum_{c\in A\cap C_n} \frac{w_{nc}}{supp_w(c)}}{\sum_{n\in N_{c'}} s_n},$$
%
which roughly corresponds to the \emph{load} candidate function being minimized in the $\phragmen$ heuristic (see Algorithm~\ref{alg:phragmen}). Therefore, our heuristic can be seen as a generalization which, by considering further linearizations of the pre-score function, grants new candidates higher scores and thus higher supports. 



\subsection{Inserting and rebalancing iteratively}\label{s:315}

We proved in Section~\ref{s:complexity} the existence of a 2-approximation algorithm for maximin support that runs in time $O(\bal\cdot |C|\cdot k)$ or $O(\bal\cdot |C|)$ (Theorems \ref{thm:mms} and \ref{thm:2eps} respectively). 
We use now our heuristic to develop $\phragmms$, a $3.15$-approximation algorithm that runs in time $O(\bal\cdot k)$, and satisfies PJR as well. 
We highlight that this is the fastest known election rule to achieve a constant-factor guarantee for maximin support, and that gains in speed are of paramount importance for our application, where there are hundreds of candidates and a large number of voters.

$\phragmms$ (Algorithm~\ref{alg:balanced}) is an iterative greedy algorithm that starts with an empty committee and alternates between inserting a new candidate with the new heuristic, and fully rebalancing the weight vector, i.e., replacing it with a balanced one. This constitutes a middle ground between the approach in $\phragmen$ where a balanced vector is never computed, and the approach in $\MMS$ where $O(|C|)$ balanced vectors are computed per iteration. 
We formalize the procedure below. Notice that running the Insert procedure (Algorithm~\ref{alg:ins}) before rebalancing is optional, but the step simplifies the analysis and may provide a good starting point to the balancing algorithm.

\begin{algorithm}[htb]
\SetAlgoLined
\KwData{Approval graph $G=(N\cup C, E)$, vector $s$ of vote strengths, committee size $k$.}
Initialize $A=\emptyset$\ and $w=0\in\R^E$\;
\For{$i$ from $1$ to $k$}{
Let $(c_{\max},t_{\max})\leftarrow \maxscore(A,w)$ \; 
\tcp{candidate w.~highest score, and its score} 
Update $(A,w)\leftarrow \ins(A,w,c_{\max},t_{\max})$ \; 
\tcp{or optionally just update $A\leftarrow A+c_{\max}$} 
Replace $w$ with a balanced weight vector for $A$\;
}
\Return $(A,w)$\;
\caption{$\phragmms$}
\label{alg:balanced}
\end{algorithm}

\begin{theorem}\label{thm:315}
$\phragmms$ offers a $3.15$-approximation guarantee for the maximin support problem, satisfies the PJR property, and executes in time $O(\bal\cdot k)$, assuming that $\bal= \Omega(|E|\cdot \log k)$.
\end{theorem}

This in turn proves Theorem~\ref{thm:intro1}. 
The claim on runtime is straightforward: we established in Theorem~\ref{thm:runtimes} that $\maxscore$ runs in time $O(|E|\cdot \log k)$, so each iteration of $\phragmms$ has a runtime of $O(|E|\cdot \log k + \bal)=O(\bal)$, assuming that $\bal= \Omega(|E|\cdot \log k)$. 
In fact, in Appendix~\ref{s:algorithms} we improve upon this analysis and show how each iteration can run in time $O(|E| + \bal)$.
Next, in order to prove the PJR property we need the following technical lemmas.

\begin{lemma}\label{lem:2balanced}
If $(A,w)$ and $(A',w')$ are two balanced partial solutions with $A\subseteq A'$, then $\supp_w(c) \geq \supp_{w'}(c)$ for each $c\in A$, and $\score_{(A,w)}(c')\geq \score_{(A',w')}(c')$ for each $c'\in C\setminus A'$.
\end{lemma}

\begin{lemma}\label{lem:localopt}
If $\supp_w(A)\geq \max_{c'\in C\setminus A} \score(c')$ holds for a full solution $(A,w)$, then $A$ satisfies PJR.
\end{lemma}

Lemma \ref{lem:2balanced} formalizes the intuition that as more candidates are added to a partial solution that is kept balanced, the scores of unelected candidates may only decrease, never increase; its proof is delayed to Appendix~\ref{s:proofs}.
Lemma~\ref{lem:localopt} establishes the key connection that exists between our definition of score -- and by extension our heuristic -- and the PJR property, and its proof is delayed to Section~\ref{s:local}. 
We prove now that the output of $\phragmms$ satisfies the condition in Lemma~\ref{lem:localopt}, and hence satisfies PJR.

\begin{lemma}\label{lem:315localoptimality}
At the end of each one of the $k$ iterations of Algorithm $\phragmms$, if $(A,w)$ is the current partial balanced solution, we have that $\supp_{w}(A)\geq \max_{c'\in C\setminus A} \score_{(A,w)}(c')$.
\end{lemma}
\begin{proof}
Let $(A_i,w_i)$ be the partial solution at the end of the $i$-th iteration. We prove the claim by induction on $i$, with the base case $i=0$ being trivial as we use the convention that $\supp_{w_0}(\emptyset)=\infty$ for any $w_0$. For $i\geq 1$, suppose that on iteration $i$ we insert a candidate $c_i$ with highest score, and let $w'$ be the vector that is output by $\ins(A_{i-1}, w_{i-1}, c_i, \score_{(A_{i-1}, w_{i-1})}(c_i))$ (Algorithm~\ref{alg:ins}). Then

\begin{align*}
\supp_{w_i}(A_i) &\geq \supp_{w'}(A_i) \\
&\geq \min\{ \supp_{w_{i-1}}(A_{i-1}), \score_{(A_{i-1}, w_{i-1})}(c_i) \} \\
&\geq \max_{c'\in C\setminus A_{i-1}} \score_{(A_{i-1}, w_{i-1})}(c') \\
&\geq \max_{c'\in C\setminus A_{i}} \score_{(A_{i}, w_{i})}(c'), 
\end{align*}
%
where the first inequality holds as $w_i$ is balanced for $A_i$, the second one is a property of our heuristic, the third one holds by induction hypothesis and the choice of candidate $c_i$, and the last one follows from Lemma~\ref{lem:2balanced}. 
This completes the proof.
\end{proof}


It remains to prove the claimed approximation guarantee for $\phragmms$. 
To do that, we use the following key technical result, whose proof is based on the flow decomposition theorem and is delayed to Apendix~\ref{s:flow}. 
This result says that if a partial solution is balanced, then not only are there unelected candidates that can be appended with high support, but they also have large scores, so we can find them efficiently with our heuristic. 
More specifically, there must be a subset of voters with large aggregate vote strength who currently have too few representatives, so these representatives all have large supports, and in turn the voters have large slack. 

\begin{lemma}\label{lem:N_a}
If $(A^*, w^*)$ is an optimal solution to the maximin support instance with $t^*=\supp_{w^*}(A^*)$, and $(A,w)$ is a balanced solution with $|A|\leq k$ and $A\neq A^*$, then for each $0\leq a\leq 1$ there is a subset $N(a)\subseteq N$ of voters such that 
\begin{enumerate}
	\item each voter $n\in N(a)$ has a neighbor in $A^*\setminus A$;
	\item for each voter $n\in N(a)$, we have $\supp_w(A\cap C_n)\geq at^*$;
	\item $\sum_{n\in N(a)} s_n \geq |A^* \setminus A|\cdot (1-a) t^*$; and
	\item for any $b$ with $a\leq b\leq 1$ we have that $N(b)\subseteq N(a)$, and for each $n\in N(a)$ we have that $n$ is also in $N(b)$ if and only if property 2 above holds for $n$ with parameter $a$ replaced by $b$.
\end{enumerate}
\end{lemma}

As a warm-up, we show how this last result easily implies a $4$-approximation guarantee for $\phragmms$.

\begin{lemma}
If $(A,w)$, $(A^*,w^*)$ and $t^*$ are as in Lemma~\ref{lem:N_a}, there is a candidate $c'\in A^*\setminus A$ with $\score(c')\geq t^*/4$. Hence, $\phragmms$ provides a $4$-approximation for the maximin support problem.
\end{lemma}

\begin{proof}
We apply Lemma~\ref{lem:N_a} with $a=1/2$. In what follows we refer to the four properties stated in that lemma. We have that

\begin{align*}
    \sum_{c'\in A^*\setminus A} \prescore(c',t^*/4) &=\sum_{c'\in A^*\setminus A} \sum_{n\in N_{c'}} \slack(n,t^*/4) \\
		&\geq \sum_{n\in N(a)} \slack(n,t^*/4) \\
    &\geq \sum_{n\in N(a)} \Big[ s_n - \frac{t^*}{4}\sum_{c\in A\cap C_n} \frac{w_{nc}}{\supp_w(c)} \Big] \\
    &\geq \sum_{n\in N(a)} \Big[ s_n - \frac{1}{2}\sum_{c\in A\cap C_n} w_{nc} \Big] \\
    &\geq \frac{1}{2}\sum_{n\in N(a)} s_n \\
		&\geq \frac{1}{2} (|A^*\setminus A|\cdot t^*/2) = |A^*\setminus A|\cdot t^*/4, 
\end{align*}
%
where the five inequalities hold respectively by property 1 (which implies $N(a)\subseteq \cup_{c'\in A^*\setminus A} N_{c'}$), by definition of slack (equation~\ref{eq:slack}), by property 2, by feasibility (inequality~\ref{eq:feasible}), and by property 3.
Therefore, by an averaging argument, there must be a candidate $c'\in A^*\setminus A$ with $\prescore(c',t^*/4)\geq t^*/4$, which in turn implies that $\score(c')\geq t^*/4$ by definition of score. 
The $4$-approximation guarantee for Algorithm $\phragmms$ easily follows by induction on the $k$ iterations, using Lemma~\ref{lem:insert} and the fact that rebalancing a partial solution never decreases its least member support.
\end{proof}

To get a better approximation guarantee for the $\phragmms$ rule and finish the proof of Theorem~\ref{thm:315}, we apply Lemma~\ref{lem:N_a} with a more carefully selected parameter $a$, and use the following technical result whose proof is delayed to Apppendix~\ref{s:proofs}.

\begin{lemma}\label{lem:Lebesgue}
Consider a strictly increasing and differentiable function $f:\mathbb{R}\rightarrow \mathbb{R}$, with a unique root $\chi$. For a finite sum $\sum_{i\in I} \alpha_i f(x_i)$ where $\alpha_i\in\mathbb{R}$ and $ x_i\geq \chi$ for each $i\in I$, we have that
$$\sum_{i\in I} \alpha_i f(x_i) = \int_{\chi}^{\infty} f'(x) \big(\sum_{i\in I: \ x_i\geq x} \alpha_i\big)dx.$$
\end{lemma}

\begin{lemma}\label{lem:candidate315}
If $(A,w)$, $(A^*,w^*)$ and $t^*$ are as in Lemma~\ref{lem:N_a}, there is a candidate $c'\in A^*\setminus A$ with $\score(c')\geq t^*/3.15$. Hence, $\phragmms$ provides a $3.15$-approximation for the maximin support problem.
\end{lemma}

\begin{proof}
We refer to Lemma~\ref{lem:N_a} and its properties, with a parameter $0\leq a\leq 1$ to be defined later. We have
\begin{align*}
    \sum_{c'\in A^*\setminus A} & \prescore(c',at^*) \\ 
		&= \sum_{c'\in A^*\setminus A} \ \sum_{n\in N_c} \slack(n, at^*) \\
		&\geq \sum_{n\in N(a)} \slack(n, at^*) \\
    &\geq \sum_{n\in N(a)} \Big[ s_n - at^* \sum_{c\in A\cap C_n} \frac{w_{nc}}{\supp_w(c)} \Big] \\
    &\geq \sum_{n\in N(a)} \Big[ s_n - \frac{at^*}{\supp_w(A\cap C_n)} \sum_{c\in A\cap C_n} w_{nc} \Big] \\
    &\geq \sum_{n\in N(a)} s_n\Big[ 1- \frac{at^*}{\supp_w(A\cap C_n)} \Big], 
\end{align*}
%
where the four inequalities hold respectively by property 1, equation~\ref{eq:slack}, property 2 and inequality~\ref{eq:feasible}, and where $\supp_w(\emptyset)=\infty$ by convention. 
At this point, we apply Lemma~\ref{lem:Lebesgue} over function $f(x):=1-a/x$, which has the unique root $\chi=a$, and index set $I=N(a)$ with $\alpha_n=s_n$ and $x_n=\supp_w(A\cap C_n)/t^*$. We obtain
\begin{align*}
    \sum_{c'\in A^*\setminus A} & \prescore(c',at^*) \\
		&\geq \int_{a}^{\infty} f'(x) \Big( \sum_{n\in N(a): \ \supp_w(A\cap C_n)\geq xt^*} s_n \Big)dx\\
    &=\int_{a}^{\infty} \frac{a}{x^2}\Big( \sum_{n\in N(x)} s_n \Big)dx \\
    &\geq \int_{a}^1 \frac{a}{x^2} \Big( |A^*\setminus A|\cdot (1-x)t^* \Big)dx \\
    & = |A^*\setminus A|\cdot at^* \int_{a}^1 \Big( \frac{1}{x^2} - \frac{1}{x} \Big)dx \\
		&= |A^*\setminus A|\cdot at^*\Big(\frac{1}{a} - 1 + \ln  a\Big),
\end{align*}
%
where we exploited properties 4 and 3. 
If we now set $a=1/3.15$, we have that $1/a - 1 + \ln a\geq 1$, so by an averaging argument there is a candidate $c'\in A^*\setminus A$ for which $\prescore(c',at^*)\geq at^*$, and hence $\score(c')\geq at^*$. The approximation guarantee for the $\phragmms$ rule follows by induction on the $k$ iterations, as before.
\end{proof}


\section{Verifying the solution}\label{s:local}

We start the section with a key property of algorithm $\phragmms$ as motivation.

\begin{theorem}\label{thm:315guarantee}
If a balanced solution $(A,w)$ observes $\supp_w(A)\geq \max_{c'\in C\setminus A} \score(c')$, then it satisfies a) the PJR property and b) a 3.15-approximation guarantee for the maximin support objective. 
Furthermore, testing all conditions (feasibility, balancedness and the previous inequality) can be done in time $O(|E|)$. 
Finally, the output solution of $\phragmms$ is guaranteed to satisfy these conditions.
\end{theorem}

\begin{proof}
The first statement follows from Lemmas \ref{lem:localopt} and \ref{lem:candidate315}, and the third one from Lemma~\ref{lem:315localoptimality}. 
Feasibility (inequality~\ref{eq:feasible}) can clearly be checked in time $O(|E|)$, as can balancedness by Lemma~\ref{lem:balanced}. 
Finally, if $t:=\supp_w(A)$, the inequality in the statement is equivalent to $t\geq \max_{c'\in C\setminus A} \prescore(c',t)$, which is tested with algorithm $\maxprescore(A,w,t)$ in time $O(|E|)$ by Theorem~\ref{thm:runtimes}.
\end{proof}

%This in turn proves Theorem~\ref{thm:intro2}. 
As we argued in the introduction, the result above is one of the most relevant features of our proposed election rule, and is essential for its implementation over a blockchain network as it enables its adaptation into a verifiable computing scheme. As such, the rule may be executed by off-chain workers, leaving only the linear-time tests mentioned in the previous theorem to be performed on-chain, to ensure the quality of the solution found. 
   
To finish the proof of Theorem~\ref{thm:intro2}, it remains to prove Lemma~\ref{lem:localopt} -- which we do at the end of this section -- and show that the verification process above admits a parallel execution -- which we do next. 
In particular, for a parameter $p$, we consider the distribution of this process over $p$ computing units that execute in sequence, such as $p$ consecutive blocks in a blockchain network. 
We remark however that our description below may be easily adapted to concurrent execution if desired.

\begin{lemma}
For any integer $p\geq 1$, both the input election instance as well as any solution $(A,w)$ to it can be distributed into $p$ data sets, such that each data set is of size $O(|E|/p + |C|)$. 
Moreover, all the tests mentioned in Theorem~\ref{thm:315guarantee} can be executed by $p$ sequential computing units such that each unit only requires access to one data set and runs in time $O(|E|/p + |C|)$. 
Therefore, for $p$ sufficiently large, each unit can be made to run in time $O(|C|)$.
\end{lemma}

\begin{proof}
Partition the voter set $N=\cup_{i=1}^p N^i$ into $p$ subsets of roughly equal size, and let $G^i$ be the subgraph of the input approval graph $G$ induced by $N^i\cup C$ (corresponding to the ballots of voters in $N^i$). 
Consider $p$ data sets where the $i$-th one stores subgraph $G^i$ along with the list of vote strengths for voters in $N^i$. 
%
Next, we assume that an untrusted party provides a solution $(A,w)$, and we assume that they also provide its corresponding vector $(\supp'_w(c))_{c\in A}$ of member supports, where the prime symbol indicates that these are claimed values to be verified. 
This solution is distributed so that the $i$-th data set stores the full committee $A$, the claimed supports, and the restriction $w^i$ of the edge weight vector $w$ over $G^i$. 
Clearly, each data set is of size $O(|E|/p + |C|)$.

Now consider $p$ computing units running in sequence, where the $i$-th unit has access to the $i$-th data set, and recall that the verification of solution $(A,w)$ consists of four tests: a) feasibility, b) balancedness, c) correctness of the claimed member supports, and d) the inequality $t\geq \max_{c'\in C\setminus A} \prescore(c',t)$, where we define $t:=\supp'_w(A)$. 
To avoid dependencies across these tests, our general strategy is to assume that the claimed supports are correct, except obviously for test c. 
For example, since both feasibility (inequality~\ref{eq:feasible}) and balancedness (properties 2 and 3 of Lemma~\ref{lem:balanced}) are checked on a per-voter basis, the $i$-th unit can perform these checks for its own subset of voters $N^i$, using the claimed supports to check property 3 of Lemma~\ref{lem:balanced}. 

Tests c and d, on the other hand, require the cooperation of all units. 
The $i$-th unit can compute a vector $(\supp_{w^i}(c))_{c\in A}$ of supports relative to the local voters in $N^i$, so it follows by induction that it can also compute the partial sum $(\supp_{\sum_{j\leq i}w^j}(c))_{c\in A}$, and communicate it to the $(i+1)$-st unit. The last unit can then compute the full vector $(\supp_{\sum_{j\leq p}w^j}(c))_{c\in A}=(\supp_{w}(c))_{c\in A}$, and check that it matches the claimed supports. 

Similarly, the $i$-th unit can compute the vector $(\slack(n,t))_{n\in N^i}$ of slacks for $N_i$ (equation~\ref{eq:slack}), and use it to find a vector of parameterized scores relative to the local voters, $(\prescore^i(c',t))_{c'\in C\setminus A}$, where $\prescore^i(c', t):=\sum_{n\in N_{c'}\cap N^i} \slack(n,t)$. 
Again by induction, this unit can also find the partial sum $(\sum_{j\leq i}\prescore^j(c',t))_{c'\in C\setminus A}$, and communicate it to the $(i+1)$-st unit. 
The last unit then retrieves the full vector $(\sum_{j\leq p}\prescore^j(c',t))_{c'\in C\setminus A}=(\prescore(c',t))_{c'\in C\setminus A}$, and verifies that all parameterized scores are bounded by $t$. 
Clearly, each unit has a time and memory complexity of $O(|E|/p + |C|)$.
\end{proof}

Next, we begin our analysis of the PJR property by defining a \emph{parametric version} of it, that measures just how well represented the voters are by a given committee $A$. It is a generalization of the property that turns it from binary to quantitative.

\begin{definition}
For any $t\in\R$, a committee $A\subseteq C$ (of any size) satisfies PJR with parameter $t$ ($t$-PJR for short) if, for any group $N'\subseteq N$ of voters and any integer $0<r\leq |A|$, we have that
\begin{itemize}
\item[a)] if $|\cap_{n\in N'} C_n|\geq r$
\item[b)] and $\sum_{n\in N'} s_n \geq r\cdot t$, 
\item[c)] then $|A\cap (\cup_{n\in N'} C_n)|\geq r$.
\end{itemize}
\end{definition}

In words, if there is a group $N'$ of voters with at least $r$ commonly approved candidates, and enough aggregate vote strength to provide each of these candidates with a support of at least $t$, then this group must be represented by at least $r$ members in committee $A$, though not necessarily commonly approved. 
Notice that the standard version of PJR is equivalent to $\hat{t}$-PJR for $\hat{t}:=\sum_{n\in N} s_n / |A|$, and that if a committee satisfies $t$-PJR then it also satisfies $t'$-PJR for each $t'\geq t$, i.e., the property gets stronger as $t$ decreases. 
This is in contrast to the maximin support objective, which implies a stronger property as it increases.

We remark that the notion of \emph{average satisfaction} introduced in~\cite{sanchez2017proportional} also attempts to quantify the level of proportional representation achieved by a committee. 
Informally speaking, that notion measures the average number of representatives in the committee that each voter in a group has, for any group of voters with sufficiently high aggregate vote strength and cohesiveness. 
In contrast, with parametric PJR we focus on providing sufficient representatives to the group as a whole and not to each individual voter, and we measure the aggregate vote strength required to gain adequate representation.
Interestingly, the average satisfaction measure is closely linked to the EJR property, and in particular in~\cite{aziz2018complexity} this measure is used to prove that a local search algorithm achieves EJR; 
similarly, in Appendix~\ref{s:LS} we use parametric PJR to prove that a local search version of $\phragmms$ achieves standard PJR.

Testing whether an arbitrary solution satisfies standard PJR is known to be coNP-complete~\cite{aziz2018complexity}, hence the same remains true for its parametric version.
We provide next a sufficient condition for a committee to satisfy $t$-PJR which is efficiently testable, based on our definitions of parameterized score and score.  

\begin{lemma} \label{lem:locality}
If for a feasible solution $(A,w)$ there is a parameter $t\in\R$ such that $\max_{c'\in C\setminus A} \prescore(c',t)<t$, or equivalently, such that $\max_{c'\in C\setminus A} \score(c') <t$, then committee $A$ satisfies $t$-PJR. 
This condition can be tested in $O(|E|)$ time.
\end{lemma}

\begin{proof} 
We prove the contrapositive of the claim. If $A$ does not satisfy $t$-PJR, there must be a subset $N'\subseteq N$ of voters and an integer $r>0$ that observe properties a) and b) above but fail property c). 
By property a) and the negation of c), set $(\cap_{n\in N'} C_n)\setminus A$ must be non-empty: let $c'$ be a candidate in it. 
We will prove that for any feasible weight vector $w\in \R^E$, it holds that $\prescore(c',t)\geq t$, and consequently $\score(c')\geq t$ by the definition of score. We have
%
\begin{align*} 
\prescore(c',t) &= \sum_{n\in N_{c'}}  \slack(n,t) \\
&\geq \sum_{n\in N'} \slack(n,t) \\
&\geq \sum_{n\in N'} \Big(s_n - t\cdot \sum_{c\in A\cap C_n} \frac{w_{nc}}{\supp_w(c)}\Big)  \\
&= \sum_{n\in N'} s_n - t \cdot \sum_{c\in A\cap (\cup_{n\in N'} C_n)} 
\frac{\sum_{n\in N'\cap N_c} w_{nc}}{\sum_{n\in N_c} w_{nc}} \\ 
& \geq t\cdot r - t\cdot |A\cap (\cup_{n\in N'} C_n)| \\
& \geq t\cdot r - t\cdot (r-1) = t, 
\end{align*}
%
where the first inequality holds as $N'\subseteq N_{c'}$ by our choice of candidate $c'$, the second one holds by definition of slack, the third one holds by property b) and because the fraction on the fourth line is at most $1$ for each candidate $c$, and the last inequality holds by negation of c). 
This proves that $\prescore(c',t) \geq t$. 

For a given solution $(A,w)$ and parameter $t$, one can verify the condition above in time $O(|E|)$ by computing $\maxprescore(A,w,t)$ and comparing the output to $t$; see Theorem~\ref{thm:runtimes}.
\end{proof}

The proof of Lemma~\ref{lem:localopt} now follows as a corollary.

\begin{proof}[Proof of Lemma~\ref{lem:localopt}]
Let $c_{\max}$ be a candidate with highest score $t_{\max}=\score(c_{\max})=\max_{c'\in C\setminus A} \score(c')$. 
If $\supp_w(A)\geq t_{\max}$, it follows from Lemma~\ref{lem:insert} that if we execute $\ins(A,w,c_{\max}, t_{\max})$, we obtain a solution $(A+c_{\max}, w')$ with $\supp_{w'}(A+c_{\max})=t_{\max}$. 
Now, by feasibility of vector $w'$, we have the inequality $\sum_{n\in N} s_n \geq \sum_{c\in A+c_{\max}} \supp_{w'}(c) \geq (k+1)\cdot t_{\max}$, which implies that $t_{\max}\leq \sum_{n\in N} s_n / (k+1) < \sum_{n\in N} s_n / k =: \hat{t}$. 
By Lemma~\ref{lem:locality} above, having $t_{\max} < \hat{t}$ implies that $A$ satisfies $\hat{t}$-PJR, which is standard PJR.
\end{proof}

We end the section with the observation that the inequality in Lemma~\ref{lem:localopt}, namely $\supp_w(A)\geq \max_{c'\in C\setminus A} \score(c')$, corresponds to a notion of local optimality for solution $(A,w)$. 
Indeed, if the inequality did not hold we could improve upon the solution by iteratively swapping the member with least support for the unelected candidate with highest score, resulting in an increase of the least member support and/or a decrease of the highest score among unelected candidates, which by Lemma~\ref{lem:locality} strengthens the level of parametric PJR of the solution. 
Therefore, the fact that $\phragmms$ always returns a locally optimal solution (Lemma~\ref{lem:315localoptimality}) implies standard PJR but can be considered to be a strictly stronger property. 
In Appendix~\ref{s:LS}, we formalize this local search algorithm and use it to prove Theorem~\ref{thm:intro3}.


\section{A validator selection protocol}\label{s:implement}

Recall that our motivating application is the selection of validators for a blockchain network that implements Nominated Proof-of-Stake (NPoS). 
In this section we sketch a proposal for such a protocol, and use the Polkadot network as a specific example.
As of late May 2021, Polkadot selects a committee of $k\approx 300$ validators to participate in its consensus protocol, out of a set of $|C|\approx 900$ candidates, and has $|N|\approx 20000$ nominators. 
It also bounds the number of candidates that each nominator may approve of to a constant $c=16$; 
this way, the size of the election instance is bounded as $O(|E|)=O(c\cdot |N|)=O(|N|)$, i.e., it stays linear in the number of voters.%
%
\footnote{We also highlight the importance of allowing nominators to vote for multiple candidates. 
Otherwise, they face the dilemma of having to choose between a popular candidate and a candidate that may represent them better but has a lower chance of being elected. Facing such dilemma, a rational nominator seeking to maximize staking rewards will prefer to vote for the popular candidate. This type of tactical voting will result in popular candidates being overrepresented, to the detriment of the network's security and decentralization goals.} 
%
Toward the end of each era --lasting roughly one day-- the protocol takes a snapshot of the current nominators' preferences and stakes and uses it as input to run a committee election rule, to select a validator committee $A$ for the following era. 
We remark that it is advantageous for the protocol to have the election rule output a stake distribution vector $w\in \R^E$ along with committee $A$: not only will this vector $w$ play a role in the verification of the committee guarantees, but it also helps distribute the era staking rewards earned by each staking pool back to the nominators.

We propose running the $\phragmms$ rule as a verifiable computing scheme. 
First, the system enters an \emph{election phase} in which the current validators --or a subset of them-- may act themselves as off-chain workers, and run $\phragmms$ as a background task logically separate from consensus, and with a relaxed time frame of (say) up to an hour. 
Then, the system enters a \emph{submission phase}, also lasting up to an hour, in which a validator or any other user may submit a prospective solution $(A,w)$ on-chain as a transaction. 

For each prospective solution received within the submission phase, the protocol executes on-chain the verification test described in Theorem~\ref{thm:315guarantee}, and declares as winner the solution that passes the test and has the highest maximin support objective. 
By allowing anyone to submit a solution, we keep the protocol decentralized and able to profit from higher quality solutions that may be found by users running other election rules off-chain.
Yet, in order to minimize spam we suggest to require that submitted solutions be accompanied by some collateral susceptible to being lost in case the solution fails the verification test.
This way, we can expect to receive only a handful of submissions per era. 
Furthermore, we suggest that the verification process of each solution be distributed over several consecutive blocks, following the parallelization presented in Lemma~\ref{lem:parallel}. 
For example, by distributing this computation over $p=20$ blocks per solution, each block only needs to process up to $|N|/20\approx 1000$ nominators and $|C|<1000$ candidates. 
As Polkadot produces roughly ten blocks per minute, this means that a solution would be verified within two minutes. 

We claim that being able to verify the strong guarantees on security and proportionality offered by $\phragmms$ (as opposed to, e.g., simply selecting the feasible prospective solution with highest maximin support objective) protects the network against a possible long-range attack, as we explain now. 
Consider a scenario where an adversary currently controlling a minority of validators creates a private fork (i.e., an alternative valid chain) right before the start of the submission phase, and in this fork it censors all prospective solutions except for one, fabricated by the adversary itself, in which it becomes grossly overrepresented and possibly even gains control of a majority of validators. 
Then, after the end of the submission phase it publishes this fork and attempts to make it canonical (for instance by making it longer than all other forks, if the consensus protocol follows the longest-chain rule). 
If this attack is successful, the adversary will have captured the network in the next era. 
To resist such an attack, we insist in discarding all submitted solutions that do not pass the verification test, and we propose that the submission phase may extend indefinitely until at least one solution passing the test has been submitted. 
%It is worth mentioning that Polkadot is also protected against long-range attacks by its use of a finality gadget~\cite{stewart2020grandpa}, which limits the possibility of long chain reorganizations.


%We just check feasibility.
%Check PJR test offchain.
%Proof of not PJR (PJR challenge): signal one candidate with high score, also give its supporters (could have hundreds). Computing slacks only requires output graph.
%Easy to prove that solution is not PJR, and polytime to find counterexample. 

%Right now we're running Phragmen, with random number of star balancing rounds. This is what validators should run.
%Solution is checked off-chain, solution is checked on chain for feasibility.
%Validators fight over submitting better solutions, they need to be epsilon better and then they overwrite each other's solution.

%Eventually the idea is to run Phragmms, but only once we have solutions from community + bot from W3F.
%Chain doesn't care what algorithm is run off chain.

%Idea: we need absolute notion of quality of committee
%What if there's governance-controlled least support? Solution is rejected if this check doesn't pass.
%That would be a good guarantee.
%What if: lots of stake needs to be used.
%Much safer! At least as temporary patch.

\section{Conclusions and open questions}\label{s:conc}

There is a recent surge of proof-of-stake (PoS) based blockchain projects that take a "representative democracy" approach, thereby letting token holders vote for the validator candidates that they trust, and automatically selecting a committee with $k$ of the most trusted candidates. However, the design choices behind these \emph{validator selection protocols} are varied, and formal justifications are scarce. 
%
We present the first social choice analysis for the electoral system of a validator selection protocol, namely for Nominated Proof-of-Stake (NPoS), used by Polkadot and Kusama. 
%
Starting from first principles and in the pursuit of security and decentralization, we formalize the problem in terms of proportional justified representation (PJR) and the maximin support objective, two criteria recently introduced in the literature of proportional representation.
With the problem definition at hand, we show that current election rules either perform poorly in terms of security, or are too slow to be compatible with the blockchain architecture. We then propose $\phragmms$, the first rule to provide formal guarantees on both criteria and be implementable on a blockchain network. 

We propose the adaptation of committee election rules as verifiable computing schemes. Indeed, this adaptation proves to be key for the implementation of the $\phragmms$ rule in the Polkadot network. 
We remark that our proposed verification process (Theorem~\ref{thm:315guarantee}) is linear in the size of the input, and leave it as an open challenge to find an election rule that achieves a constant-factor approximation guarantee for maximin support verifiably, with a verification runtime sublinear in the number of voters.

We present the first approximability analysis for a Phragm\'{e}n objective: we show that the maximin support objective can be approximated within a factor of 2, but not within a factor of $1.2-\eps$ for any $\eps>0$ unless P=NP. This gap between approximability and hardness awaits to be closed in future work. 
Similarly, it remains open to establish whether the approximation factors we proved for $\MMS$ and $\phragmms$ ($2$ and $3.15$ respectively) are tight.

We highlight that our approximation analyses are based on network flow theory, a promising tool that is not widely used in the literature of committee election rules. 
Similarly, we leverage Phragm\'{e}n's notion of load balancing, formalize what it means for a vote distribution over the edges of the approval graph to be balanced, and show how to find a balanced distribution efficiently using new results for parametric flow. We then synthesize the heuristics behind $\phragmen$, $\MMS$ and $\phragmms$ in terms of how well partial solutions are rebalanced in between iterations. 
Further election rules might be analyzed in the future using network flow theory and our definition of balancedness. 



\bibliographystyle{ACM-Reference-Format}
\bibliography{references}

\appendix

\section{Exploiting local optimality}\label{s:local}

We start the section with an key property of algorithm $\balanced$ as motivation.

\begin{theorem}\label{thm:315guarantee}
If $(A,w)$ is a balanced solution such that $supp_w(A)\geq \max_{c'\in C\setminus A} score(c')$, then it satisfies PJR and guarantees a 3.15-factor approximation for the maximin support problem. 
Furthermore, testing the required conditions (feasibility, balancedness and the previous inequality) can be done in time $O(|E|)$. 
Finally, the output solution of the $\balanced$ algorithm is guaranteed to satisfy these conditions.
\end{theorem}
\begin{proof}
The first statement follows from Lemmas \ref{lem:localopt} and \ref{lem:candidate315}, and the third one from Lemma~\ref{lem:315localoptimality}. 
Feasibility (inequality~\ref{eq:feasible}) can clearly be checked in time $O(|E|)$, as can balancedness by Lemma~\ref{lem:balanced}, because properties 2 and 3 in that lemma can both be tested in this time. 
Finally, if $t:=supp_w(A)$, the inequality in the statement is equivalent to $t\geq \max_{c'\in C\setminus A} prescore(c',t)$, which is tested with algorithm $\maxprescore(A,w,t)$ in time $O(|E|)$ by Theorem~\ref{thm:runtimes}.
\end{proof}

This in turn proves Theorem~\ref{thm:intro2}. 
A result such as the one above is essential in a scenario where a computationally limited user (the \emph{verifier}) offloads a heavy task -- in this case an election algorithm -- to one or more external entities with more computational power (the \emph{prover}). Yet, as the task is executed privately and the user does not trust the entities, the user must be in condition to provably and efficiently check the quality of the output.
%
In the case of an election protocol over a decentralized blockchain, it would be very costly to run the full election algorithm as an \emph{on-chain} process, meaning that validators must execute it simultaneously and the chain cannot progress until they have finished the execution and agreed on the output. 
Instead, a much more scalable solution is to execute the protocol \emph{off-chain}, meaning that one or more parties run it privately and separate from block production, and then submit the output which is approved if it passes the verification test on-chain. 
The fact that algorithm $\balanced$ can have both of its guarantees efficiently verified on it outputs is in our opinion its most relevant feature. 

The inequality $supp_w(A)\geq \max_{c'\in C\setminus A} score(c')$ mentioned in Lemma~\ref{lem:localopt} and Theorem~\ref{thm:315guarantee} corresponds to a notion of local optimality, for a local search variant of the $\phragmms$ heuristic. 
In Section~\ref{s:tPJR} we explore the relation between this notion and the PJR property. 
Then, in Section~\ref{s:LS} we use the corresponding local search algorithm to devise a post-computation which takes an arbitrary solution $(A,w)$ as input, and returns an output $(A',w')$ which observes $supp_{w'}(A')\geq supp_w(A)$ and provably satisfies PJR.

\subsection{A parametric version of proportional justified representation}\label{s:tPJR}

We define next a parametric version of the PJR property that measures just how well represented the voters are by a given committee $A$. It is a generalization of the property which turns it from binary to quantitative.

\begin{definition}
For any $t\in\R$, a committee $A\subseteq C$ (of any size) satisfies Proportional Justified Representation with parameter $t$ ($t$-PJR for short) if there is no group $N'\subseteq N$ of voters and integer $0<r\leq |A|$ such that:
\begin{itemize}
\item[a)] $\sum_{n\in N'} s_n \geq r\cdot t$,
\item[b)] $|\cap_{n\in N'} C_n|\geq r$, and
\item[c)] $|A\cap (\cup_{n\in N'} C_n)|<r$.
\end{itemize}
\end{definition}

In words, if there is a group $N'$ of voters with at least $r$ commonly approved candidates, who can afford to provide these candidates with a support of value $t$ each, then this group is represented by at least $r$ members in committee $A$, though not necessarily commonly approved. Notice that the standard version of PJR is equivalent to $\hat{t}$-PJR for $\hat{t}:=\sum_{n\in N} s_n / |A|$ (see Section~\ref{s:prel}), and that if a committee satisfies $t$-PJR then it also satisfies $t'$-PJR for each $t'\geq t$, i.e.~the property gets stronger as $t$ decreases. 
Notice as well that this is in contrast to the maximin support objective, which implies a stronger property as it increases; we will exploit the opposite natures of these two parameters of quality in Section~\ref{s:LS} to define a procedure that improves upon a given solution.

%For instance, in the example described in Figure~\ref{fig:phragmen}, the optimal committee satisfies $(1/H_k)$-PJR, while the output of $\phragmen$ does not even satisfy $1$-PJR. However, since in this case we have that $\hat{t}=\frac{k+(1/H_k)}{k}=1+\frac{1}{k\cdot H_k}>1$ and the output does satisfy $\hat{t}$-PJR, it narrowly satisfies PJR. This example shows how much more expressive our parameterized definition of PJR is to assess the quality of a committee. 

We remark that in~\cite{sanchez2017proportional} the related notion of \emph{average satisfaction} is introduced, also to quantify the level of proportional representation achieved by a committee. Roughly speaking, that notion measures the average number of representatives in the committee that each voter in a group has, for any group of voters with sufficiently high vote strength and cohesiveness level. 
In contrast, with parametric PJR we focus on providing sufficient representatives to the group as a whole and not to each individual voter, and we measure the required vote strength for a group to achieve it.
Interestingly, the average satisfaction measure is closely linked to the property of extended justified representation (EJR)~\cite{aziz2017justified}, and in~\cite{aziz2018complexity} that measure is used to prove that a local search algorithm achieves EJR. 
Similarly, we use parametric PJR to prove that a local search algorithm achieves standard PJR.

Testing whether an arbitrary solution satisfies standard PJR is known to be coNP-complete~\cite{aziz2018complexity}, hence the same remains true for its parametric version.
We provide next a sufficient condition for a committee to satisfy $t$-PJR, which is efficiently testable, based on our definitions of pre-score and score.  

\begin{lemma} \label{lem:locality}
If for a feasible solution $(A,w)$ there is a parameter $t\in\R$ such that $\max_{c'\in C\setminus A} prescore(c',t)<t$, or equivalently, such that $\max_{c'\in C\setminus A} score(c') <t$, then committee $A$ satisfies $t$-PJR. 
\end{lemma}

\begin{proof} 
We prove the contrapositive of the claim. If $A$ does not satisfy $t$-PJR, there must be a subset $N'\subseteq N$ of voters and an integer $r>0$ that observe points a) through c) in the definition above. By points b) and c), set $(\cap_{n\in N'} C_n)\setminus A$ must be non-empty: let $c'$ be a candidate in it. 
We will prove that for any feasible weight vector $w\in \R^E$, it holds that $prescore(c',t)\geq t$, and consequently $score(c')\geq t$ by the definition of score. We have
%
\begin{align*} 
prescore(c',t) &= \sum_{n\in N_{C'}}  slack(n,t) \geq \sum_{n\in N'} slack(n,t) & (\text{as } N'\subseteq N_{c'}) \\
&\geq \sum_{n\in N'} \Big(s_n - t\cdot \sum_{c\in A\cap C_n} \frac{w_{nc}}{supp_w(c)}\Big)  
& (\text{by definition \ref{eq:slack}})\\
&= \sum_{n\in N'} s_n - t \cdot \sum_{c\in A\cap (\cup_{n\in N'} C_n)} 
\frac{\sum_{n\in N'\cap N_c} w_{nc}}{\sum_{n\in N_c} w_{nc}} 
& (\text{where fraction is } \leq 1)\\ 
& \geq t\cdot r - t\cdot |A\cap (\cup_{n\in N'} C_n)| & (\text{by a)})\\
& \geq t\cdot r - t\cdot (r-1) = t & (\text{by c)}). 
\end{align*}
%
This proves that $prescore(c',t) \geq t$, which is what we needed to show.
\end{proof}

For a given solution $(A,w)$ and parameter $t$, one can verify the condition above in time $O(|E|)$ by checking whether $\maxprescore(A,w,t)<t$; see Theorem~\ref{thm:runtimes}. Alternatively, from just solution $(A,w)$ one can execute $\maxscore(A,w)$ in time $O(|E|\cdot \log k)$ to obtain the highest score $t_{\max}$ and ascertain that $A$ satisfies $t$-PJR for each $t>t_{\max}$. 
The proof of Lemma~\ref{lem:localopt} now follows as a corollary.

\begin{proof}[Proof of Lemma~\ref{lem:localopt}]
Let $t_{\max}:=\max_{c'\in C\setminus A} score(t_{\max})$ and let $c_{\max}\in C\setminus A$ be a candidate with such highest score. 
If $supp_w(A)\geq t_{\max}$, it follows from Lemma~\ref{lem:insert} that if we execute $\ins(A,w,c_{\max}, t_{\max})$, we obtain a solution $(A+c_{\max}, w')$ with $supp_{w'}(A+c_{\max})=t_{\max}$. 
Now, by feasibility of vector $w'$, we have the inequality $\sum_{n\in N} s_n \geq \sum_{c\in A+c_{\max}} supp_{w'}(c) \geq (k+1)\cdot t_{\max}$, and thus $t_{\max}\leq \sum_{n\in N} s_n / (k+1) < \sum_{n\in N} s_n / k = \hat{t}$. 
By Lemma~\ref{lem:locality} above, having $\max_{c'\in C\setminus A} score(t_{\max}) < \hat{t}$ implies that $A$ satisfies $\hat{t}$-PJR, which is standard PJR.
\end{proof}

\subsection{A local search algorithm}\label{s:LS}

Suppose that we know of an efficient algorithm with good guarantees for maximin support but no guarantee for PJR, or we happen to know of a high quality solution in terms of maximin support but we ignore if it satisfies PJR. Can we use it to find a new solution of no lesser quality which is also guaranteed to satisfy PJR? And can we efficiently prove to a verifier that the new solution indeed satisfies PJR? We answer these questions in the positive for the first time.

We present a local search algorithm that takes an arbitrary feasible solution as input, and iteratively drops a member of least support and inserts a new candidate using the $\phragmms$ heuristic. The procedure always maintains or increases the value of the least member support, hence the quality of the solution is preserved. Furthermore, as the solution converges to a local optimum, it is guaranteed to satisfy the PJR after a certain number of iterations. 
Therefore, when used as a post-computation, this procedure can make any approximation algorithm for maximin support also satisfy PJR in a black-box manner. 
We remark here that such an application of the $\phragmms$ heuristic goes to show the robustness of the election rule; in particular, there is no evident way to build a similar post-computation from $\phragmen$~\cite{brill2017phragmen}, as the analysis of the PJR property in that rule is heavily dependent on the way the solution is constructed. 

As we did in Section~\ref{s:inserting}, in the following algorithm we assume that the background instance $(G=(N\cup C, E), s, k)$ is known and that does not need to be passed as input. 
Instead, the input is a feasible full solution $(A,w)$, and a parameter $\eps>0$. 
Our proposed algorithm $\local$ (Algorithm~\ref{alg:localpjr}) is presented below. 

\begin{algorithm}[htb]\label{alg:localpjr}
\SetAlgoLined
\KwData{Full feasible solution $(A,w)$, parameter $\eps>0$.}

Let $\hat{t}\leftarrow \sum_{n\in N} s_n / |A|$\;
\While{True}{
  Find tuple $(c_{\min}, t_{\min})$ so that $c_{\min}\in A$ and $t_{\min}=supp_w(c_{\min})=supp_w(A)$\;
  Let $(c_{\max}, t_{max})\leftarrow \maxscore(A,w)$ \quad \emph{// the candidate with highest score, and its score}\;
  \lIf {($t_{\max}< \min\{ (1+\eps)\cdot t_{\min}, \hat{t}\}$)} {\Return $(A,w)$}
  Update $(A,w)\leftarrow \ins(A-c_{\min},w,c_{\max},t_{\max})$\;
}
\caption{$\LSPJR(A,w,\eps)$}
\end{algorithm}

\begin{theorem}\label{thm:enabler}
For any parameter $\eps>0$ and a feasible full solution $(A,w)$, let $(A',w')$ be the output solution to $\LSPJR(A,w,\eps)$. Then: 
\begin{enumerate}
    \item solution $(A',w')$ is feasible and full, and $supp_{w'}(A')\geq supp_w(A)$; \label{item:support}
    \item solution $(A', w')$ satisfies the condition on Lemma~\ref{lem:locality} for parameter $t=\min\{(1+\eps)\cdot supp_{w'}(A'), \hat{t}\}$, where $\hat{t}=\sum_{n\in N} s_n / k$, so $A'$ satisfies $t$-PJR and standard PJR; 
		\label{item:tPJR}
		\item if $(A,w)$ has an $\alpha$-approximation guarantee for the maximin support objective, for some $\alpha\geq 1$, then the algorithm performs at most $k\cdot (1+\log_{1+\eps} \alpha)+1$ iterations, each taking time $O(|E|\cdot \log k)$; and \label{item:iterations}
		\item by setting $\eps\rightarrow\infty$, the algorithm finds a solution satisfying standard PJR in at most $k+1$ iterations. \label{item:infinity}

\end{enumerate}
\end{theorem}

This proves Theorem~\ref{thm:intro3}. 
Notice that point~\ref{item:iterations} establishes that $\LSPJR$ can be executed as a post-computation of any constant-factor approximation algorithm for maximin support in time $O(|E|\cdot k\log k)$. In particular, this complexity is dominated by that of all constant-factor approximations presented in this paper.
By point~\ref{item:infinity}, the algorithm can be sped up if we only care about standard PJR, or run further iterations to achieve a stronger parametric PJR guarantee. 
For the proof, we will need the following simple observation whose proof is delayed to Appendix~\ref{s:proofs}.

\begin{lemma}\label{lem:sameweight}
For a feasible vector $w\in \R^E$, two committees $A\subseteq A'$, and any threshold $t\geq 0$ it holds that $prescore_{(A,w)}(c',t)\geq score_{(A',w)}(c',t)$ for each candidate $c'\in C\setminus A'$.
\end{lemma}

\begin{proof}[Proof of Theorem~\ref{thm:enabler}]
We start with some notation. We use $i$ as a superscript to indicate the value of the different variables at the beginning of the the $i$-th iteration. We define $t^i_{stop}:=\min\{(1+\eps)\cdot t^i_{min}, \hat{t}\}$, so that the stopping condition reads $t^i_{\max} \stackrel{?}{<} t^i_{stop}$. Notice that $t^i_{stop}\geq t^i_{min}$ always holds by feasibility of $w^i$ and definition of $\hat{t}$. 

We prove point~\ref{item:support} by induction on $i$. 
If the stopping condition does not hold then $t^i_{\max} \geq t^i_{stop}\geq t^i_{\min}$. 
Next, by Lemma~\ref{lem:sameweight} we have $prescore_{(A^i-c_{\min}^i, w^i)}(c^i_{\max}, t^i_{\max}) \geq prescore_{(A^i, w^i)}(c^i_{\max}, t^i_{\max})=t^i_{\max}\geq t^i_{\min}$. 
On the other hand, it is evident that $supp_{w^i}(A^i-c^i_{\min})\geq supp_{w^i}(A^i)=t^i_{\min}$. 
Therefore, by Lemma~\ref{lem:insert} we have that  
$$supp_{w^{i+1}}(A^{i+1})\geq \min\{supp_{w^i}(A^i-c^i_{\min}), prescore_{(A^i-c_{\min}^i, w^i)}(c^i_{\max}, t^i_{\max})\} \geq t^i_{\min},$$
and that $(A^{i+1},w^{i+1})$ is feasible.  
This proves point~\ref{item:support}. Point~\ref{item:tPJR} is apparent from the stopping condition. 

We continue to point \ref{item:iterations}. Each iteration is dominated in complexity by the call to Algorithm $\maxscore$, which takes time $O(|E|\cdot \log k)$ by Theorem~\ref{thm:runtimes}. 
Hence, it remains to give a bound on the number $T$ of total iterations. 
To do that, we analyze the evolution of the least member support $t^i_{\min}=supp_{w^i}(A^i)$. First, we argue that if ever $t^i_{\min}=\hat{t}$, then the algorithm terminates immediately, i.e.~$i=T$; this is because in this case all members in $A^i$ must have a support of exactly $\hat{t}$, all voters a zero slack for threshold $\hat{t}$, and all candidates in $C\setminus A^i$ a zero prescore for threshold $\hat{t}$ and hence a score strictly below $\hat{t}$, and the stopping condition is fulfilled.
Next, we claim that for any iteration $i$ with $1\leq i<T-k$, we have $t^{i+k}_{\min}\geq (1+\eps)\cdot t^{i}_{\min}$. This is because, by Lemma~\ref{lem:insert}, in each iteration $j\geq i$ we are removing a member with least support while not increasing the number of members with support below 
$$prescore_{(A^j - c^j_{\min}, w^j)}(c^j_{\max}, t^j_{\max})\geq t^j_{\max}\geq t^j_{stop}\geq t^i_{stop}.$$
 
As there are only $k$ candidates, we must have that 
$t^{i+k}_{\min}\geq t^i_{stop}=\min\{(1+\eps)\cdot t^i_{\min}, \hat{t}\}$. 
Thus $t^{i+k}_{\min}$ is either at least $(1+\eps)\cdot t^i_{\min}$, or it is $\hat{t}$, but it cannot be the latter by the previous claim and the fact that $i+k<T$. This proves the new claim. Therefore, if the algorithm outputs a solution $(A^T,w^T)$ and has an input solution $(A^1, w^1)$ with an $\alpha$-approximation guarantee for the maximin support objective, then 
$$\alpha\cdot supp_{w^1}(A^1) \geq supp_{w^T}(A^T) \geq (1+\eps)^{\lfloor (T-1)/k \rfloor} \cdot supp_{w^1}(A^1).$$
Hence, $\alpha\geq (1+\eps)^{\lfloor (T-1)/k \rfloor}$, and $T\leq k\cdot(1+\log_{1+\eps} \alpha)+1$. This completes the proof of point \ref{item:iterations}.

For point \ref{item:infinity}, we consider again the previous inequality $t^{i+k}_{\min}\geq t^i_{stop}=\min\{(1+\eps)\cdot t^i_{\min}, \hat{t}\}$. By setting $i=1$ and $\eps\rightarrow \infty$, we have that either the stopping condition is triggered before the $(k+1)$-st iteration, or it must be the case that $t^{k+1}_{\min}\geq \hat{t}$, and the algorithm terminates immediately as argued previously. In either case, the claim on standard PJR follows from point~\ref{item:tPJR}.
\end{proof}

\section{Properties and algorithms for balanced solutions} \label{s:balanced}

Consider an instance $(G=(N\cup C, E), s, k)$ as defined in Section~\ref{s:prel}. 
For a fixed (possibly partial) committee $A\subseteq C$, recall from Section~\ref{s:prel} that a feasible edge weight vector $w\in\R^E$ is said to be \emph{balanced for $A$} if %
%
\begin{enumerate}
    \item it maximizes $\sum_{c\in A} supp_{w'}(c)$, over all feasible weight vectors $w'$, and \label{balanced1}
    \item it minimizes $\sum_{c\in A} supp_{w'}^2(c)$, over all vectors $w'$ that satisfy the condition above. \label{balanced2}
\end{enumerate}

\begin{proof}[Proof of Lemma~\ref{lem:balanced}]
The first statement is simply a special case of the second statement, for $r=1$. 
We temporarily skip the second statement and start by proving the third one, which we do by contradiction. 
We thus assume there is a voter $n\in N$ and two candidates $c\in C_n$, $c'\in C_n\cap A$, such that $w_{nc}>0$ and $supp_w(c) > supp_w(c')$. 
Let $\eps:=\min\{w_{nc}, (supp_w(c)-supp_w(c'))/2\}>0$, and define a new vector $w'\in\mathbb{R}^E$ where $w'_{nc}=w_{nc}-\eps$, $w'_{nc'}=w_{nc'}+\eps$, and $w'_e=w_e$ for every other edge $e\in E$. 
It can be checked that $w'$ is non-negative and feasible, and that it does not decrease the sum of member supports, i.e. $\sum_{d\in A} supp_{w'}(d)\geq \sum_{d\in A} supp_{w}(d)$, where the inequality is tight only if $c\in A$. We assume this to be the case, as otherwise the first condition on vector $w$ being balanced for $A$ is violated. On the other hand,
\begin{align*}
    \sum_{d\in A} supp_{w}^2(d) - \sum_{d\in A} supp_{w'}^2(d) &= supp_w^2(c) + supp_w^2(c) - supp_{w'}^2(c) - supp_{w'}^2(c') \\
    &= supp_w^2(c) + supp_w^2(c') - (supp_w(c)-\eps)^2 - (supp_w(c')+\eps)^2 \\
    &= 2\eps( supp_w(c) - supp_w(c')) -2\eps^2 \\
    &\geq 2\eps(2\eps) - 2\eps^2 = 2\eps^2 >0,
\end{align*}
which contradicts the second condition on vector $w$ being balanced for $A$.

We now prove the second statement, which says that function 
$F_{w'}^k:=\min_{A'\subseteq A, |A'|=r} \quad \sum_{c\in A'} supp_{w'}(c)$ 
is maximized by vector $w$ over all feasible vectors $w'\in\R^E$, for all $1\leq r\leq |A|$. 
Assume by contradiction that there is a threshold $r$ and feasible $w'$ such that $F_r^A(w')>F_r^A(w)$. We can also assume without loss of generality that 
\begin{enumerate}
    \item $\sum_{c\in A} supp_{w'}(c)=\sum_{c\in A} supp_{w}(c)$, i.e. $w'$ also satisfies the first condition of a balanced weight vector; and  
    \item we enumerate the candidates in $A=\{c_1, \cdots, c_{|A|}\}$ so that whenever $i<j$ we have that $supp_w(c_i)\leq supp_w(c_j)$, and in case of a tie, $supp_w(c_i)= supp_w(c_j)$, we have that $supp_{w'}(c_i)\leq supp_{w'}(c_j)$. 
\end{enumerate}

With a candidate enumeration as above, it follows that $F^A_r(w)=\sum_{i=1}^r supp_w(c_i)$, while for vector $w'$ we have the inequality $F^A_r(w')\leq \sum_{i=1}^r supp_{w'}(c_i)$. 
Thus, by our assumption by contradiction, 
$$\sum_{i=1}^r supp_{w'}(c_i) \geq F_r^A(w') > F_r^A(w) = \sum_{i=1}^r supp_{w}(c_i), \quad \text{and}$$
\begin{align*}
    \sum_{i=r+1}^{|A|} supp_{w'}(c_i) &= \sum_{i=1}^{|A|} supp_{w'}(c_i) - \sum_{i=1}^{r} supp_{w'}(c_i) \\
    & = \sum_{i=1}^{|A|} supp_{w}(c_i) - \sum_{i=1}^{r} supp_{w'}(c_i) \\
    & < \sum_{i=1}^{|A|} supp_{w}(c_i) - \sum_{i=1}^{r} supp_{w}(c_i) 
    = \sum_{i=r+1}^{|A|} supp_{w}(c_i). \\
\end{align*}
Now define the edge vector $f:=w'-w\in\mathbb{R}^E$ and consider it as a flow over the network $(N\cup A, E)$. 
We have that $f$ has a zero net demand over set $N$, by our first assumption, and the previous two inequalities show that $f$ has a positive net demand over set $\{c_1, \cdots, c_r\}$ and a positive net excess over set $\{c_{r+1}, \cdots, c_{|A|}\}$. Thus, by the flow decomposition theorem, $f$ can be decomposed into circulations and simple paths, where every path starts in a vertex with positive demand and ends in a vertex with positive excess, and there must be a simple path carrying some $\delta>0$ flow from $c_j$ to $c_i$ for some $1\leq i\leq r<j\leq |A|$. 
Moreover, by our second assumption, it must be the case that $supp_w(c_i)<supp_w(c_j)$, because in case of a tie we would have that $supp_{w'}(c_i)<supp_{w'}(c_j)$ and so $f$ would have a net excess on $c_i$ and net demand on $c_j$, and the $c_i$-to-$c_j$ path would not exist in the flow decomposition.

Let $f'$ be a sub-flow of $f$ that carries $\eps$ units of flow from $c_j$ to $c_i$, where $\eps:=\min\{\delta, \frac{1}{2}(supp_w(c_j) - supp_w(c_i))\}>0$. 
The fact that $f'$ is a sub-flow of $f$ implies that $w'':=w+f'$ is a feasible non-negative solution,%
\footnote{For a proof of feasibility and non-negativity, see the argument at the end of the proof in Lemma~\ref{lem:2sols}.} 
with 
\begin{align*}
    \sum_{c\in A} supp_{w}^2(c) - \sum_{c\in A} supp_{w''}^2(c) &= supp_{w}^2(c_i) + supp_w^2(c_j) - (supp_{w}(c_i) +\eps)^2 - (supp_{w}(c_j) - \eps)^2\\
    &= 2\eps(supp_w(c_j) - supp_w(c_i)) - 2\eps^2 \\
    &\geq 2\eps(2\eps) - 2\eps^2 = 2\eps^2>0,
\end{align*}
which contradicts the fact that $w$ minimizes the sum of supports squared. This completes the proof. 
\end{proof}

\subsection{Algorithms to compute balanced weight vectors}

Let $E_A\subseteq E$ be the restriction of the input edge set over edges incident to committee $A$, and let $D\in\{0,1\}^{A\times E_A}$ be the vertex-edge incidence matrix for $A$. 
For any weight vertex $w\in\R^{E_A}$, the support that $w$ assigns to candidates in $A$ is given by vector $Dw$, so that $suppw(c)=(Dw)_c$ for each $c\in A$. 
We can now write the problem of finding a balanced weight vector as a convex program:
\begin{align*}
    \text{Minimize} \quad & \|Dw\|^2 \\
    \text{Subject to } \quad & w\in\R^{E_A}, \\
    & \sum_{c\in C_n} w_{nc} \leq s_n \quad \text{for each } n\in N, \text{ and} \\
    & \mathbbm{1}^{\intercal} Dw = \sum_{n\in \cup_{c\in A} N_c} s_n,
\end{align*}
where the first two lines of constraints corresponds to non-negativity and feasibility conditions (see inequality~\ref{eq:feasible}), and the last line ensures that the sum of supports is maximized, where $\mathbbm{1}\in\mathbb{R}^A$ is the all-ones vector. 
A balanced weight vector can thus be computed with numerical methods for quadratic convex programs.

However, there is a more efficient method using techniques for parametric flow, which we sketch now. Hochbaum and Hong~\cite[Section 6]{hochbaum1995strongly} consider a network resource allocation problem which generalizes the problem of finding a balanced weight vector: given a network with a single source, single sink and edge capacities, minimize the sum of squared flows over the edges reaching the sink, over all maximum flows. 
They show that this is equivalent to a parametric flow problem called \emph{lexicographically optimal flow}, studied by Gallo, Gregoriadis and Tarjan~\cite{gallo1989fast}. 
In turn, in this last paper the authors show that, even though a parametric flow problem usually requires solving several consecutive max-flow instances, this particular problem can be solved running a single execution of the FIFO preflow-push algorithm proposed by Goldberg and Tarjan~\cite{goldberg1988new}.

Therefore, the complexity of finding a balanced weight vector is bounded by that of Goldberg and Tarjan’s algorithm, which is $O(n^3)$ for a general $n$-node network. 
However, Ahuja et al.~\cite{ahuja1994improved} showed how to optimize several popular network flow algorithms for the case of bipartite networks, where one of the partitions is considerably smaller than the other. Assuming the sizes of the bipartition are $n_1$ and $n_2$ with $n_1 \ll n_2$, they implement a two-edge push rule that allows one to "charge" most of the computation weight to the nodes on the small partition, and hence obtain algorithms whose running times depend on $n_1$ rather than $n$. 
In particular, they show how to adapt Goldberg and Tarjan’s algorithm to run in time $O(e\cdot n_1+n_1^3)$, where $e$ is the number of edges. 
For our particular problem, which can be defined on a bipartite graph $(N\cup A, E_A)$ where $|A|\leq k\ll |N|$, we obtain thus an algorithm that runs in time $O(|E_A|\cdot k + k^3)$.

\section{Network flow results}\label{s:flow}

In the section we present proofs to Lemmas \ref{lem:2sols} and \ref{lem:N_a}, the key results at the heart of the constant-factor approximation guarantees we derive for the $\MMS$ and the $\phragmms$ rules, respectively. 
We start by introducing some necessary definitions and results related to network flow theory.

Throughout this section we regard the input bipartite approval graph $G=(N\cup C,E)$ as a network, and regard an edge vector $f\in\mathbb{R}^{E}$ as a \emph{flow} over it. For each edge $nc\in E$, we consider the flow on that edge to be directed toward $c$ if $f_{nc}>0$, and directed toward $n$ if $f_{nc}<0$. 
Consequently, the \emph{excess} of a voter $n\in N$ is defined as $e_f(n):=\sum_{c\in C_n} f_{nc}$, and the excess of a candidate $c\in C$ is defined as $e_f(c):=-\sum_{n\in N_c} f_{nc}$. 
%A vertex $x\in N\cup C$ is an \emph{excess} vertex if $e_f(x)>0$, and a \emph{deficit} vertex if $e_f(x)<0$. 
Moreover, for a set of vertices $S\subseteq N\cup C$, we define its \emph{net excess} as $e_f(S):=\sum_{x\in S} e_f(x)$.
A vector $f'\in\mathbb{R}^E$ is a \emph{sub-flow of $f$} if 
\begin{itemize}
	\item for each edge $nc\in E$ with $f'_{nv}\neq 0$, flows $f'_{nc}$ and $f_{nc}$ have the same sign (i.e., direction) and $|f'_{nc}|\leq |f_{nc}|$, and
	\item for each vertex $x\in N\cup C$ with $e_{f'}(x)\neq 0$, the excesses $e_{f'}(x)$ and $e_{f}(x)$ have the same sign and $|e_{f'}(x)|\leq |e_{f}(x)|$.
\end{itemize}

We now list two properties of flows and sub-flows. 
The proof of Theorem~\ref{thm:decomposition} can be found in \cite[Thm.~3.15]{ahuja1994improved}, while the proof of Lemma~\ref{lem:subflow} is delayed to Appendix~\ref{s:proofs}.

\begin{theorem}[Flow Decomposition Theorem]\label{thm:decomposition}
Any flow $f\in\mathbb{R}^E$ can be decomposed into a finite number of cycles and simple paths, such that every path $p$ is a non-zero sub-flow of $f$ that starts in a vertex with strictly positive excess and ends in a vertex with strictly negative excess. 
\end{theorem}

\begin{lemma}\label{lem:subflow}
If $w, w'\in\R^E$ are two non-negative and feasible edge weight vectors for the given election instance, and $f'\in\mathbb{R}^E$ is a sub-flow of $f:=w'-w$, then both $w+f'$ and $w'-f'$ are non-negative and feasible as well.
\end{lemma}

We prove now that for any partial solution, there is always an unelected candidate that can be added with large support. To show this, we assume we know the edge weight vector of an optimal solution, and combine it with the weight vector of the current solution. For convenience, we repeat the statement of Lemma~\ref{lem:2sols}.

\begin{lemma*}
If $(A^*, w^*)$ is an optimal solution to maximin support, and $(A,w)$ is a partial solution with $|A|\leq k$ and $A\neq A^*$, there is a candidate $c'\in A^*\setminus A$ and feasible solution $(A+c', w')$ such that 
$$\supp_{w'}(A+c')\geq \min\Big\{\supp_w(A), \frac{1}{2} \supp_{w^*}(A^*)\Big\}.$$
\end{lemma*}

\begin{proof}
Let $(A,w)$ and $(A^*, w^*)$ be as in the statement, with corresponding values $t^*:=\supp_{w^*}(A^*)$ and $t:=\min\{\supp_w(A), t^*/2\}$. To prove the lemma, it suffices to find a candidate $c'\in A^*\setminus A$ and a feasible weight vector $w'\in\R^E$ such that $\supp_{w'}(A+c)\geq t$.

By decreasing some components in $w$ and $w^*$, we can assume without loss of generality that $\supp_w(c)=t$ if $c\in A$, zero otherwise, and $\supp_{w^*}(c)=t^*$ if $c\in A^*$, zero otherwise. 
Consider flow $f:=w^* - w\in\mathbb{R}^E$ over the network induced by $N\cup A\cup A^*$. 
%We partition this network into four subsets: $N$, $A\setminus A^*$, $A^*\setminus A$ and $A^*\cap A$. 
It is easy to see that 
\begin{itemize}
\item $N$ has a net excess $e_f(N)=|A^*|\cdot t^* - |A|\cdot t$,
\item $A\setminus A^*$ has a net excess $e_f(A\setminus A^*)=|A\setminus A^*|\cdot t$,
\item $A^*\setminus A$ has a net excess $e_f(A^*\setminus A)=-|A^*\setminus A|\cdot t^*$, and
\item $A^*\cap A$ has a net excess $e_f(A^*\cap A)=-|A^*\cap A| \cdot (t^*-t)$.
\end{itemize}

By Theorem~\ref{thm:decomposition}, we can decompose flow $f$ into circulations and simple paths, where each path is a sub-flow of $f$ that starts (ends) in a vertex with positive (negative) excess. 
Let $f'$ is the sum of all paths that start inside subset $N\cup (A^*\cap A)$ and end outside of it. 
It follows that $f'$ is also a sub-flow of $f$, and that the amount of flow it extracts from this subset is at least
\begin{align*}
e_f(N\cup(A^*\cap A)) &= e_f(N) + e_f(A^*\cap A)\\
&= |A^*|\cdot t^* - |A|\cdot t - |A^*\cap A| \cdot (t^*-t)\\
&= |A^*\setminus A|\cdot t^* - |A\setminus A^*|\cdot t \\
&\geq |A^*\setminus A|\cdot (t^*-t) \geq |A^*\setminus A|\cdot t,
\end{align*}
where the last two inequalities follow from $|A^*|\geq |A|$ and $t\leq t^*/2$, respectively.

Next, we claim that each path in $f'$ actually must start in $N$ and end in $A^*\setminus A$. 
Indeed, none of these paths can start in $A^*\cap A$, because each vertex in this last set has negative excess, and similarly none can end in $A\setminus A^*$, because each vertex in this last set has positive excess. 
Therefore, $f'$ carries flow exclusively from $N$ to $A^*\setminus A$, and by the previous inequality and an averaging argument, there must be a vertex $c'$ in $A^*\setminus A$ that receives a flow from $f'$ of value at least $t$. 

Finally, if we define vector $w':=w+f'$, it is non-negative and feasible by Lemma~\ref{lem:subflow}, and it provides the same supports to the members of $A$ as $w$ does, namely $t$, and a support of at least $t$ to candidate $c'$. Hence, $\supp_{w'}(A+c')\geq t$, as claimed.   
\end{proof}

Next, we prove that if we start with a partial solution that is balanced, then not only are there unelected candidates that can be appended with high support, but they also have large scores, so we can find such a candidate efficiently with our heuristic. 
To show this, we prove that there must be a subset of voters with large aggregate vote strength who do not get as many representatives in the current solution as they do in the optimal solution, so they have large slacks and their unelected representatives have large scores. For convenience, we repeat the statement of Lemma~\ref{lem:N_a}.

\begin{lemma*}
If $(A^*, w^*)$ is an optimal solution with $t^*=\supp_{w^*}(A^*)$, and $(A,w)$ is a balanced solution with $|A|\leq k$ and $A\neq A^*$, then for each $0\leq a\leq 1$ there is a subset $N(a)\subseteq N$ of voters such that 
\begin{enumerate}
	\item each voter $n\in N(a)$ has a neighbor in $A^*\setminus A$;
	\item for each voter $n\in N(a)$, we have $\supp_w(A\cap C_n)\geq at^*$;
	\item $\sum_{n\in N(a)} s_n \geq |A^* \setminus A|\cdot (1-a) t^*$; and
	\item for any $b$ with $a\leq b\leq 1$ we have that $N(b)\subseteq N(a)$, and for each $n\in N(a)$ we have that $n$ is also in $N(b)$ if and only if property 2 above holds for $n$ with parameter $a$ replaced by $b$.
\end{enumerate}
\end{lemma*}

\begin{proof}
Fix a parameter $0\leq a\leq 1$ and define the set of voters $N':=\{n\in N: \ \supp_w(A\cap C_n)\geq at^*\}$, where $\supp_w(\emptyset)=\infty$ by convention. If we define $N(a)\subseteq N'$ as those voters in $N'$ that have a neighbor in $A^*\setminus A$, then properties 1, 2 and 4 become evident. Hence, it only remains to prove the third property.

\emph{Claim A.} There is no edge with non-zero weight in $w$ between $N\setminus N'$ and $A':=\{c\in A: \ \supp_w(c)\geq at^*\}$. 
Indeed, if there was a pair $n\in N\setminus N'$, $c\in A'$ with $w_{nc}>0$, then by property 3 of Lemma~\ref{lem:balanced} we would have $\supp_w(A\cap C_n)=\supp_w(c)\geq at^*$, contradicting the fact that $n$ is not in set $N'$. 
Thus, all of the vote strength from voters in $N\setminus N'$ must be directed to members in $A\setminus A'$, and we get the inequality
$$\sum_{n\in N\setminus N'} s_n \leq \sum_{c\in A\setminus A'} \supp_w(c) < |A\setminus A'|\cdot at^*< |A^*\setminus A'|\cdot at^*.$$

Next, by reducing some components in vectors $w$ and $w^*$, we can assume without loss of generality that $\supp_{w^*}(c)=t^*$ if $c\in A^*$, zero otherwise, and $\supp_{w}(c)=a t^*$ if $c\in A^*\cap A'$, zero otherwise; we call this the ``wlog assumption'', and notice that $w$ and $w^*$ are not necessarily balanced anymore, but are still feasible.
Consider flow $f:=w^* - w\in\mathbb{R}^E$ over the network induced by $N\cup A^*$. 
We partition this network into five subsets: $N'$, $N\setminus N'$, $A^*\cap A'$, $A^*\setminus A$ and $A^*\cap A\setminus A'$; see Figure~\ref{fig:sets}.  
It is easy to see that 
\begin{itemize}
\item $A^*\cap A'$ has a net excess $e_f(A^*\cap A')=-|A^*\cap A'|\cdot (1-a)t^*$,
\item $N$ has a net excess $e_f(N)=|A^*|\cdot t^* - |A^*\cap A'|\cdot a t^*$, and
\item by the last inequality, $N\setminus N'$ has a net excess $e_f(N\setminus N')\leq \sum_{n\in N\setminus N'} s_n < |A^*\setminus A'|\cdot at^*$.
\end{itemize} 

\begin{figure}[htb]
  \centering
	\includegraphics[width=\linewidth,natwidth=360,natheight=300]{figure-Na.pdf}
  \caption{Flow $f'$ starts in $N'\cup (A^*\cap A')$ and must end in $A^*\setminus A$, because it cannot visit $N\setminus N'$ nor $A^*\cap A\setminus A'$.}
  \label{fig:sets}
\end{figure}

By Theorem~\ref{thm:decomposition}, we can decompose flow $f$ into circulations and simple paths, where each path is a sub-flow of $f$. 
Let $f'$ be the sum of all paths that start inside subset $N'\cup(A^*\cap A')$ and end outside of it. 
It follows that $f'$ is also a sub-flow of $f$, and the amount of flow it extracts from this subset is at least
\begin{align*}
e_f & (N'\cup(A^*\cap A')) \\
=& e_f(N) - e_f(N\setminus N') + e_f(A^*\cap A') \\
>& |A^*|\cdot t^* - |A^*\cap A'|\cdot a t^* - |A^*\setminus A'|\cdot at^* - |A^*\cap A'|\cdot (1-a)t^* \\
=& |A^*\setminus A'|\cdot (1-a)t^*.
\end{align*}
Now, where does all this flow go?

\emph{Claim B.} Every path in $f'$ must end in $A^*\setminus A$. 
This is because every such path starts in either $N'$ or $A^*\cap A'$, vertices in $N'$ have no neighbors in $A\setminus A'$ (by definitions of $N'$ and $A'$), and furthermore there is no flow possible from $(A^*\cap A')\cup (A^*\setminus A)$ to $N\setminus N'$ in $f=w^*-w$ because $w$ has no flow from $N\setminus N'$ toward $A'$ (by Claim A) nor toward $A^*\setminus A$ (by the wlog assumption); see Figure~\ref{fig:sets}. 

Therefore, the paths in $f'$ carry a flow of at least $|A^* \setminus A'|\cdot (1-a)t^*$ toward $A^*\setminus A$. 
Finally, for each such path, the last edge goes from $N'$ to $A^*\setminus A$, so it originates in $N(a)$. 
This proves that $\sum_{n\in N(a)} s_n> |A^* \setminus A|\cdot (1-a) t^*$, which is the third property.
\end{proof}

\section{Algorithmic considerations for the new election rule}\label{s:algorithms}

The goal of this section is threefold. 
First, we prove Theorem~\ref{thm:runtimes} and establish how our heuristic for candidate selection, described in Section~\ref{s:inserting}, can be computed efficiently. 
Second, we improve upon the runtime analysis of $\phragmms$ given in Section~\ref{s:315}, and show that each iteration can be executed in time $O(\bal + |E|)$, down from $O(\bal + |E|\cdot \log k)$. 
Finally, we provide further details on the similarities and differences between $\phragmms$ and $\phragmen$. 


As we did in Section~\ref{s:inserting}, we assume in the following that the election instance $(G=(N\cup C, E), s, k)$ is known and does not need to be given as input. Instead, the input is a partial solution $(A,w)$ with $|A|\leq k$. The list of member supports $(supp_w(c))_{c\in A}$ is implicitly passed by reference and updated in every algorithm.

We start with Algorithm~\ref{alg:maxprescore}, which shows how to find the candidate with highest parameterized score for a given threshold $t$.

\begin{algorithm}[htb]
\SetAlgoLined
\KwData{Partial solution $(A,w)$, threshold $t\geq 0$.}
\lFor{each voter $n\in N$}{
compute $slack(n,t)=s_n-\sum_{c\in A\cap C_n} w_{nc}\cdot \min\{1, t/supp_w(c)\}$
}
\lFor{each candidate $c'\in C\setminus A$}{
compute $\prescore(c',t)=\sum_{n\in N_{c'}} \slack(n,t)$
}
Find a candidate $c_t\in\argmax_{c'\in C\setminus A} \prescore(c', t)$\;
\Return $(c_t, \prescore(c_t, t))$\;
 \caption{$\maxprescore(A,w,t)$}
\label{alg:maxprescore}
\end{algorithm}

\begin{lemma}
For a partial solution $(A,w)$ and threshold $t\geq 0$, $\maxprescore(A,w,t)$ executes in time $O(|E|)$ 
and returns a tuple $(c_t,p_t)$ such that $c_t\in C\setminus A$ 
and 
$$p_t=\prescore(c_t,t)=\max_{c'\in C\setminus A} \prescore(c',t).$$
\end{lemma}

\begin{proof}
The correctness of the algorithm directly follows from the definitions of slack and parameterized score. The running time is $O(|E|)$ because each edge in the approval graph $G=(N\cup V, E)$ is inspected at most once in each of the two loops. The first loop also inspects each voter, but we have $|N|=O(|E|)$ since we assume that $G$ has no isolated vertices.
\end{proof}

We move on to computing the highest score. 
For a fixed partial solution $(A,w)$ and for a candidate $c'\in C\setminus A$, consider the function 
\begin{align}\label{eq:scorefunction}
f_{c'}(t):=\prescore(c',t)-t
\end{align}
$$$$ 
in the interval $[0,\infty)$. 
Notice from the definition of parameterized score that this function is convex, continuous and strictly decreasing with no lower bound, and that $f_{c'}(0)\geq 0$; hence it has a unique root corresponding to $score(c')$. We could approximate this root via binary search -- however, we can do better. 
Function $f_{c'}(t)$ is piece-wise linear: if we sort the member supports $\{supp_w(c): \ c\in A\}=\{t_1, \cdots, t_r\}$ so that $t_1 < \cdots < t_r$ for some $r\leq |A|$, then $f_{c'}(t)$ is linear in each interval $[0, t_1), [t_1, t_2), \cdots, [t_r, \infty)$.
%
Similarly, 
$$f_{\max}(t):= \max_{c'\in C\setminus A} f_{c'}(t) = \max_{c'\in C\setminus A} \prescore(c',t) -t$$ 
is a continuous and strictly decreasing function in the interval $[0,\infty)$, with a unique root $t_{\max}=\max_{c'\in C\setminus A} score(c')$. Unfortunately, this function is in general not linear within each of the intervals above.%
%
\footnote{It is easy to see that function $f(t)$ is piece-wise linear with $O(|C|\cdot k)$ pieces in total. Hence, one could find its root via binary search by making $O(\log |C|+ \log k)$ calls to $\maxprescore$. 
We present a better approach that only requires $O(\log k)$ such calls.} %
%
Still, it will be convenient to use binary search to identify the interval that contains $t_{\max}$. We do so in Algorithm~\ref{alg:interval}. The next lemma follows from our exposition and its proof is skipped.

\begin{algorithm}[htb]
\SetAlgoLined
\KwData{Partial solution $(A,w)$.}
Sort the member supports to obtain $0=t_0<t_1<\cdots <t_r$, where $\{t_1, \cdots, t_r\}=\{supp_w(c): \ c\in A\}$\;
\If{$p_{t_r}\geq t_r$ where $(c_{t_r},p_{t_r})\leftarrow \maxprescore(A,w,t_r)$}{
	\Return $t_r$\;
}
Let $j_{lo}=0$, $j_{hi}=r-1$\;
\While{$j_{lo}<j_{hi}$}{
  Let $j=\lceil (j_{lo}+j_{hi})/2 \rceil$\;
  \leIf{$p_{t_j}\geq t_j$ where $(c_{t_j},p_{t_j})\leftarrow \maxprescore(A,w,t_j)$}{
  Set $j_{lo}\leftarrow j$}{
  Set $j_{hi}\leftarrow j-1$}
}
\Return $t_{j_{lo}}$\;

 \caption{$\interval(A,w)$}
\label{alg:interval}
\end{algorithm}

\begin{lemma}\label{lem:interval}
For a partial solution $(A,w)$, $\interval(A,w)$ makes $O(\log |A|)$ calls to $\maxprescore$, and thus runs in time $O(|E|\cdot \log k)$. It returns a value $t'$ with $t'\leq t_{\max}:=\max_{c'\in C\setminus A} \score(c')$, and such that for each candidate $c'\in C\setminus A$, the value of $\prescore(c',t)$ is linear in $t$ within the interval $[t',t_{\max}]$.
\end{lemma}

Moving on, for a candidate $c'\in C\setminus A$ and a value $x\geq 0$, consider the linearization of function $f_{c'}(t)$ at $x$ -- more precisely, the linear function that coincides with $f_{c'}(t)$ over the interval $[x, x+\eps]$ as $\eps>0$ tends to zero. 
If we denote by $r_{c', x}$ the unique root of this linearization, we have that
\begin{align*}
    0=& f_{c'}(r_{c', x})|_{\text{linearized at } x}\\
    =& \prescore(c', r_{c', x})|_{\text{linearized at } x} - r_{c', x}\\
    =& \sum_{n\in N_{c'}} \slack(n,r_{c', x})|_{\text{linearized at } x} - r_{c', x}\\
    =& \sum_{n\in N_{c'}} \bigg( s_n - \sum_{c\in A\cap C_n: \ \supp_{w}(c)< x}w_{nc} \\
		&- \sum_{c\in A\cap C_n: \ \supp_w(c)\geq x} \frac{w_{nc}\cdot r_{c', x} }{\supp_w(c)} \bigg) - r_{c', x},
\end{align*} 
%
where we used the definitions of parameterized score and slack. Solving for $r_{c', x}$, we obtain
%
\begin{align}\label{eq:linearized}
    r_{c', x}=\frac{\sum_{n\in N_{c'}} \Big( s_n - \sum_{c\in A\cap C_n: \ \supp_w(c)< x} w_{nc} \Big)}%
    {1+\sum_{n\in N_{c'}} \sum_{c\in A\cap C_n: \ \supp_w(c)\geq x} \frac{w_{nc}}{\supp_w(c)}}.
\end{align}

Now we make a couple of remarks about these linearization roots. 
First, since $f_{c'}(t)$ is a convex decreasing function, any linearization will lie to its left, and in particular any linearization root will lie to the left of its own root, i.e., 
$$r_{c', x}\leq \score(c') \quad \text{for each } c'\in C\setminus A \text{ and each } x\geq 0.$$

\begin{figure}[h]
  \centering
	\includegraphics[width={\linewidth},natwidth=300,natheight=270]{figure-maxscore.pdf}
  \caption{For each candidate $c_i$, the root $\score(c_i)$ of function $f_{c_i}(t)$ lies to the right of $r_{c_i, t'}$, the root of its linearization at $t'$. These two roots coincide for $c_2=c_{\max}$. }
  \label{fig:maxscore}
\end{figure}

On the other hand, for the candidate $c_{\max}$ with highest score $t_{\max}$, and for $x=t'$, the output of $\interval(A,w)$, we have that the corresponding linearization coincides with function $f_{c_{\max}}(t)$ in the interval $[t', t_{\max}]$, so the linearization root $r_{c_{\max}, t'}$ equals the function root $t_{\max}=\score(c_{\max})$. 
See Figure~\ref{fig:maxscore}. Consequently, %
%
\begin{align*}
r_{c_{\max}, t'} &= \score(c_{\max}) = \max_{c'\in C\setminus A} \score(c') \\
	&\geq \max_{c'\in C\setminus A} r_{c', t'} \geq r_{c_{\max}, t},
\end{align*}
%
i.e., $c_{\max}$ is simultaneously the candidate with highest score and the one with highest linearization root at $t'$, and these values coincide. 
We use this fact to find the candidate and its score. We formalize these observations in Algorithm~\ref{alg:maxscore} and the lemma below.

\begin{algorithm}[htb]
\SetAlgoLined
\KwData{Partial solution $(A,w)$.}
Let $t'\leftarrow \interval(A,w)$\;

\For{each voter $n\in N$}{
Compute $p_n:=s_n-\sum_{c\in A\cap C_n: \ \supp_w(c)< t'} w_{nc}$\;
Compute $q_n:=\sum_{c\in A\cap C_n: \ \supp_w(c)\geq t'} w_{nc}/\supp_w(c)$\;
}
\lFor{each candidate $c'\in C\setminus A$}{
compute $r_{c', t'}=\frac{\sum_{n\in N_{c'}} p_n}{1+\sum_{n\in N_{c'}} q_n}$}
Find a candidate $c_{\max}\in\argmax_{c'\in C\setminus A} r_{c', t'}$\;
\Return $(c_{\max}, r_{c_{\max}, t'})$\;
 \caption{$\maxscore(A,w)$}
\label{alg:maxscore}
\end{algorithm}

\begin{lemma}\label{lem:maxscore}
For a partial solution $(A,w)$, $\maxscore(A,w)$ runs in time $O(|E|\cdot \log k)$ and returns a tuple $(c_{\max}, t_{\max})$ such that $c_{\max}\in C\setminus A$ and $t_{\max}=\score(c_{\max})=\max_{c'\in C\setminus A} \score(c')$.
\end{lemma}
\begin{proof}
The correctness of the algorithm follows from the arguments above. 
Each of the \textbf{for} loops executes in time $O(|E|)$ because in each one of them each edge is examined at most once. 
The running time is dominated by the call to algorithm $\interval(A,w)$, taking time $O(|E|\cdot \log k)$.
\end{proof}

This completes the proof of Theorem~\ref{thm:runtimes}. 
We highlight again that the heuristic for candidate selection in $\phragmms$ runs in time $O(|E|\cdot \log k)$, thus almost matching the complexity of the heuristic in $\phragmen$ which is $O(|E|)$ per iteration. 

Next, we reconsider the complexity of $\phragmms$ (Algorithm~\ref{alg:balanced}). 
At the start of each iteration with current partial solution $(A,w)$, notice by Lemma~\ref{lem:315localoptimality} that the highest score $t_{\max}$ must be lower than the least member support $t_1=supp_w(A)$. So, $t_{\max}$ lies in the interval $[0,t_1]$, and we can skip the computation of Algorithm $\interval(A,w)$ as we know that it would return $t'=0$. 
Without this computation, $\maxscore(A,w)$ (Algorithm~\ref{alg:maxscore}) runs in time $O(|E|)$, so the runtime of a full iteration of $\phragmms$ can be performed in time $O(\bal + |E|)$, down from $O(\bal + |E|\cdot \log k)$ as was established in Section~\ref{s:315}.

Finally, we discuss some similarities and differences between the $\phragmms$ and $\phragmen$ heuristics. 
For the sake of completeness, we present here the $\phragmen$ algorithm explicitly. 
We note that the version of $\phragmen$ proposed in~\cite{brill2017phragmen} only considers unit votes. 
In Algorithm~\ref{alg:phragmen} we give a generalization that admits arbitrary vote strengths. 
Clearly, each one of the $k$ iterations of the main loop runs in time $O(|E|)$, because each of the two internal \textbf{for} loops examines each edge in $E$ at most once. 

\begin{algorithm}[htb]
\SetAlgoLined
\KwData{Bipartite approval graph $G=(N\cup C, E)$, vector $s$ of vote strengths, target committee size $k$.}
Initialize $A=\emptyset$, $load(n)=0$ for each $n\in N$, and $load(c')=0$ for each $c'\in C$\;
\For{$i=1,2,\cdots k$}{
\lFor{each candidate $c'\in C\setminus A$}{
update $load(c') \leftarrow \frac{1+\sum_{n\in N_{c'}} s_n\cdot load(n)}{\sum_{n\in N_{c'}} s_n}$}
Find $c_{\min}\in \arg\min_{c'\in C\setminus A} load(c')$\;
Update $A\leftarrow A+c_{\min}$\;
\For{each voter $n\in N_{c_{\min}}$}{
Update $load(n)\leftarrow load(c_{\min})$\;
}
}
\Return $A$\;
\caption{$\phragmen$, proposed in~\cite{brill2017phragmen}}
\label{alg:phragmen}
\end{algorithm}

Assume that we consider inserting a candidate $c'\in C\setminus A$ to the partial solution $(A,w)$, and recall that for a voter $n\in N_{c'}$ approving of that candidate, and a threshold $t$, we define 
$$\slack(n,t)=s_n - \sum_{c\in A\cap C_n} w_{nc} \cdot\min \{1, t/\supp_w(c)\}.$$ 
%
This formula expresses the fact that to each current member $c\in A\cap C_n$, we reduce its edge weight $w_{nc}$ by multiplying it by a factor $\min \{1, t/\supp_w(c)\}$, and use the now-available vote strength from voter $n$ (its \emph{slack}) to give support to the new member $c'$. This edge multiplication factor is somewhat involved but sensible, as it removes a higher fraction of vote from members with higher support, and leaves members with low support untouched; see Section~\ref{s:inserting} for further intuition.

In contrast, in the same context, we claim that the $\phragmen$ heuristic can be thought of as using a constant edge multiplication factor $t/\supp_w(A\cap C_n)$, where we recall that $\supp_w(A\cap C_n):=\min_{c\in A\cap C_n} \supp_w(c)$. This is, of course, a much simpler approach, corresponding to a coarser solution rebalancing method. 

We now prove our claim. Suppose we use the edge multiplication factor above, and consequently define the voter's slack as
\begin{align}\label{eq:alt-slack}
\slack'(n,t):=s_n - \frac{t}{\supp_w(A\cap C_n)}\sum_{c\in A\cap C_n} w_{nc},
\end{align}
%
If we define parameterized scores and scores as before, we can retrieve the new score value for candidate $c'$ by finding the root of the function in equation~\eqref{eq:scorefunction}, which is now linear. 
With a similar computation as the one we did for equation~\eqref{eq:linearized}, we obtain
\begin{align*}
\score'(c') 
&=\frac{\sum_{n\in N_{c'}} s_n}{1+\sum_{n\in N_{c'}} \frac{1}{\supp_w(A\cap C_n)} \sum_{c\in A\cap C_n} w_{nc} } \\
&\geq  \frac{\sum_{n\in N_{c'}} s_n}{1+\sum_{n\in N_{c'}} \frac{s_n}{\supp_w(A\cap C_n)}}, 
\end{align*}
%
where the inequality follows by feasibility (inequality~\ref{eq:feasible}), and is tight if we assume that the current partial solution $(A,w)$ uses up all the vote strength of voter $n$ whenever $A\cap C_n$ is non-empty. If $A\cap C_n=\emptyset$, then $\supp_w(A\cap C_n)=\infty$ by convention and the corresponding term vanishes in the denominator. 
%
This new score, to be maximizes among all unelected candidates, corresponds precisely to the inverse of the \emph{candidate load} being minimized in the $\phragmen$ heuristic; see Algorithm~\ref{alg:phragmen}. The corresponding \emph{voter load} is in turn set to the inverse of $\supp_w(A\cap C_n)$, which the algorithm updates with the assumption that the new candidate $c'$ always becomes the member with least support. This completes the proof of the claim.

In view of this last result, we can say that our new heuristic provides two main advantages with respect to $\phragmen$: 
First, by using edge weights explicitly, the algorithm handles a more robust notion of loads. 
This enables $\phragmms$ to deal with arbitrary input solutions, a fact that we exploit in Appendix~\ref{s:LS}, and in contrast to $\phragmen$ which needs to make assumptions on the structure of the current solution at the beginning of each iteration.  
Second, our heuristic uses a better rebalancing method that provides more slack to voter $v$ for the same threshold $t$. Indeed, identity $\eqref{eq:slack}$ is at least as large as identity $\ref{eq:alt-slack}$, and usually larger. Hence, new candidates are granted higher scores and are added to the committee with higher supports.




\section{A lazy greedy algorithm}\label{s:lazymms}

In this section we prove Theorem~\ref{thm:2eps} and present $\lazy$ (Algorithm~\ref{alg:lazy}), a variant of $\MMS$ (Algorithm~\ref{alg:mms}) that is faster by a factor $\Theta(k)$ and offers virtually the same approximation guarantee.

\begin{algorithm}[htb]\label{alg:lazy}
\SetAlgoLined
\KwData{Approval graph $G=(N\cup C, E)$, vector $s$ of vote strengths, committee size $k$, threshold support $t\geq 0$.}
Initialize $A=\emptyset$, $w=0\in\R^E$, and $U=C$ \quad \emph{// $U$ is the set of ``uninspected'' candidates} \;
\While{$U\neq \emptyset$}{
	Find $c_{\max}\in \argmax_{c'\in U} \score(c')$ \quad \emph{// try the uninspected candidate with highest score} \;
	Remove $c_{\max}$ from $U$\;
	Compute a balanced edge weight vector $w'$ for $A+c_{\max}$\;
	\If(\quad \emph{// candidate is ``good enough'' to add}){$\supp_{w'}(A+c_{\max})\geq t$}{
		Update $A\leftarrow A+c_{\max}$ and $w\leftarrow w'$ \;
		\lIf{$|A|=k$} { \Return $(A,w)$}
	}
}
\Return a failure message\;
\caption{$\lazy$}
\end{algorithm}

Algorithm $\lazy$ is lazier than $\MMS$ in the sense that for each candidate it inspects, it decides on the spot whether to add it to the current partial solution, if the candidate is "good enough", or permanently reject it. In particular, each inspected candidate entails the computation of a single balanced weight vector, as opposed to $O(k)$ vectors in $\MMS$. 
For a threshold $t\geq 0$ given as input, the algorithm either succeeds and returns a full solution $(A,w)$ with $\supp_w(A)\geq t$, or it returns a failure message. 
The idea is then to run trials of $\lazy$ over several input thresholds $t$, performing binary search to converge to a value of $t$ where it flips from failure to success, and return the output of the last successful trial. 
In terms of runtime, our binary search requires only $O(\log (1/\eps))$ trials -- as we shall prove -- and in each trial each of the $O(|C|)$ iterations computes a balanced weight vector in time $\bal$, for an overall complexity of $O(\bal\cdot |C| \log (1/\eps))$. 
In each iteration, the highest score in $U$ can be found in time $O(|E|\log k)$ with a variant of Algorithm $\maxscore$ -- see Theorem~\ref{thm:runtimes} -- hence this complexity is dominated by that of computing a balanced vector.  

We start by proving that for low enough values of $t$, the algorithm is guaranteed to succeed. 
We highlight that in the following proof the order in which the candidate set $C$ is traversed is actually irrelevant. 

\begin{lemma}\label{lem:success}
If $(A^*, w^*)$ is an optimal solution to the given instance of maximin support, and $t^*=\supp_{w^*}(A^*)$, then for any input threshold $t$ with $0\leq t\leq t^*/2$, Algorithm $\lazy$ is guaranteed to succeed.
\end{lemma}

\begin{proof}
Assume by contradiction that for some input threshold $t\leq t^*/2$, $\lazy$ fails. Thus, after traversing the whole candidate set $C$, the algorithm ends up with a partial solution $(A,w)$ with $|A|<k$ and $\supp_w(A)\geq t$. By Lemma~\ref{lem:2sols}, there must be a candidate $c'\in A^*\setminus A$ and a feasible solution $(A+c', w')$ such that $\supp_{w'}(A+c')\geq t$. Notice as well that for any subset $S$ of $A+c'$, vector $w'$ provides a support of at least $t$, so any balanced weight vector for $S$ also provides a support of at least $t$. This implies that at whichever point the algorithm inspected candidate $c'$, it should have included it in the partial solution, which at that time was a subset of $A$. Hence, $c'$ should be contained in $A$, and we reach a contradiction.
\end{proof}

In the next lemma we establish the number of trials needed to achieve a solution whose value is within a factor $(2+\eps)$ from optimal for any $\eps>0$. 

\begin{lemma}\label{lem:lazybinary}
For any $\eps>0$, $O(\log(1/\eps))$ trials of $\lazy$ are sufficient to obtain a solution whose maximin support value is within a factor $(2+\eps)$ from optimal.
\end{lemma}

\begin{proof}
First, we need to compute a constant-factor estimation of the optimal objective value $t^*$. 
One way to do that is to use the $\balanced$ algorithm (Section~\ref{s:315}), which provides an approximation guarantee of $\alpha=3.15$ and runs in time $O(\bal\cdot k)$.%
\footnote{In fact, it can be checked that the $\balanced$ algorithm is equivalent to an execution of $\lazy$ with threshold $t=0$.} 
If $t$ is the objective value of its output, and we initialize the variables $t'\leftarrow t/2$, $t''\leftarrow \alpha\cdot t$, then we have the properties that $t'<t''$ and that $\lazy$ succeeds for threshold $t'$ and fails for threshold $t''$. We keep these properties as loop invariants as we perform binary search over Algorithm $\lazy$, in each iteration setting the new threshold value to the geometric mean of $t'$ and $t''$. This way, the ratio $t''/t'$ starts with a constant value $2 \alpha$, and is square-rooted in each iteration. 
By Lemma~\ref{lem:success}, to achieve a $(2+\eps)$-factor guarantee it suffices to find threshold values $t'<t''$ such that $\lazy$ succeeds for $t'$ and fails for $t''$ and whose ratio is bounded by $t''/t'\leq 1+\eps/2$, and return the output for $t'$. If it takes $T+1$ iterations for our binary search to bring this ratio below $(1+\eps/2)$, then $(2\alpha)^{1/2^T} > (1+\eps/2)$, so $T= O(\log (\eps^{-1} \log(\alpha))) = O(\log(\eps^{-1}))$. This completes the proof. 
\end{proof}

Finally, we prove that whenever Algorithm $\lazy$ succeeds and returns a full solution, this solution satisfies PJR. For this, we exploit the order in which we inspected the candidates. This completes the proof of Theorem~\ref{thm:2eps}. 

\begin{lemma}
For any input threshold $t$, at the end of each iteration of Algorithm $\lazy$ we have that if $(A,w)$ is the current partial balanced solution, then $\supp_w(A)\geq \max_{c'\in C\setminus A} \score_{(A,w)}(c')$. 
Therefore, if threshold $t$ is such that the algorithm succeeds and returns a full solution, this solution satisfies PJR.
\end{lemma}

\begin{proof}
The second statement immediately follows from the first one together with Lemma~\ref{lem:localopt}, hence we focus on proving the first statement. 
Fix an input threshold $t$ and some iteration of $\lazy$, and let $(A,w)$ be the partial solution at the end of it. 
We consider three cases. Case 1: If all candidates inspected so far have been added to the solution, then up to this point the construction coincides with Algorithm $\balanced$, and the claim follows by Lemma~\ref{lem:315localoptimality}. 
Case 2: Suppose the last iteration was the first to reject a candidate, and let $c'$ be this candidate. Then, $c'$ has the highest score in $C\setminus A$, and we claim that this score must be below threshold $t$, and hence below $\supp_w(A)$. Otherwise, by Lemma~\ref{lem:insert} we have that Algorithm $\ins(A,w,c',t)$ could find a weight vector that gives $A+c'$ a support above $t$, so a balanced weight vector would also give $A+c'$ a support above $t$ which contradicts the fact that $c'$ was rejected. 
Case 3: If a candidate was rejected in a previous iteration, then at the time of the first rejection we had that the highest score in $C\setminus A$ was below $t$, and this inequality must continue to hold true by Lemma~\ref{lem:2balanced}, because scores can only decrease in future iterations. This completes the proof.
\end{proof}


\section{Delayed proofs}\label{s:proofs}

\begin{proof}[Proof of Lemma~\ref{lem:lazybinary}]
$\lazy$ (Algorithm~\ref{alg:lazy}) clearly executes in time $O(B\cdot k)$. Hence, we focus on proving that the claimed number of iterations is enough, for a given parameter $\eps>0$.

Let $(A^*, w^*)$ be an optimal solution with support $t^*$. We start by finding bounds for $t^*$ via a pre-computation whose running time will be dominated by that of the main algorithm. For each candidate $c\in C$, we define its "potential support" as $psupp(c):=\sum_{n\in N_c} s_n$; clearly, it holds that $psupp(c)\geq supp_w^*(c)$ for any member $c\in A^*$. We find the potential supports of all candidates in $C$, and identify the $k$-th highest one, counting repetitions, which we denote by $psupp_k$. It follows that $psupp_k\geq supp_{w^*}(A^*)=t^*$. On the other hand, we also have that $t^* \geq psupp_k /k$, as we show now: Consider the solution $(A',w')$ where $A'$ contains the $k$ candidates in $C$ with highest potential support, and $w'$ is defined by $w'_{nc}:=s_n/|A\cap C_n| \geq s_n/k$ for each edge $nc$ with $c\in A\cap C_n$; it can then be checked that vector $w'$ is feasible, and $supp_w'(c)\geq psupp(c)/k$ for each $c\in A'$, so $t^*\geq supp_w'(A')\geq psupp_k/k$. Therefore, $psupp_k/k\leq t^*\leq psupp_k$.

Due to Lemma~\ref{lem:success}, to achieve a $(2+\eps)$-factor guarantee it suffices to find two target values $t'<t''$ such that $\lazy$ succeeds for $t'$ and fails for $t''$, and whose ratio is bounded by $t''/t'\leq 1+\eps/2$, and return the output for $t'$. We can find such target values via binary search, within the range $[t^*/2, t^*]\subseteq [psupp_k/(2k), psupp_k]$. If we initialize these variables to $t'=psupp_k/(2k)$ and $t''=psupp_k$, then their ratio starts at $2k$, and can be square-rooted at each iteration of the binary search by always testing their geometric mean. Hence, if it takes $r+1$ iterations to bring this ratio below $(1+\eps/2)$, then $(2k)^{1/2^r} > (1+\eps/2)$, so $r=O(\log(\eps^{-1}\log k))$. This completes the proof. 
\end{proof}

\begin{proof}[Proof of Lemma~\ref{lem:Lebesgue}]
Recall that for any set $A\subseteq \mathbb{R}$, the indicator function $1_A:\mathbb{R}\rightarrow \mathbb{R}$ is defined as $1_A(t)=1$ if $t\in A$, and $0$ otherwise. For any $i\in I$, we can write
$$\alpha_i f(x_i) = \alpha_i \int_{0}^{f(x_i)} dt = \alpha_i\int_0^{\lim_{x\rightarrow \infty} f(x)} 1_{(-\infty, f(x_i)]}(t)dt,$$
and thus
\begin{align*}
    \sum_{i\in I} \alpha_i f(x_i) = \int_0^{\lim_{x\rightarrow \infty} f(x)} \Big(\sum_{i\in I} \alpha_i 1_{(-\infty, f(x_i)]}(t)\Big)dt = \int_0^{\lim_{x\rightarrow \infty} f(x)} \Big(\sum_{i\in I: \ f(x_i)\geq t} \alpha_i \Big)dt.
\end{align*}
This is a Lebesgue integral over the measure with weights $\alpha_i$. Now, conditions on function $f(x)$ are sufficient for its inverse $f^{-1}(t)$ to exist, with $f^{-1}(0)=\chi$. Substituting with the new variable $x=f^{-1}(t)$ on the formula above, where $t=f(x)$ and $dt=f'(x)dx$, we finally obtain
$$\sum_{i\in I} \alpha_i f(x_i) =\int_{\chi}^{\infty} \Big( \sum_{i\in I: \ x_i\geq x} \alpha_i \Big)(f'(x)dx).$$
\end{proof}

\section{Extra bits (to be removed or completed before publication)}

\begin{lemma}\label{lem:list}
Consider a list $L$ of $k$ real numbers, all within the range $[0,t]$ for some threshold $t>0$, and consider an iterative process that in each iteration: 
1) replaces the smallest entry in $L$ with the number $t$, and then 2) alters the entries in $L$ in such a way that neither the minimum nor the average value decreases.
For any $\eps>0$, we have that after at most $k\cdot (1+\ln (1/\eps))$ iterations the smallest entry in $L$ is at least $(1-\eps)\cdot t$.
\end{lemma}

\begin{proof}
Fix a constant $\eps>0$. Let $a^i$ and $m^i$ be respectively the average and the minimum value of list $L$ at the end of the $i$-th iteration, where $a^0$ and $m^0$ correspond to these values for the initial list $L$. 
We make two claims: first, that for $i\geq k\cdot \ln(1/\eps)$ we have that $a^i \geq (1-\eps) \cdot t$; 
second, that $m^{i+k-1}\geq a^i$ for any $i\geq 0$. Notice that the lemma easily follows from these two claims.

We start with the first claim. At each iteration $i\geq 1$, the first operation increases the average by an amount $(t - m^{i-1})/k$, and the second operation does not decrease the average. Hence, $a^i - a^{i-1}\geq (t-m^{i-1})/k \geq (t-a^{i-1})/k$. We thus obtain the following recursive inequality between $a^{i}$ and $a^{i-1}$:
\begin{align*}
a^i &\geq \frac{t}{k} + \Big( 1-\frac{1}{k} \Big) \cdot a^{i-1}\\
  &\geq \Big[ 1-\Big( 1-\frac{1}{k} \Big)^i \Big]\cdot t + \Big(1 - \frac{1}{k}\Big)^i\cdot a^0 & \text{(by induction on $i$)} \\
	&\geq \Big[ 1-\Big( 1-\frac{1}{k} \Big)^i \Big]\cdot t 
	 > \big( 1 - e^{-i/k} \big) \cdot t.
\end{align*}
Finally, it can be checked that if $i\geq k\cdot \ln (1/\eps)$, then $a^i > (1 - e^{i/k})\cdot t \geq (1-\eps)\cdot t$. 

We continue with the second claim. Fix an iteration $i\geq 0$. Our previous remark on the increase of the average value from one iteration to the next gives us 
$$a^{i+k}\geq a^{i+k-1} + \frac{t - m^{i+ k -1}}{k} \geq a^{i+ k - 2} + \frac{2t - m^{i+k-1} - m^{i+k-2}}{k} \geq \cdots \geq a^i + t - \frac{m^{i+k-1}+\cdots + m^i}{k}.$$ 
Notice now that neither of the two operations decreases the minimum value in the list, so the sequence of minimum values must be non-decreasing. 
In particular, we obtain that $(m^i + \cdots +m^{i+k-1})/k \leq m^{i+k-1}$, and the inequality above yields $a^{i+k}\geq a^i + t - m^{i+k-1}$.  
Finally, solving for $m^{i+k-1}$ we obtain that $m^{i+k-1}\geq a^i + t - a^{i+k}\geq a^i$, where we use the observation that the average must be upper bounded by $t$ in all iterations. This completes the proof.
\end{proof}

\begin{algorithm}[htb]\label{alg:unified}
\SetAlgoLined
\KwData{Balanced partial solution $(A,w)$ with $|A|\leq k$, error parameter $\eps>0$, Boolean $\balancing$.}

Let $\hat{t}\leftarrow \sum_{n\in N} s_n / |A|$\;
\While{True}{
  Let $(c_{\max},t_{\max})\leftarrow \maxscore(\bar{A},\bar{w})$\;
	\If{$|A|==k$}{
	  Find tuple $(c_{\min}, t_{\min})$ where $c_{\min}\in A$ and $t_{\min}=supp_w(c_{\min})=supp_w(A)$\; 
		\lIf {($t_{\min} \geq (1-\eps)\cdot t_{\max}$)} { \Return $(A,w)$ }
		Set $A\leftarrow A-c_{\min}$;%, and $w_{nc_{\min}}\leftarrow 0$ for each $n\in N_{c_{\min}}$;
	}
	\eIf{$\balancing$}{
	  Set $A\leftarrow A+c_{\max}$ and rebalance $w$ (replace $w$ with a balanced vector for $A$)\;
	}{
	  $(A,w)\leftarrow \ins(A,w,t_{\max})$\;
	}
}
\caption{Unified algorithm}
\end{algorithm}

\begin{lemma}
If the unified algorithm is given as input an arbitrary balanced partial solution $(A,w)$, an error parameter $\eps>0$, and a Boolean $\balancing$, it executes at most $k\cdot(1+\ln(1/\eps))$ iterations. 
\end{lemma}
\begin{proof}
Consider the list of numbers $\{\min\}$
\end{proof}



\end{document}
