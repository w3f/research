\begin{abstract}

\emph{Polkadot} is a decentralized blockchain platform to be launched in 2020. It will implement \emph{nominated proof-of-stake} (NPoS), a proof-of-stake-based mechanism where $k$ nodes are periodically selected by the network as \emph{validators} to participate in the consensus protocol, according to the preferences expressed by token holders who take the role of \emph{nominators}. 
This setup leads to an approval-based multiwinner election problem, where each nominator submits a list of trusted candidates, and has a vote strength proportional to their stake. 
A solution consists of a committee of $k$ elected validators, together with a fractional distribution of each nominator's vote among them. 
We consider two objectives for the election rule, both recently studied in the literature of social choice. The first one is ensuring the property of \emph{proportional justified representation} (PJR). The second objective, called \emph{maximin support}, is to maximize the minimum amount of vote support assigned to any elected validator. We argue that the former objective aligns with the notion of decentralization among validators, while the latter aligns with the security level of the system.

We prove that the maximin support objective is constant-factor approximable, as we provide several approximation algorithms for it, as well as present a new inapproximability result.  
Furthermore, we present an efficient post-computation which, when paired with an approximation algorithm for maximin support, returns a new solution that a) preserves the approximation guarantee, b) satisfies the PJR property, and c) can be efficiently verified to satisfy PJR by an untrusting third party. 
Besides being of independent theoretical interest, our results enable the network to run an efficient validator election protocol that simultaneously achieves the PJR property and a constant-factor approximation for maximin support, thus offering formal guarantees on decentralization and security. 
\end{abstract}
