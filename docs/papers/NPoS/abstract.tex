\begin{abstract}

The classical property of proportional representation in approval-based committee elections has appeared in the social choice literature for over a century, and is typically understood as avoiding the underrepresentation of minorities. In this work we pursue the opposite goal of \emph{avoiding overrepresentation} of any minority, which leads us to an optimization objective known as \emph{maximin support}, closely related to the axiom of \emph{proportional justified representation} (PJR). We provide a new inapproximability result for this objective, and propose a new election rule inspired in Phragm\'{e|}n's methods that achieves a) a constant-factor approximation guarantee for the objective, and b) the PJR property. Furthermore, a structural property allows one to quickly \emph{verify} that the winning committee satisfies the two aforementioned properties, even if the algorithm was executed by an untrusted party who only communicates the output. Finally, we present an efficient post-computation which, when paired with any approximation algorithm for maximin support, returns a new solution that a) preserves the approximation guarantee, b) satisfies the PJR property, and c) can be efficiently verified to satisfy PJR.

Our work is motivated by an application on blockchains based on \emph{nominated proof-of-stake} (NPoS), where the community must elect a committee of validators to participate in its consensus protocol, and where fighting overrepresentation protects the system against attacks by an adversarial minority. Our election rule enables a validator election protocol with formal and verifiable guarantees on security and proportionality. We propose a specific protocol which can be successfully implemented in spite of the stringent time constraints of a blockchain architecture, and which will be the basis for an implementation in the \emph{Polkadot} network, launched in 2020.

\end{abstract}