
\section{Delayed proofs}\label{s:proofs}

\begin{lemma}\label{lem:path}
Let $w\in \R^E$ be a feasible weight vector for a given instance, let $c,c'\in C$ be two candidates with $supp_w(c)>supp_w(c')$, and suppose there is a path $p\in\mathbb{R}^E$ that carries some non-zero flow from $c$ to $c'$, with zero excess elsewhere. If $w+p$ is non-negative and feasible, then $w$ is not balanced for any committee $A$ that contains $c'$.
\end{lemma}

\begin{proof}
Fix a committee $A\subseteq C$ that contains $c'$. If $c$ is not in $A$, then $w+p$ provides a greater sum of member supports over $A$ than $w$, so the latter is not balanced as it does not maximize this sum. Now suppose both $c$ and $c'$ are in $A$. Let $\lambda>0$ be the flow value carried by $p$, let $\eps:=\min\{\lambda, (supp_w(c) - supp_w(c'))/2\}>0$, and let $p'$ be the scalar multiple of $p$ whose flow value is $\eps$. By an application of Lemma~\ref{lem:subflow} over $w$ and $w':=w+p$, and the fact that $p'$ is a sub-flow of $p=w'-w$, we have that $w+p'$ is non-negative and feasible. Moreover, both $w$ and $w+p'$ clearly provide the same sum of member supports over $A$. Finally, if we compare their sums of member supports squared, we have that
\begin{align*}
\sum_{d\in A} supp_w^2(d)-\sum_{d\in A} supp^2_{w+p'}(d) &=supp_w^2(c) +supp_w^2(c')-supp^2_{w+p'}(c) -supp^2_{w+p'}(c')\\
&= supp_w^2(c) +supp_w^2(c') - (supp_w(c) - \eps)^2 - (supp_w(c) + \eps)^2 \\
&= 2\eps\cdot (supp_w(c) - supp_w(c') - \eps) \geq 2\eps\cdot(2\eps - \eps)=2\eps^2>0.
\end{align*}

Therefore, $w$ is not balanced for $A$, as it does not minimize the sum of member supports squared.  
\end{proof}

\begin{proof}[Proof of Lemma~\ref{lem:balanced}]
Fix a balanced partial solution $(A,w)$. The first statement says that function 
$F_r(w'):=\min_{A'\subseteq A, |A'|=r} \sum_{c\in A'} supp_{w'}(c)$ 
is maximized by vector $w$ over all feasible vectors $w'\in\R^E$, for all $1\leq r\leq |A|$. 
Assume by contradiction that there is a parameter $r$ and feasible $w'$ such that $F_r(w')>F_r(w)$. 
We can also assume without loss of generality that 
\begin{enumerate}
    \item $\sum_{c\in A} supp_{w'}(c)=\sum_{c\in A} supp_{w}(c)$, i.e. $w'$ also maximizes the sum of member supports, as does $w$ by definition of balancedness; and  
    \item we enumerate the candidates in $A=\{c_1, \cdots, c_{|A|}\}$ so that whenever $i<j$ we have that $supp_w(c_i)\leq supp_w(c_j)$, and in case of a tie, $supp_w(c_i)= supp_w(c_j)$, we have that $supp_{w'}(c_i)\leq supp_{w'}(c_j)$. 
\end{enumerate}

With a candidate enumeration as above, it follows that $F_r(w)=\sum_{i=1}^r supp_w(c_i)$, while for vector $w'$ we have the inequality $F_r(w')\leq \sum_{i=1}^r supp_{w'}(c_i)$. 
Thus, by our assumption by contradiction, 
$$\sum_{i=1}^r supp_{w'}(c_i) \geq F_r(w') > F_r(w) = \sum_{i=1}^r supp_{w}(c_i), \quad \text{and}$$
\begin{align*}
    \sum_{i=r+1}^{|A|} supp_{w'}(c_i) &= \sum_{i=1}^{|A|} supp_{w'}(c_i) - \sum_{i=1}^{r} supp_{w'}(c_i) \\
    & = \sum_{i=1}^{|A|} supp_{w}(c_i) - \sum_{i=1}^{r} supp_{w'}(c_i) \\
    & < \sum_{i=1}^{|A|} supp_{w}(c_i) - \sum_{i=1}^{r} supp_{w}(c_i) 
    = \sum_{i=r+1}^{|A|} supp_{w}(c_i). \\
\end{align*}

Now define the edge vector $f:=w'-w\in\mathbb{R}^E$ and consider it as a flow over the network $(N\cup A, E)$. 
We have that $f$ has a zero net demand over set $N$, by our first assumption, and the previous two inequalities show that $f$ has a positive net demand over set $\{c_1, \cdots, c_r\}$ and a positive net excess over set $\{c_{r+1}, \cdots, c_{|A|}\}$. Thus, by the flow decomposition theorem, $f$ can be decomposed into circulations and simple paths, where every path starts in a vertex with positive demand and ends in a vertex with positive excess, and there must be a simple path $p$ carrying non-zero flow from $c_j$ to $c_i$ for some $1\leq i\leq r<j\leq |A|$. 
Moreover, by our second assumption, it must be the case that $supp_w(c_i)<supp_w(c_j)$, because in case of a tie we would have that $supp_{w'}(c_i)<supp_{w'}(c_j)$ and so $f$ would have a net excess on $c_i$ and net demand on $c_j$, and the $c_i$-to-$c_j$ path would not exist in the flow decomposition. 
But now, by Lemma~\ref{lem:subflow}, vector $w+p$ is non-negative and feasible, and by Lemma~\ref{lem:path}, $w$ is not balanced for $A$, and we reach a contradiction. 

The second statement follows directly from the fact that $w$ maximizes the sum of member supports, and thus all of the vote strength of all represented voters (i.e. all voters in $\cup_{c\in A} N_c$) must be directed to members of $A$. 
We move on to the third statement. 
Assume by contradiction that there is a voter $n\in N$ and two candidates $c, c'\in A\cap C_n$ such that $w_{nc}>0$ and $supp_w(c)>supp_w(c')$. 
Let $\eps:=\min\{w_{nc}, (supp_w(c)-supp_w(c'))/2\}>0$, and define a path $p\in\mathbb{R}^E$ carrying flow $\eps$ from $c$ to $c'$ via $n$, i.e~$p_{nc'}=-p_{nc}=\eps$ and $p_e=0$ for every other edge $e\in E$. 
It can be checked that $w+p$ is non-negative and feasible, so by Lemma~\ref{lem:path} $w$ is not balanced for $A$, which is a contradiction. 

Finally, we prove that if a feasible weight vector satisfies conditions 2 and 3, then it is balanced for $A$. 
In fact, we claim that all such weight vectors provide the same list of member supports $(supp_w(c))_{c\in A}$, and hence are all balanced. 
Let $w, w'\in\R^E$ be two such weight vectors. It easily follows from feasibility (inequality~\ref{eq:feasible}) and condition 2 that both provide the same sum of member supports, namely $\sum_{c\in A} supp_w(c) = \sum_{c\in A} supp_{w'}(c) =\sum_{n\in \cup_{c\in A} N_c} s_n$. 
Now, assume by contradiction and wlog that there is a candidate $c\in A$ for which $supp_{w}(c)>supp_{w'}(c)$, and consider the flow vector $f:=w'-w$ over the network induced by $N\cup A$. Clearly, all nodes in $N$ have zero excess, while $c$ has positive excess.
By the flow decomposition theorem, $f$ can be decomposed into circulations and single paths, where every path starts in a node with net excess and ends in a node with net demand; so, there must be non-zero path $p$ that starts in $c$ and ends in a candidate $c'\in A$ with net demand. 
Now, path $p$ alternates between candidates and voters, and there must be three consecutive nodes $c_1, n, c_2$ in it, with $c_1, c_2\in A\cap C_{n}$, such that $c_1$ has positive excess and $c_2$ does not, i.e. $supp_{w'}(c_1)<supp_w(c_2)$ and $supp_{w'}(c_2)\geq supp_w(c_2)$, which in turn implies that either $supp_{w'}(c_1)<supp_{w'}(c_2)$ or $supp_w(c_2)<supp_{w}(c_1)$ (or both). 
If $supp_{w'}(c_1)<supp_{w'}(c_2)$, we reach a contradiction to the fact that $w'$ satisfies condition 3 and that $w'_{nc_2}$ must be positive since the flow in $p$ moves from $n$ to $c_2$; 
and similarly if $supp_w(c_2)<supp_{w}(c_1)$ we reach a contradiction to the fact that $w$ satisfies condition 3 and that $w_{nc_1}$ must be positive since the flow in $p$ moves from $c_1$ to $n$. This completes the proof of the lemma.
\end{proof}



\begin{proof}[Proof of Lemma~\ref{lem:equivalence}]
Let $A'\subseteq A$ be the non-empty subset that minimizes the expression $\frac{1}{|A'|} \sum_{n\in \cup_{c\in A'} N_c} s_n$. Hence, 
\begin{align*}
    supp_w(A) &\leq supp_w(A') \leq \frac{1}{|A'|} \sum_{c\in A'} supp_w(c) &\text{(by an averaging argument)}\\
    & = \frac{1}{|A'|} \sum_{c\in A'} \sum_{n\in N_c} w_{nc} 
     \leq \frac{1}{|A'|}  \sum_{n\in \cup_{c\in A'} N_c} \quad \sum_{c\in C_n} w_{nc} \\
    & \leq \frac{1}{|A'|} \sum_{n\in \cup_{c\in A'} N_c} s_n &\text{(by \ref{eq:feasible})}.
\end{align*}

This proves one inequality. To prove the opposite inequality, we use the fact that $(A,w)$ is a balanced solution. 
Let $A_{\min}\subseteq A$ be the set of committee members with least support, i.e.~those $c\in A$ with $supp_w(c)=supp_w(A)$. Then,
\begin{align*}
    supp_w(A_{\min}) &= \frac{1}{|A_{\min}|} \sum_{c\in A_{\min}} supp_w(c) \\
    &= \frac{1}{|A_{\min}|} \sum_{c\in A_{\min}} \sum_{n\in N_c} w_{nc} 
    = \frac{1}{|A_{\min}|} \sum_{n\in \cup_{c\in A_{\min}} N_c} \quad \sum_{c\in C_n\cap A_{\min}} w_{nc} \\
    &= \frac{1}{|A_{\min}|} \sum_{n\in \cup_{c\in A_{\min}} N_c} \Big( \sum_{c\in C_n\cap A} w_{nc} 
		- \sum_{c\in C_n \cap (A\setminus A_{\min})} w_{nc}\Big)\\
		&= \frac{1}{|A_{\min}|}\sum_{n\in \cup_{c\in A_{\min}} N_c} s_n \geq \min_{\emptyset\neq A'\subseteq A} \frac{1}{|A'|}\sum_{n\in \cup_{c\in A'}N_c} s_n,
\end{align*}

where we used the fact that for each voter $n\in \cup_{c\in A_{\min}} N_c$, the term $\sum_{c\in C_n\cap A} w_{nc}$ equals $s_n$ by point 2 of Lemma~\ref{lem:balanced}, and the term $\sum_{c\in C_n \cap (A\setminus A_{\min})} w_{nc}$ vanishes by point 3 of Lemma~\ref{lem:balanced} and the definition of set $A_{\min}$. 
This proves the second inequality and completes the proof.
\end{proof}


\begin{proof}[Proof of Lemma~\ref{lem:subflow}]
We prove the claim only for $w+f'$, as the proof for $w'-f'$ is similar. 
For each edge $e\in E$, we have that $(w+f')_e$ is a value between $w_e$ and $(w+f)_e=w'_e$. As both of these values are non-negative, the same holds for $(w+f')_e$. 
Notice now from inequality \eqref{eq:feasible} that proving feasibility corresponds to proving that the excess $(w+f')(n)$ is at most $s_n$ for each voter $n\in N$. We have 
$$(w+f')(n) = \sum_{c\in C_n} (w+f')_{nc}= \sum_{c\in C_n} w_{nc} + \sum_{c\in C_n} f_{nc}' = w(n) + f'(n). $$
If excess $f'(n)$ is non-positive, then $(w+f')(n)\leq w(n) \leq s_n$, since $w$ is feasible. 
Otherwise, $f'(n)\leq f(n)$, and thus $(w+f')(n)\leq w(n) + f(n) = (w+f)(n) = w'(n) \leq s_n$, since $w'$ is feasible. This completes the proof.
\end{proof}



\begin{proof}[Proof of Lemma~\ref{lem:2balanced}]
The second statement follows directly from the first one and the definitions of slack, pre-score and score in Section~\ref{s:heuristic}. Hence we focus on the first statement, i.e.~that $supp_w(c)\geq supp_{w'}(c)$ for each candidate $c\in A$.

Consider vector $f:=w'-w\in\mathbb{R}^E$ as a flow vector over the network induced by $N\cup A'$. We need to prove that there is no candidate in $A$ with positive demand relative to $f$. Assume by contradiction that there is such a candidate $c'\in A$. By the flow decomposition theorem, $f$ can be decomposed into circulations and simple paths, where each path starts in a vertex with positive excess and ends in a vertex with positive demand. By our assumption, there must be a path $p$ carrying non-zero flow to $c'$.  
Now, where does path $p$ start? It cannot start in $N$ nor in $A'\setminus A$, as otherwise weight vector $w+p$ is feasible by Lemma~\ref{lem:subflow}, and has a greater sum of candidate supports over $A$ than $w$, which contradicts the fact that $w$ is balanced for $A$ and thus maximizes this sum. Therefore, it must start in another candidate $c$ in $A$ with positive excess. 
The fact that $f$ has positive excess in $c$ and positive demand in $c'$ implies that
$$supp_{w}(c') - supp_{w'}(c') = f(c')< 0 < f(c) = supp_{w}(c) - supp_{w'}(c),$$
which in turn implies that either $supp_w(c) > supp_w(c')$ or $supp_{w'}(c') > supp_{w'}(c)$, or both. 
If $supp_w(c) > supp_w(c')$, then Lemma~\ref{lem:path} implies that $w$ is not balanced for $A$. 
Similarly, if $supp_{w'}(c') > supp_{w'}(c)$, notice that $w'-p$ is non-negative and balanced by Lemma~\ref{lem:subflow}, so again Lemma~\ref{lem:path} applied over vector $w'$ and path $-p$ implies that $w'$ is not balanced for $A'$. Hence, in either case we reach a contradiction.
\end{proof}

\begin{proof}[Proof of Lemma~\ref{lem:Lebesgue}]
Recall that for any set $A\subseteq \mathbb{R}$, the indicator function $1_A:\mathbb{R}\rightarrow \mathbb{R}$ is defined as $1_A(t)=1$ if $t\in A$, and $0$ otherwise. For any $i\in I$, we can write
$$\alpha_i f(x_i) = \alpha_i \int_{0}^{f(x_i)} dt = \alpha_i\int_0^{\lim_{x\rightarrow \infty} f(x)} 1_{(-\infty, f(x_i)]}(t)dt,$$
and thus
\begin{align*}
    \sum_{i\in I} \alpha_i f(x_i) = \int_0^{\lim_{x\rightarrow \infty} f(x)} \Big(\sum_{i\in I} \alpha_i 1_{(-\infty, f(x_i)]}(t)\Big)dt = \int_0^{\lim_{x\rightarrow \infty} f(x)} \Big(\sum_{i\in I: \ f(x_i)\geq t} \alpha_i \Big)dt.
\end{align*}
This is a Lebesgue integral over the measure with weights $\alpha_i$. Now, conditions on function $f(x)$ are sufficient for its inverse $f^{-1}(t)$ to exist, with $f^{-1}(0)=\chi$. Substituting with the new variable $x=f^{-1}(t)$ on the formula above, where $t=f(x)$ and $dt=f'(x)dx$, we finally obtain
$$\sum_{i\in I} \alpha_i f(x_i) =\int_{\chi}^{\infty} \Big( \sum_{i\in I: \ x_i\geq x} \alpha_i \Big)(f'(x)dx).$$
\end{proof}

