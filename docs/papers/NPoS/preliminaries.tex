\section{Preliminaries}\label{s:prel}

Throughout the paper we consider the following approval-based multiwinner election instance. 
We are given a bipartite approval graph $G=(N\cup C, E)$ where $N$ is a set of voters and $C$ is a set of candidates. 
We are additionally given a vector $s\in\R^N$ of vote strengths, where $s_n$ is the strength of $n$'s vote, and a target number $k$ of candidates to elect, where $0< k<|C|$.
For each voter $n\in N$, $C_n:=\{c\in C: \ nc\in E\}$ represents her approval ballot, i.e.~the subset of candidates that $n$ approves of, and for each candidate $c\in C$ we denote by $N_c:=\{n\in N: \ nc\in E\}$ the set of voters approving $c$, where $nc$ is shorthand for edge $\{n,c\}$. 
To avoid trivialities, we assume that graph $G$ in the input has no isolated vertices. 
For any $c\in C\setminus A$, we write $A+c$ and $A-c$ as shorthands for $A\cup\{c\}$ and $A\setminus \{c\}$ respectively. 

\paragraph{Proportional justified representation (PJR).} 
The PJR property was introduced in~\cite{sanchez2017proportional} for voters with unit vote strength. We present its natural generalization to arbitrary vote strengths. A committee $A\subseteq C$ of $k$ members satisfies PJR if there is no group $N'\subseteq N$ of voters and integer $0<r\leq k$ such that:
\begin{itemize}
\item[a)] $\sum_{n\in N'} s_n \geq \frac{r}{k} \sum_{n\in N}s_n$,
\item[b)] $|\cap_{n\in N'} C_n|\geq r$, and
\item[c)] $|A\cap (\cup_{n\in N'} C_n)|<r$.
\end{itemize}

In words, if there is a group $N'$ of voters with at least $r$ commonly approved candidates, and enough aggregate vote strength to provide each of these candidates with a vote support of value $\hat{t}:=\sum_{n\in N} s_n / k$, then this group has a justified right to be represented by at least $r$ members in committee $A$, though not necessarily commonly approved. As a sanity check, notice that $\hat{t}$ is an upper bound on the average vote support that voter set $N$ can possibly provide to any committee of $k$ members. 


\paragraph{The maximin support objective.} 
For the given instance, we consider a solution consisting of a tuple $(A,w)$, where $A\subseteq C$ is a committee of $k$ elected candidates, and $w\in\R^E$ is a vector of non-negative edge weights that represents a fractional distribution of each nominator's vote among her approved candidates. Vector $w$ is considered \emph{feasible} if  % 
%
\begin{equation}
    \sum_{c\in C_n} w_{nc}\leq s_n \quad \text{ for each voter } n\in N.%
    \footnote{Intuitively, a feasible solution $(A,w)$ should also observe $w_{nc}=0$ for each edge $nc$ with $c\not\in A$. 
However, as this constraint can always be enforced in post-computation, we ignore it so that the feasibility of a vector $w$ is independent of any committee.} 
    \label{eq:feasible}
\end{equation}

In our analyses, we will also consider \emph{partial} committees, with $|A|<k$. If $|A|=k$, we call it \emph{full}. 
All solutions $(A,w)$ in this paper are assumed to be feasible and full unless stated otherwise. 
Given a (possibly partial, unfeasible) solution $(A,w)$, we define the \emph{support} over the committee members as 
\begin{equation}
supp_w(c):=\sum_{n\in N_c} w_{nc} \quad \text{for each $c\in A, \quad$ and } \quad supp_w(A):=\min_{c\in A} supp_w(c), \label{eq:support}
\end{equation}
where we use the convention that $supp_w(\emptyset)=\infty$ for any weight vector $w\in\R^E$. 
The maximin support objective, introduced in~\cite{sanchez2016maximin}, seeks to maximize $supp_w(A)$ over all feasible full solutions $(A,w)$. 

\paragraph{Balanced solutions.}
For a fixed committee $A$, a feasible weight vector $w\in\R^E$ that maximizes $supp_w(A)$ can be found efficiently. In this paper we seek additional desirable properties for this vector which can still be achieved efficiently. We say that a feasible $w\in\R^E$ is \emph{balanced for $A$}, or that $(A,w)$ is a balanced solution, if
\begin{enumerate}
    \item it maximizes the sum of member supports, $\sum_{c\in A} supp_w(c)$, over all feasible weight vectors, and 
    \item it minimizes the sum of supports squared, $\sum_{c\in A} (supp_w(c))^2$, over all vectors that observe the point above. 
\end{enumerate}

In words, a balanced weight vector maximizes the sum of supports and, subordinate to that, minimizes their variance. 
In the next lemma we list some properties of balanced weight vectors that we exploit in our analyses.

\begin{lemma}\label{lem:balanced}
Let $w\in\R^E$ be a balanced weight vector for a given (possibly partial) committee $A$. Then,
\begin{enumerate}
    \item vector $w$ maximizes $supp_{w'}(A)$ over all feasible weight vectors $w'\in\R^E$; 
    \item more generally, for each $1\leq r\leq |A|$, vector $w$ simultaneously maximizes 
    $$\min_{A'\subseteq A, |A'|=r} \quad \sum_{c\in A'} supp_{w'}(c)$$
over all feasible weight vectors $w'\in\R^E$; and
    \item if there is a voter $n\in N$ and candidates $c\in C_n$, $c'\in C_n\cap A$ with $supp_w(c)>supp_w(c')$, then $w_{nc}=0$.
\end{enumerate}
\end{lemma}

Notice that the quantity defined on the second point defines a lower bound on the cost for an adversary to gain control of $r$ validators in Polkadot, so maximizing these objectives for all thresholds $r$ aligns with our security objective as it makes any attack as costly as possible. 
The third point is a consequence of having the candidate supports as evenly distributed as possible within $A$: if candidate $c$ has a higher support than another candidate $c'$, then none of $n$'s vote should go to $c$, as all of it must have been assigned to $c'$ or to other members with lower support. 

We delay the proof of Lemma~\ref{lem:balanced} to Appendix~\ref{s:balanced}. 
In that appendix we also present new algorithms for computing a balanced weight vector for a given committee $A$. 
In particular, we prove that one can be found in time $O(|E|\cdot k + k^3)$ using techniques for parametric flow problems, which to the best of our knowledge is the current fastest algorithm in the literature even for the simpler problem of just maximizing $supp_w(A)$.
In the remainder of the paper, we denote by $B$ the time complexity of finding a balanced weight vector, which will depend on the precise algorithm used.

\paragraph{Equivalence of objectives.} 
We now prove that our security objective~\eqref{eq:security}, developed in the introduction, is indeed equivalent to the maximin support objective. For this, we use the fact that in the latter objective we can reduce the search space to only balanced solutions, without loss of generality, in view of point 1 of Lemma~\ref{lem:balanced}

\begin{lemma} \label{lem:maximin-support-eqiv} If $(A,w)$ is a balanced solution, then
$$supp_w(A) = \min_{\emptyset\neq A' \subseteq A} \frac{1}{|A'|} \sum_{n\in \cup_{c\in A'} N_c} s_n.$$
Consequently, maximin support, the problem of maximizing the left-hand side over all balanced full solutions $(A,w)$, is equivalent to the problem of maximizing the right-hand side over all full committees $A$. 
Furthermore, this equivalence preserves approximations, as any balanced solution $(A,w)$ provides the same objective value to both problems.
\end{lemma}

\begin{proof}
Let $A'\subseteq A$ be the non-empty subset that minimizes the expression $\frac{1}{|A'|} \sum_{n\in \cup_{c\in A'} N_c} s_n$. Hence, 
\begin{align*}
    supp_w(A) &\leq supp_w(A') \leq \frac{1}{|A'|} \sum_{c\in A'} supp_w(c) &\text{(by an averaging argument)}\\
    & = \frac{1}{|A'|} \sum_{c\in A'} \sum_{n\in N_c} w_{nc} 
     \leq \frac{1}{|A'|}  \sum_{n\in \cup_{c\in A'} N_c} \quad \sum_{c\in C_n} w_{nc} \\
    & \leq \frac{1}{|A'|} \sum_{n\in \cup_{c\in A'} N_c} s_n &\text{(by \ref{eq:feasible})}.
\end{align*}

This proves one inequality. To prove the opposite inequality, we use the fact that $(A,w)$ is a balanced solution. 
Let $A_{\min}\subseteq A$ be the set of committee members with least support, i.e.~those $c\in A$ with $supp_w(c)=supp_w(A)$. Then,
\begin{align*}
    supp_w(A_{\min}) &= \frac{1}{|A_{\min}|} \sum_{c\in A_{\min}} supp_w(c) \\
    &= \frac{1}{|A_{\min}|} \sum_{c\in A_{\min}} \sum_{n\in N_c} w_{nc} 
    = \frac{1}{|A_{\min}|} \sum_{n\in \cup_{c\in A_{\min}} N_c} \quad \sum_{c\in C_n\cap A_{\min}} w_{nc} \\
    &= \frac{1}{|A_{\min}|} \sum_{n\in \cup_{c\in A_{\min}} N_c} \Big( \sum_{c\in C_n} w_{nc} - \sum_{c\in C_n \setminus A_{\min}} w_{nc}\Big).
\end{align*}
We now make two claims, one for each of the terms inside the parenthesis. The first claim is that $\sum_{c\in C_n} w_{nc}=s_n$, i.e. inequality~\eqref{eq:feasible} defining feasibility must be tight, for each nominator $n$ in $\cup_{c\in A_{\min}} N_c$, or more generally each nominator in $\cup_{c\in A} N_c$, because balanced weight vector $w$ by definition maximizes the sum of supports over $A$. The second claim is that the second term is zero. Indeed, if there is a nominator $n$ in $\cup_{c\in A_{\min}} N_c$ and a candidate $c$ in $C_n\setminus A_{\min}$, then there must be at least one member $c'$ in $C_n\cap A'$, with $supp_w(c')<supp_w(c)$ by definition of set $A_{\min}$, and hence $w_{nc}=0$ by point 3 of Lemma~\ref{lem:balanced}. 
We obtain that 
$$supp_w(A_{\min}) = \frac{1}{|A_{\min}|}\sum_{n\in \cap_{c\in A_{\min}} N_c} s_n \geq \min_{\emptyset\neq A'\subseteq A} \frac{1}{|A'|}\sum_{n\in \cup_{c\in A'}N_c} s_n.$$
This proves the second inequality and completes the proof.
\end{proof}

\begin{remark}
In all algorithms analyzed, we assume that all numerical operations take constant time.
\end{remark}
