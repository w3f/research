\section{Preliminaries}\label{s:prel}

Throughout the paper we consider the following approval-based multiwinner election instance. 
We are given a bipartite approval graph $G=(N\cup C, E)$ where $N$ is a set of voters and $C$ is a set of candidates. 
We are additionally given a vector $s\in\R^N$ of vote strengths, where $s_n$ is the strength of $n$'s vote, and a target number $k$ of candidates to elect, where $0< k<|C|$.
For each voter $n\in N$, $C_n:=\{c\in C: \ nc\in E\}$ represents her approval ballot, i.e.~the subset of candidates that $n$ approves of, and for each candidate $c\in C$ we denote by $N_c:=\{n\in N: \ nc\in E\}$ the set of voters approving $c$, where $nc$ is shorthand for edge $\{n,c\}$. 
To avoid trivialities, we assume that graph $G$ in the input has no isolated vertices. 
For any $c\in C\setminus A$, we write $A+c$ and $A-c$ as shorthands for $A\cup\{c\}$ and $A\setminus \{c\}$ respectively. 

\emph{Proportional justified representation (PJR).} 
The PJR property was introduced in~\cite{sanchez2017proportional} for voters with unit vote strength. We present its natural generalization to arbitrary vote strengths. A committee $A\subseteq C$ of $k$ members satisfies PJR if there is no group $N'\subseteq N$ of voters and integer $0<r\leq k$ such that:
%\begin{itemize}
%\item[a)] $\sum_{n\in N'} s_n \geq \frac{r}{k} \sum_{n\in N}s_n$,
%\item[b)] $|\cap_{n\in N'} C_n|\geq r$, and
%\item[c)] $|A\cap (\cup_{n\in N'} C_n)|<r$.
%\end{itemize}
%
$$\text{a) } \sum_{n\in N'} s_n \geq \frac{r}{k} \sum_{n\in N}s_n, \quad \quad \text{b) } |\cap_{n\in N'} C_n|\geq r, 
\quad \quad \text{and c) } |A\cap (\cup_{n\in N'} C_n)|<r.$$
%
In words, if there is a group $N'$ of voters with at least $r$ commonly approved candidates, and enough aggregate vote strength to provide each of them with a vote support of value $\hat{t}:=\sum_{n\in N} s_n / k$, then this group has a justified right to be represented by at least $r$ members in committee $A$, though not necessarily commonly approved. 
Notice that $\hat{t}$ is an upper bound on the average vote support that voter set $N$ can possibly provide to any committee of $k$ members. 

\emph{Maximin support objective.} 
For the given instance, we consider a solution consisting of a tuple $(A,w)$, where $A\subseteq C$ is a committee of $k$ elected candidates, and $w\in\R^E$ is a vector of non-negative edge weights that represents a fractional distribution of each voter's vote among her approved candidates.%
\footnote{This weight vector is related to the notions of \emph{support distribution function} in~\cite{sanchez2016maximin} and \emph{price system} in~\cite{peters2019proportionality}. In particular, all voting rules considered in this paper are \emph{priceable}, as defined in~\cite{peters2019proportionality}.} 
For instance, for voter $n$ this distribution may assign a third of $s_n$ to $c_1$ and two thirds of $s_n$ to $c_2$, where $c_1, c_2\in C_n$.
Vector $w$ is considered \emph{feasible}%
\footnote{Intuitively, a feasible solution $(A,w)$ should also observe $w_{nc}=0$ for each edge $nc$ with $c\not\in A$. 
However, as this constraint can always be enforced in post-computation, we ignore it so that the feasibility of a vector $w$ is independent of any committee.} 
if  % 
%
\begin{equation}
    \sum_{c\in C_n} w_{nc}\leq s_n \quad \text{ for each voter } n\in N. \label{eq:feasible}
\end{equation}

In our analyses, we will also consider \emph{partial} committees, with $|A|\leq k$. If $|A|=k$, we call it \emph{full}. 
All solutions $(A,w)$ in this paper are assumed to be feasible and full unless stated otherwise. 
Given a (possibly partial, unfeasible) solution $(A,w)$, we define the \emph{support} over the committee members as 
\begin{equation}
supp_w(c):=\sum_{n\in N_c} w_{nc} \quad \text{for each $c\in A, \quad$ and } \quad supp_w(A):=\min_{c\in A} supp_w(c), \label{eq:support}
\end{equation}
where we use the convention that $supp_w(\emptyset)=\infty$ for any weight vector $w\in\R^E$. 
The maximin support objective, introduced in~\cite{sanchez2016maximin}, asks to maximize the least member support $supp_w(A)$ over all feasible full solutions $(A,w)$. 

\emph{Balanced solutions.}
For a fixed committee $A$, a feasible weight vector $w\in\R^E$ that maximizes $supp_w(A)$ can be found efficiently. In this paper we seek additional desirable properties on a weight vector which can still be achieved efficiently. We say that a feasible $w\in\R^E$ is \emph{balanced for $A$}, or that $(A,w)$ is a balanced solution, if
\begin{enumerate}
    \item it maximizes the sum of member supports, $\sum_{c\in A} supp_w(c)$, over all feasible weight vectors, and 
    \item it minimizes the sum of supports squared, $\sum_{c\in A} (supp_w(c))^2$, over all vectors that observe the point above. 
\end{enumerate}

In other words, a balanced weight vector maximizes the sum of supports and then minimizes their variance. 
In the next lemma, whose proof is delayed to Appendix~\ref{s:proofs}, we establish some key properties that we exploit in our analyses. 

\begin{lemma}\label{lem:balanced}
Let $(A,w)$ be a balanced, possibly partial solution. Then,
\begin{enumerate}
    \item for each $1\leq r\leq |A|$, vector $w$ simultaneously maximizes $\min_{A'\subseteq A, |A'|=r} \ \sum_{c\in A'} supp_{w'}(c)$ over all feasible weight vectors $w'\in\R^E$; 
		\item the sum of member supports must be exactly $\sum_{c\in A} supp_w(c)=\sum_{n\in \cup_{c\in A} N_c} s_n$; and
    \item if there is a voter $n\in N$ and candidates $c\in C_n$, $c'\in C_n\cap A$ with $supp_w(c)>supp_w(c')$, then $w_{nc}=0$.
\end{enumerate}
Furthermore, a feasible solution $(A,w)$ is balanced if and only if it observes properties 2 and 3 above.
\end{lemma}

Notice that by setting $r=1$ on the first point, we obtain that balanced vector $w$ indeed maximizes the least member support $supp_w(A)$ over all feasible weight vectors. 
More generally, for each $r$ the quantity defined in the first point defines a lower bound on the cost for an adversary to get $r$ representatives in the validator committee in NPoS, so maximizing these objectives for all thresholds $r$ aligns with our security objective as it makes any attack as costly as possible. 
The second point follows from the fact that the sum of member supports is maximal, so the feasibility inequality~\eqref{eq:feasible} must be tight for each voter $n$ with approved candidates in $A$. The third point is a consequence of having the supports as evenly distributed as possible within $A$: if candidate $c$ has a higher support than another candidate $c'$, then none of $n$'s vote can go to $c$, as all of it must have been assigned to $c'$ or to other members with lower support. 

In Appendix~\ref{s:balanced} we present new algorithms for computing a balanced weight vector for a given committee $A$. 
In particular, we prove that one can be found in time $O(|E|\cdot k + k^3)$ using parametric flow techniques, which to the best of our knowledge is the current fastest algorithm in the literature even for the simpler problem of maximizing $supp_w(A)$.

\begin{remark}\label{rem:bal}
In the remainder of the paper, we denote by $\bal$ the time complexity of finding a balanced weight vector, which will depend on the precise algorithm used.
\end{remark}

\emph{Equivalence of objectives.} 
We now establish that our security objective~\eqref{eq:security} is indeed equivalent to maximin support. 
For this we use the fact that, in view of point 1 in Lemma~\ref{lem:balanced}, in the maximin support objective we can reduce the solution space to only balanced solutions without loss of generality. The proof of the next lemma is delayed to Appendix~\ref{s:proofs}.

\begin{lemma} \label{lem:equivalence} If $(A,w)$ is a balanced solution, then
$$supp_w(A) = \min_{\emptyset\neq A' \subseteq A} \frac{1}{|A'|} \sum_{n\in \cup_{c\in A'} N_c} s_n.$$
Consequently, maximin support, the problem of maximizing the left-hand side over all balanced full solutions $(A,w)$, 
is equivalent to the problem of maximizing the right-hand side over all full committees $A$. 
Furthermore, this equivalence preserves approximations, as any balanced solution $(A,w)$ provides the same objective value to both problems.
\end{lemma}

\emph{Network flows.}
In many proofs we deal with a vector $f\in\mathbb{R}^{E}$ of edge weights over the input graph $G=(N\cup C,E)$, which we regard as a vector of flows with positive signs considered to be flow directed toward $C$, and negative signs as flow directed toward $N$. 
Consequently, the \emph{excess} of a voter $n\in N$ relative to $f$ is $f(n):=\sum_{c\in C_n} f_{nc}$, and the excess of a candidate $c\in C$ is $f(c):= -\sum_{n\in N_c} f_{nc}$. 
A set of vertices $S\subseteq N\cup C$ has \emph{net excess} if $\sum_{x\in S} f(x)>0$, and it has \emph{net demand} if $\sum_{x\in S} f(x)<0$.    
A vector $f'\in\mathbb{R}^E$ is a \emph{sub-flow of $f$} if a) for each edge $e\in E$ with $f'_e\neq 0$, flows $f'_e$ and $f_e$ have the same sign and $|f'_e|\leq |f_e|$, and b) for each vertex $x\in N\cup C$ with $f'(x)\neq 0$, excesses $f'(x)$ and $f(x)$ have the same sign and $|f'(x)|\leq |f(x)|$. 
The proof of the next lemma is delayed to Appendix~\ref{s:proofs}.

\begin{lemma}\label{lem:subflow}
If weight vectors $w, w'\in\R^E$ are non-negative and feasible for the given instance, and $f'\in\mathbb{R}^E$ is a sub-flow of $f:=w'-w$, then both $w+f'$ and $w'-f'$ are non-negative and feasible as well.
\end{lemma}

\begin{remark}
In all algorithms analyzed, we assume that all numerical operations take constant time.
\end{remark}
